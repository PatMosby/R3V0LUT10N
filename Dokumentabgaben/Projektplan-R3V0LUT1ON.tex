\documentclass[fontsize=12pt,paper=a4,twoside]{scrartcl}

%Befehl um ToDo-Vermerke zu erstellen
%Autor: Stefan Macke
%Quelle: http://blog.stefan-macke.com/2007/04/23/todo-befehl-in-latex/
\newcommand{\todo}[1]{\textbf{\textsc{\textcolor{red}{(TODO: #1)}}}}

\newcommand{\grad}{\ensuremath{^{\circ}} }
\renewcommand{\strut}{\vrule width 0pt height5mm depth2mm}

\usepackage[utf8]{inputenc}
\usepackage[final]{pdfpages}
% obere Seitenränder gestalten können
\usepackage{fancyhdr}
\usepackage{moreverb}
% Graphiken als jpg, png etc. einbinden können
\usepackage{graphicx}
\usepackage{stmaryrd}
% Floats Objekte mit [H] festsetzen
\usepackage{float}
% setzt URL's schön mit \url{http://bla.laber.com/~mypage}
\usepackage{url}
% Externe PDF's einbinden können
\usepackage{pdflscape}
% Verweise innerhalb des Dokuments schick mit " ... auf Seite ... "
% automatisch versehen. Dazu \vref{labelname} benutzen
%\usepackage[ngerman]{varioref}
%\usepackage[ngerman]{babel}
%\usepackage{ngerman}
% Bibliographie
%\usepackage{bibgerm}
% Tabellen
%\usepackage{tabularx}
%\usepackage{supertabular}
%\usepackage[colorlinks=true, pdfstartview=FitV, linkcolor=blue,
%            citecolor=blue, urlcolor=blue, hyperfigures=true,
%            pdftex=true]{hyperref}
%\usepackage{bookmark}

% Damit Latex nicht zu lange Zeilen produziert:
\sloppy
%Uneinheitlicher unterer Seitenrand:
%\raggedbottom

% Kein Erstzeileneinzug beim Absatzanfang
% Sieht aber nur gut aus, wenn man zwischen Absätzen viel Platz einbaut
\setlength{\parindent}{0ex}

% Abstand zwischen zwei Absätzen
\setlength{\parskip}{1ex}

% Seitenränder für Korrekturen verändern
\addtolength{\evensidemargin}{-1cm}
\addtolength{\oddsidemargin}{1cm}

\bibliographystyle{gerapali}

% Lustige Header auf den Seiten
  \pagestyle{fancy}
  \setlength{\headheight}{70.55003pt}
  \fancyhead{}
  \fancyhead[LO,RE]{Software--Projekt 2\\ WiSe 2013/2014
  \\Projektplan}
  \fancyhead[LE,RO]{Seite \thepage\\\slshape \leftmark\\\slshape \rightmark}

%
% Und jetzt geht das Dokument los....
%

\begin{document}

% Lustige Header nur auf dieser Seite
  \thispagestyle{fancy}
  \fancyhead[LO,RE]{ }
  \fancyhead[LE,RO]{Universität Bremen\\FB 3 -- Informatik\\
  Prof. Dr. Rainer Koschke \\TutorIn: Sabrina Wilske}
  \fancyfoot[C]{}

% Start Titelseite
  \vspace{3cm}

  \begin{minipage}[H]{\textwidth}
  \begin{center}
  \bf
  \Large
  Software--Projekt 2 2013/2014\\
  \smallskip
  \small
  VAK 03-BA-901.02\\
  \vspace{3cm}
  \end{center}
  \end{minipage}
  \begin{minipage}[H]{\textwidth}
  \begin{center}
  \vspace{1cm}
  \bf
  \Large Projektplan\\
  \vfill
  \end{center}
  \end{minipage}
  \vfill
  \begin{minipage}[H]{\textwidth}
  \begin{center}
  \sf
  \begin{tabular}{lrr}
  Sebastian Bredehöft & sbrede@tzi.de & 2751589\\
  Patrick Damrow & damsen@tzi.de & 2056170\\
  Tobias Dellert & to\_de@uni-bremen.de & 2936941\\
  Tim Ellhoff & tellhoff@tzi.de & 2520913\\
  Daniel Pupat & dpupat@tzi.de & 2703053\\
  Mohamadreza (Amir) Khostevan & amirkh@tzi.de & 1234567\\
  \end{tabular}
  \\ ~
  \vspace{2cm}
  \\
  \it Abgabe: 20. Oktober. 2013 --- Version 1.0\\ ~
  \end{center}
  \end{minipage}

% Ende Titelseite

% Start Leerseite

\newpage

  \thispagestyle{fancy}
  \fancyhead{}
  \fancyhead[LO,RE]{Software--Projekt \\  2013/2014
  \\Projektplan}
  \fancyhead[LE,RO]{Seite \thepage\\\slshape \leftmark\\~}
  \fancyfoot{}
  \renewcommand{\headrulewidth}{0.4pt}
  \tableofcontents

\newpage

  \fancyhead[LE,RO]{Seite \thepage\\\slshape \leftmark\\\slshape \rightmark}


%%%%%%%%%%%%%%%%%%%%%%%%%%%%%%%%%%%%%%%%%%%%%%%%%%%%%%%%%%%%%%%%%%%%%%%%
\section*{Version und Änderungsgeschichte}

{\em Die aktuelle Versionsnummer des Dokumentes sollte eindeutig und gut zu
identifizieren sein, hier und optimalerweise auf dem Titelblatt.}

\begin{tabular}{ccl}
Version & Datum & Änderungen \\
\hline
1.0 & 20.10.2013 & Erste veröffentlichte Version. \\
\end{tabular}


%%%%%%%%%%%%%%%%%%%%%%%%%%%%%%%%%%%%%%%%%%%%%%%%%%%%%%%%%%%%%%%%%%%%%%%%
\section{Einleitung}

\subsection{Projektübersicht}


\subsubsection{Ziele}
Das Ziel unserer Gruppe IT\_R3V0LUT10N ist es, das Softwareprojekt 2 der Universität Bremen zu bestehen. Dies setzt die Einhaltung der Fristen und Termine, eine ausreichende Fertigstellung des Projekts und die Abgabe aller in SWP2 geforderten Dokumente wie Projektplan, Anforderungsspezifikation und Angebot, Architekturbeschreibung, Schnittstellenbeschreibung, Testplan inklusive Blackbox-Tests und ein elektronisch geführtes Berichtsheft voraus. Darüber hinaus wollen wir einen GUI-Prototypen erstellen und den Akzeptanztest bestehen. Ein Bibliothekssystem zu erstellen steht aber im Vordergrund.

Das Bibliothekssystem beinhaltet sowohl eine Website, als auch einen Zugang für mobile Geräte mit kleinem Display. Ziel ist es, die Mindestanforderungen\footnote{\url{http://www.informatik.uni-bremen.de/st/Lehre/swpII_1314/mindestanforderungen.html}} und eventuell weitergehende Funktionen zu implementieren.

Zu den Mindestanforderungen gehören die Erstellung und Abgabe einer Bibliothekssoftware, eines Serverprogramms mit Datenbankanbindung,einen Administrationszugang und einen Zugang für mobile Geräte mit kleinem Display. Wir haben uns entschieden den Zugang für die mobilen Geräte in Form einer Android-App zu realisieren. Die zu erstellende Bibliothekssoftware dient in erster Linie zur Verwaltung des Medienbestandes der Oberschule Rockwinkel. Der Administrationszugang wird benötigt um Backup


\subsubsection{Hauptarbeitsaktivitäten und --produkte}

\subsubsection{Haupt--Meilensteine und grober Zeitplan}

{\em Meilensteine, jeweils mit konkretem Datum,
 Kriterien für die Erfüllung der Meilensteine.}


\subsubsection{Benötigte Ressourcen}
ENTFÄLLT
\begin{itemize}
\item \textbf{Menschliche Ressourcen}

\item \textbf{Hardware}

\item \textbf{Räume}

\dots

\end{itemize}

\subsubsection{Budget}
ENTFÄLLT
{\em Beinhaltet auch konkrete Angaben zu Entwicklerstunden und Kosten in Euro.}


\subsubsection{Kontaktdaten des Kunden}
ENTFÄLLT
\subsubsection{Mitarbeiter}
{\em Hier finden sich alle Mitarbeitenden der Gruppe mit Kontaktdaten und Foto.}

\subsection{Auszuliefernde Produkte}


\subsection{Evolution des Plans}
ENTFÄLLT
{\em Wird der Plan verändert? Wann? Wie oft? Von wem? Wenn bereits Aktualisierungen vorgesehen sind, welche sind das? Möglicherweise betrifft das die Zeitplanung, die Risikobewertung, oder andere Teile des Plans. Gibt es möglicherweise auch unvorhergesehene Aktualisierungen?}

\subsection{Referenzen}
% mit \nocite kann man Literatur auflisten, die im Text nicht explizit
% erwähnt ist. \nocite{*} zitiert dann das ganze .bib-File
%
% Die Bibliographie erzeugt man indem man erst
%
% pdflatex bericht.tex
% bibtex bericht
% pdflatex bericht.tex
% pdflatex bericht.tex
%
% benutzt
%\nocite{Knudsen1}
%\nocite{*}
%\bibliography{literatur}

% Das renewcommand verhindert dass für die Literatur eine section* angelegt wird.
% auftaucht
{\renewcommand\section[2]{}
\bibliography{referenzen}
}

\subsection{Definitionen und Akronyme}
ENTFÄLLT
{\em Hier sollen Begriffe definiert werden, die nötig sind, um den
  Projektplan zu verstehen. Diese kommen insbesondere aus der Welt des
  Kunden (Projektdomäne) und der Welt des Softwareproduzenten.}

\section{Projektorganisation}
ENTFÄLLT
\subsection{Prozessmodell}
ENTFÄLLT
\subsection{Organisationsstruktur}
ENTFÄLLT
{\em Genaue Beschreibung der Rollen, Rechte und Pflichten!}

{\em z.B. auch regelmäßiges Treffen im Chat, Einrichtung einer
  Groupware oder eines Forums, o.ä. \dots}

\subsection{Organisationsgrenzen und --schnittstellen}
ENTFÄLLT
{\em Hierher gehören auch evtl. Kontaktpersonen für Fremdbibliotheken u.ä.}

\subsection{Verantwortlichkeiten}
ENTFÄLLT
\section{Managementprozess}

\subsection{Managementprozess und --prioritäten}

\subsection{Annahmen, Abhängigkeiten und Einschränkungen}

\subsection{Risikomanagement}\label{riskmanagement}

{\em Wenn Ihr Euch entschieden habt, bestimmte vorbeugende Maßnahmen 
     durchzuführen, solltet Ihr dies deutlich kennzeichnen. Hoffentlich
     haben diese Maßnahmen dann einen Einfluss auf Eintrittswahrscheinlichkeit oder Schadenshöhe (zum Beispiel
     ist die Eintrittswahrscheinlichkeit von komplettem Datenverlust durch regelmäßige Backups deutlich 
     geringer). Daher solltet Ihr für diese Fälle dann die verringerten Werte für Eintrittswahrscheinlichkeit, 
     Schadenshöhe und Risikopotential zusätzlich angeben. }

{\em Wie werden neue Risiken erkannt/erfasst? Wer ist für was
  zuständig? Wie ist der Informationsfluss? \ldots 

Dieser Teil ist ein
  wichtiger Schwerpunkt des Projektplans und sollte daher ausführlich
  behandelt werden.}

\subsection{Projektüberwachung}\label{3.4-controlling}
{\em Wie wird der Projektstatus verfolgt? Wie stellt Ihr sicher, dass
  der Phasenleiter jederzeit über den Stand der Entwicklung informiert
  ist? Wie werden Probleme bzw. Verzögerungen frühzeitig erkannt und
  angegangen?}

\subsection{Mitarbeiter}
{\em Kompetenzen der und Anforderungen an die Mitarbeiter.}

%%%%%%%%%%%%%%%%%%%%%%%%%%%%%%%%%%%%%%%%%%%%%%%%%%%%%%%%%%%%%%%%%%%%%%%%

\section{Technische Prozesse}
\subsection{Methoden, Werkzeuge und Techniken}
\subsubsection{Entwicklungsplattform}
Folgende Werkzeuge werden im Entwicklungsprozess von uns benutzt:
\begin{itemize}

\item{Eclipse\footnote{\url{http://www.eclipse.org/}} ist unsere Entwicklungsumgebung (beinhaltet AndroidSDK\footnote{\url{https://developer.android.com/sdk/index.html}} für die Androidentwicklung}
\item{Maven\footnote{\url{http://maven.apache.org/}} ist unser Build-Management Tool}
\item{GlassFish 3.1\footnote{\url{http://glassfish.java.net/}} ist unser Application-Server}
\item{jUnit\footnote{\url{http://junit.org/}} ist unser Framework zum testen}
\item{GitHub\footnote{\url{http://github.com/}} zur Versionsverwaltung}
\item{GanttProject\footnote{\url{http://www.ganttproject.biz/}}} für Gantt-Diagramme
\item{\LaTeX{}\footnote{\url{http://www.latex-project.org/}}} zur Dokumentenerstellung

\end{itemize}
\todo{Sind das alle?}

\subsubsection{Entwicklungsmethode}

\todo{Abhängigkeit von Prozessmodell, Mittwoch abstimmen}

\subsubsection{Programmiersprache und Bibliotheken}
Die Programmiersprache wird Java  (mindestens Version 5) sein. Außerdem wird in geringen Umfang HTML und XML benutzt.\\
Ob und welche Bibliotheken genutzt werden, kann zu diesem Zeitpunkt(\emph{Abgabe: 20.10.2013}) nicht gesagt werden.\\
\\
Sobald wir Bibliotheken nutzen, wird dieser Punkt aktualisiert.
\subsection{Dokumentationsplan}
Wir werden als Ergebnis verschiedene Dokumentationen vorweisen können. Diese sind:

\begin{itemize}
\item{Nutzerhandbuch}
\item{Installationsanleitung}
\item{Dokumentation des Quellcodes}
\end{itemize}

\subsubsection{Codingstyle}
Unsere Implementierungen werden sich an die \emph{Code Conventions for the Java Programming Language}\footnote{\url{http://www.oracle.com/technetwork/java/codeconv-138413.html}} halten.

\subsubsection{Kommentarsprache}
Die Sprache in der unsere Kommentare verfasst sind, wird Deutsch sein. Dies verhindert mögliche Missverständnisse innerhalb unserer Gruppe.

\subsubsection{JavaDoc}
Wir benutzen JavaDoc zur Dokumentation unseres Quellcodes. Dieses lässt eine einfache Erstellung von HTML-Dokumentationsdateien zu.
Zur zukünftigen Wartung wird bis auf triviale Codezeilen, der komplette Code in JavaDoc dokumentiert.
\subsubsection{Begleitende Dokumentation}
\todo{Weitere Erläuterung zur Doku?}
\subsection{Unterstützende Projektfunktionen}
In Abschnitt \todo{Abschnitt nennen} werden unsere Phasenleiter benannt, die für die jeweiligen Phasen verantwortlich sind und als Ansprechpartner und Leiter dienen.\\
Unsere Projektdateien stehen jederzeit auf \emph{GitHub}\footnote{\url{asd}} zur Verfügung. Außerdem werden regelmäßig von \emph{jedem} Gruppenmmitglied Datensicherungen, in Form von Updates und Backups des Repositorys auf dem eigenen Rechner stattfinden.\\

%{\em Gibt es Maßnahmen zur Qualitätssicherung? Wer ist zuständig?
%  Wieviel Zeit ist dafür vorgesehen?}

\todo{Qualitätssicherung}

%%%%%%%%%%%%%%%%%%%%%%%%%%%%%%%%%%%%%%%%%%%%%%%%%%%%%%%%%%%%%%%%%%%%%%%%

\section{Arbeitspakete, Zeitplan und Budget}

{\em Dieser Teil ist ein zweiter Schwerpunkt des Projektplans. Hier sollt Ihr die nächste Phase detailliert planen (siehe Arbeitspakete). Die weiteren Phasen sollen ebenfalls wenigstens grob geplant werden. Ein Gantt-Diagramm ist zwingend! 

Ihr sollt den Plan in der kommenden Phase auch tatsächlich benutzen -- und so
  Erfahrungen sammeln, was evtl. bei der Planung unberücksichtigt
  blieb. Bei der nächsten Zeitplanung (für die nächste Phase) bekommt
  Ihr dann evtl.\ eine noch bessere Planung hin.}

\subsection{Arbeitspakete}\label{aps}


{\em Besonderen Wert legen wir auf die Granularität der APs. Diese
  sollten von 1-2 Personen in max. einer Woche Zeitdauer (kalendarisch, nicht
  Aufwand) bearbeitbar sein. Die Beschreibungen sollten so genau sein,
  dass der Bearbeiter damit genau weiß, was zu tun ist.}

\subsection{Zeitplan und Abhängigkeiten}

{\em Die Abhängigkeiten zwischen Arbeitspaketen oder Meilensteinen müssen genannt werden, sowie im
  Gantt-Diagramm eingezeichnet werden. Der kritische Pfad soll
  angegeben und/oder eingezeichnet werden!}

\subsection{Ressourcenanforderung}

{\em Jedem Arbeitspaket muss mind.\ ein Bearbeiter zugeordnet
  werden. Die Zuordnung der ganzen Gruppe sollte nur in Ausnahmefällen
  erfolgen -- und dann vermutlich begründet werden!}


%%%%%%%%%%%%%%%%%%%%%%%%%%%%%%%%%%%%%%%%%%%%%%%%%%%%%%%%%%%%%%%%%%%%%%%%
\section{Sonstige Elemente}
ENTFÄLLT
\subsection{Pläne für die Konvertierung von Daten}
ENTFÄLLT

\subsection{Managementpläne für Unterauftragsnehmer}
ENTFÄLLT
{\em Wenn Fremdbibliotheken benutzt werden\dots}

\subsection{Ausbildungspläne}
ENTFÄLLT
{\em Hierunter fallen z.B. auch interne Schulungen, die Ihr
  durchführen wollt.}

\subsection{Raumpläne}
ENTFÄLLT
\dots

\subsection{Installationspläne}
ENTFÄLLT
\dots

\subsection{Pläne für die Übergabe des Systems}
ENTFÄLLT
\dots


\end{document}
