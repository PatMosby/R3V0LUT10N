\documentclass[fontsize=12pt,paper=a4,twoside]{scrartcl}

%Befehl um ToDo-Vermerke zu erstellen
%Autor: Stefan Macke
%Quelle: http://blog.stefan-macke.com/2007/04/23/todo-befehl-in-latex/
\newcommand{\todo}[1]{\textbf{\textsc{\textcolor{red}{(TODO: #1)}}}}

\newcommand{\grad}{\ensuremath{^{\circ}} }
\renewcommand{\strut}{\vrule width 0pt height5mm depth2mm}

\usepackage[utf8]{inputenc}
\usepackage[final]{pdfpages}
% obere Seitenränder gestalten können
\usepackage{fancyhdr}
\usepackage{moreverb}
% Graphiken als jpg, png etc. einbinden können
\usepackage{graphicx}
\usepackage{stmaryrd}
% Floats Objekte mit [H] festsetzen
\usepackage{float}
% setzt URL's schön mit \url{http://bla.laber.com/~mypage}
\usepackage{url}
% Externe PDF's einbinden können
\usepackage{pdflscape}
% Verweise innerhalb des Dokuments schick mit " ... auf Seite ... "
% automatisch versehen. Dazu \vref{labelname} benutzen
\usepackage[ngerman]{varioref}
\usepackage[ngerman]{babel}
\usepackage{ngerman}
 %Bibliographie
\usepackage{bibgerm}
% Tabellen
\usepackage{tabularx}
\usepackage{supertabular}
\usepackage[colorlinks=true, pdfstartview=FitV, linkcolor=blue,
            citecolor=blue, urlcolor=blue, hyperfigures=true,
            pdftex=true]{hyperref}
%\usepackage{bookmark}

% Damit Latex nicht zu lange Zeilen produziert:
\sloppy
%Uneinheitlicher unterer Seitenrand:
\raggedbottom

% Kein Erstzeileneinzug beim Absatzanfang
% Sieht aber nur gut aus, wenn man zwischen Absätzen viel Platz einbaut
\setlength{\parindent}{0ex}

% Abstand zwischen zwei Absätzen
\setlength{\parskip}{1ex}

% Seitenränder für Korrekturen verändern
\addtolength{\evensidemargin}{-1cm}
\addtolength{\oddsidemargin}{1cm}

\bibliographystyle{gerapali}

% Lustige Header auf den Seiten
  \pagestyle{fancy}
  \setlength{\headheight}{70.55003pt}
  \fancyhead{}
  \fancyhead[LO,RE]{Software--Projekt 2\\ WiSe 2013/2014
  \\Projektplan}
  \fancyhead[LE,RO]{Seite \thepage\\\slshape \leftmark\\\slshape \rightmark}

%
% Und jetzt geht das Dokument los....
%

\begin{document}

% Lustige Header nur auf dieser Seite
  \thispagestyle{fancy}
  \fancyhead[LO,RE]{ }
  \fancyhead[LE,RO]{Universität Bremen\\FB 3 -- Informatik\\
  Prof. Dr. Rainer Koschke \\TutorIn: Sabrina Wilske}
  \fancyfoot[C]{}

% Start Titelseite
  \vspace{3cm}

  \begin{minipage}[H]{\textwidth}
  \begin{center}
  \bf
  \Large
  Software--Projekt 2 2013/2014\\
  \smallskip
  \small
  VAK 03-BA-901.02\\
  \vspace{3cm}
  \end{center}
  \end{minipage}
  \begin{minipage}[H]{\textwidth}
  \begin{center}
  \vspace{1cm}
  \bf
  \Large Projektplan\\
  \vfill
  \end{center}
  \end{minipage}
  \vfill
  \begin{minipage}[H]{\textwidth}
  \begin{center}
  \sf
  \begin{tabular}{lrr}
  Sebastian Bredehöft & sbrede@tzi.de & 2751589\\
  Patrick Damrow & damsen@tzi.de & 2056170\\
  Tobias Dellert & tode@tzi.de & 2936941\\
  Tim Ellhoff & tellhoff@tzi.de & 2520913\\
  Daniel Pupat & dpupat@tzi.de & 2703053\\
  Mohamadreza (Amir) Khostevan & amirkh@tzi.de & 1234567\\
  \end{tabular}
  \\ ~
  \vspace{2cm}
  \\
  \it Abgabe: 20. Oktober. 2013 --- Version 1.0\\ ~
  \end{center}
  \end{minipage}

% Ende Titelseite

% Start Leerseite

\newpage

  \thispagestyle{fancy}
  \fancyhead{}
  \fancyhead[LO,RE]{Software--Projekt \\  2013/2014
  \\Projektplan}
  \fancyhead[LE,RO]{Seite \thepage\\\slshape \leftmark\\~}
  \fancyfoot{}
  \renewcommand{\headrulewidth}{0.4pt}
  \tableofcontents

\newpage

  \fancyhead[LE,RO]{Seite \thepage\\\slshape \leftmark\\\slshape \rightmark}


%%%%%%%%%%%%%%%%%%%%%%%%%%%%%%%%%%%%%%%%%%%%%%%%%%%%%%%%%%%%%%%%%%%%%%%%
\section*{Version und Änderungsgeschichte}

{\em Die aktuelle Versionsnummer des Dokumentes sollte eindeutig und gut zu
identifizieren sein, hier und optimalerweise auf dem Titelblatt.}

\begin{tabular}{ccl}
Version & Datum & Änderungen \\
\hline
0.1 & 14.10.2013 & Ziele hinzugefügt.\\
0.1.1 & 14.10.2013 & Ziele vervollständigt\\
0.2 & 15.10.2013 & Hauptarbeitsaktivitäten und -produkte hinzugefügt.\\
0.3 & 15.10.2013 & Meilensteine eingefügt.\\
0.4 & 15.10.2013 & Benötigte Ressourcen -Menschen hinzugefügt.\\
0.5 & 16.10.2013 & Ressourcen ergänzt -Hardware und -Räume.\\
1.0 & 20.10.2013 & Erste veröffentlichte Version. \\
%1.1 & TT.MM.JJJJ & Zeitplanung für die Anforderungsspezifikation hinzugefügt. \\
%1.2 & TT.MM.JJJJ & .... 
\end{tabular}


%%%%%%%%%%%%%%%%%%%%%%%%%%%%%%%%%%%%%%%%%%%%%%%%%%%%%%%%%%%%%%%%%%%%%%%%
\section{Einleitung}

Dieses Dokument ist der Projektplan der Gruppe IT\_R3V0LUT10N im des Softwareprojekts im Wintersemester 2013/2014 an der Universität Bremen. Der Projektplan entspricht der Struktur ANSI/IEEE Std. 1058.1-1987\footnote{\url{http://ieeexplore.ieee.org/stamp/stamp.jsp?tp=&arnumber=25325&userType=inst}}.

\subsection{Projektübersicht}


\subsubsection{Ziele}
Das Ziel unserer Gruppe IT\_R3V0LUT10N ist es, das Softwareprojekt 2 der Universität Bremen zu bestehen. Dies setzt die Einhaltung der Fristen und Termine, eine aus-reichende Fertigstellung des Projekts und die Abgabe aller in SWP2 geforderten Dokumente wie Projektplan, Anforderungsspezifikation und Angebot, Architekturbeschreibung, Schnittstellenbeschreibung, Testplan inklusive Blackbox-Tests und ein elektronisch geführtes Berichtsheft voraus. Darüber hinaus wollen wir einen GUI-Prototypen erstellen und den Akzeptanztest bestehen. Ein Bibliothekssystem zu erstellen steht aber im Vordergrund.

Das Bibliothekssystem beinhaltet sowohl eine Website, als auch einen Zugang für mobile Geräte mit kleinem Display. Ziel ist es, die \label{sec:minreq}Mindestanforderungen\footnote{\url{http://www.informatik.uni-bremen.de/st/Lehre/swpII_1314/mindestanforderungen.html}} und eventuell weitergehende Funktionen zu implementieren.

Zu den Mindestanforderungen gehören die Erstellung und Abgabe einer Bibliothekssoftware, eines Serverprogramms mit Datenbankanbindung, einen Administrationszugang und einen Zugang für mobile Geräte mit kleinem Display. Wir haben uns entschieden den Zugang für die mobilen Geräte in Form einer Android-App zu realisieren, weil wir dies für Zeitgemäß und einfacher für den Leser, an den diese Form des Zugangs gerichtet ist, halten. Die zu erstellende Bibliothekssoftware dient in erster Linie zur Verwaltung des Medienbestandes der Bibliothek und dem Verleihen dieser Medien an der Oberschule Rockwinkel. Der Administrationszugang wird benötigt um Bibliothekare anzulegen, zu löschen, deren Stammdaten zu ändern, oder deren Rechte innerhalb der Software zu verändern.\\ \todo{Weitere Features sind den oben genannten Mindestanforderungen zu entnehmen, oder werden im weiteren Verlauf des Projektplans erläutert.}

\subsubsection{Hauptarbeitsaktivitäten und -produkte}
\todo{Beschreibung}
\begin{table}[htbp]
\caption{Hauptaktivitäten und -produkte}
\centering
\begin{tabular}{p{7cm}|p{7cm}}
\hline Aktivität & Arbeitsprodukt \\ 
\hline Projektplanung & Projektplan\\
\hline Anforderungsanalyse, Angebotserstellung & Anforderungsspezifikation, Angebot\\
\hline Entwurf (Globale Analyse, Konzeptionelles Modell, Modulblickwinkel, Ausführungsblickwinkel, Codeblickwinkel) & Architekturbeschreibung\\
\hline erstellen des Testplans, Tests & Testplan, Schnittstellentests\\
\hline Implementierung & lauffähiges Programm\\
\hline Dokumentation & Installationsanweisung/-Skript\\
\hline Auslieferung & Kunde erhält Produkt\\
\hline 
\end{tabular}
\end{table}

\subsubsection{Haupt--Meilensteine und grober Zeitplan}

\begin{description}
\item[M0 - 14.10.2013] Beginn des Projektes
\item[M1 - 20.10.2013] Abgabe initialer Projektplan\\
Jedes Mitglied muss seinen Teil fertig gestellt haben. Anschließend werden alle Einzelteile zusammengeführt und von allen auf Korrektheit geprüft.
\item[M2 - 13.11.2013] Anforderungsspezifikation (Intern) \\
Jedes Mitglied hat seinen Teil der Anforderungsspezifikation fertiggestellt. Anschließend werden die Teile zusammengeführt und von allen auf Korrektheit geprüft.
\item[M3 - 17.11.2013] Abgabe der Anforderungsspezifikation \\
Meilenstein 2 muss bereits fertig sein. Abgabe der Anforderungsspezifikation via MEMS.
\item[M4 - 18.12.2013] Architektur- und Schnittstellenbeschreibung, Testplan, Tests (Intern)\\
Jedes Mitglied muss seine Aufgaben erfüllt haben. Teile werden zusammengeführt und kontrolliert. Tests müssen implementiert sein.
\item[M5 - 22.12.2013] Architekturbeschreibung, Testplan und Schnittstellentests fertig\\
Meilenstein 4 muss bereits erreicht worden sein. Tests wurden lauffähig implementiert. Abgabe via MEMS.
\item[M6 - 26.01.2014] Erste lauffähige Basisversion\\
Jedes Mitglied muss seine Arbeitspakete fertig gestellt haben. Das Team muss die oben genannten Mindestanforderungen\todo{referenz auf fußnote}, die mit einem * markiert sind, implementiert haben
\item[M7 - 23.02.2014] Vollständige Abgabe der Dokumente und der Software \\
Die Software muss lauffähig und vollständig implementiert sein, \\
Abgabe des Build-/Installationsskriptes
\end{description}

\subsubsection{Benötigte Ressourcen}

\begin{itemize}
\item \textbf{Menschliche Ressourcen}

An Menschlichen Ressourcen stehen sechs Informatik Studenten der Universität Bremen zur Verfügung. Wir haben als durchschnittliche Arbeitszeit pro Woche und Person einen Aufwand von ca. 14,5 Stunden für das Projekt errechnet. Dieser Wert ergibt sich folgendermaßen:\\
Für das Modul Software Projekt 2 gibt es 9CP. 1CP enstpricht 30 Semesterstunden. 9 x 30 = 270 Stunden. Da wir 19 Wochen lang an dem Projekt arbeiten werden, ergibt sich ein aufgerundeter Wert von 14,5 Stunden pro Woche (270 / 19 = 14,21). Unsere Kontaktdaten sind dem Punkt Mitarbeiter zu entnehmen.\todo{Referenz}

\item \textbf{Hard-/ und Software}

Jedes unserer Mitglieder ist im Besitz, oder hat Zugriff, auf Computer, die folgenden Anforderungen und Verfügbarkeiten gerecht werden müssen:

\begin{itemize}
\item zum Anfertigen der Dokumente wird ein Textsatzprogramm benötigt (\LaTeX wird bevorzugt).
\item für die Entwicklung der Software müssen Java-Runtime, ein Texteditor und eine Entwicklungsumgebung mit Android-SDK installiert sein.
\item Git wird zum gleichzeitigen Bearbeiten der Dokumente und zum Datenaustausch der Entwickler benötigt.
\end{itemize}

\item \textbf{Räume}

Das Team wird sich während der gesamten Projektlaufzeit Montags, soweit verfügbar, in einer der Lerninseln im GW2 A2370 oder A3440 der Universität Bremen von 10 Uhr bis 14 Uhr treffen. Weitere spezielle Räumlichkeiten werden nicht benötigt, da wir den Kontakt regelmäßig via Skype oder Email gewährleisten.

\end{itemize}

\subsubsection{Budget}

Ein Budget für dieses Projekt in Form von Geld entfällt, da die Software im Rahmen des Moduls Software Projekt 2 entwickelt wird. Wenn wir über 19 Wochen (vom 14.10.2013 bis zum 23.02.2014) an dem Projekt mit 6 Studenten 14,5 Stunden pro Woche arbeiten, ergibt sich eine Gesamtsumme von 1653 Entwicklerstunden (19 x 6 x 14,5 = 1653).

Wir entnehmen einer Studie von Gulp \footnote{\url{http://www.gulp.de/presse/pressemitteilungen/marktstudie-freiberufliche-software-entwickler-sind-gefragt.html}} das zwei Drittel der Software-Entwickler zwischen 60 und 80 Euro fordern. Da wir alle Studenten sind und somit noch in der Ausbildung, setzen wir den Studenlohn für jeden Entwickler bei 40 Euro an. Somit würden sich für den Arbeitsaufwand der Entwicklerstunden Kosten von insgesamt 66.120 Euro ergeben.

\subsubsection{Kontaktdaten des Kunden}

{\em Oberschule Rockwinkel\\
	Uppe Angst 31 \\
	28355 Bremen \\
	Telefon: 0421 - 361 16 627 \\
	Fax: 0421 - 361 16 637 \\
	E-Mail: 416@bildung.bremen.de\\}

\subsubsection{Mitarbeiter}

\begin{table}[htbp]
\begin{tabular}{|c|c|c|}
\hline 
Name & Email & Foto\\ \hline
Bredehöft, Sebastian & sbrede@tzi.de & bild einfügen \\ \hline
Damrow, Patrick & damsen@tzi.de & bild einfügen\\ \hline
Dellert, Tobias & tode@tzi.de & bild einfügen\\\hline
Pupat, Daniel & dpupat@informatik.uni-bremen.de & bild einfügen \\ \hline
Ellhoff, Tim & tellhoff@tzi.de & bild einfügen\\ \hline
Khostevan, Mohamadreza (Amir) & amirkh@tzi.de & bild einfügen \\ \hline
\end{tabular}
\end{table}

\subsection{Auszuliefernde Produkte}


\subsection{Evolution des Plans}
ENTFÄLLT
{\em Wird der Plan verändert? Wann? Wie oft? Von wem? Wenn bereits Aktualisierungen vorgesehen sind, welche sind das? Möglicherweise betrifft das die Zeitplanung, die Risikobewertung, oder andere Teile des Plans. Gibt es möglicherweise auch unvorhergesehene Aktualisierungen?}

\subsection{Referenzen}
% mit \nocite kann man Literatur auflisten, die im Text nicht explizit
% erwähnt ist. \nocite{*} zitiert dann das ganze .bib-File
%
% Die Bibliographie erzeugt man indem man erst
%
% pdflatex bericht.tex
% bibtex bericht
% pdflatex bericht.tex
% pdflatex bericht.tex
%
% benutzt
%\nocite{Knudsen1}
%\nocite{*}
%\bibliography{literatur}

% Das renewcommand verhindert dass für die Literatur eine section* angelegt wird.
% auftaucht
{\renewcommand\section[2]{}
\bibliography{referenzen}
}

\subsection{Definitionen und Akronyme}
ENTFÄLLT
{\em Hier sollen Begriffe definiert werden, die nötig sind, um den
  Projektplan zu verstehen. Diese kommen insbesondere aus der Welt des
  Kunden (Projektdomäne) und der Welt des Softwareproduzenten.}

\section{Projektorganisation}
ENTFÄLLT
\subsection{Prozessmodell}
ENTFÄLLT
\subsection{Organisationsstruktur}
ENTFÄLLT
{\em Genaue Beschreibung der Rollen, Rechte und Pflichten!}

{\em z.B. auch regelmäßiges Treffen im Chat, Einrichtung einer
  Groupware oder eines Forums, o.ä. \dots}

\subsection{Organisationsgrenzen und --schnittstellen}
ENTFÄLLT
{\em Hierher gehören auch evtl. Kontaktpersonen für Fremdbibliotheken u.ä.}

\subsection{Verantwortlichkeiten}
ENTFÄLLT

%%%%%%%%%%%%%%%%%%%%%%%%%%%%%%%%%%%%%%%%%%%%%%%%%%%%%%%%%%%%%%%%%%%%%%%%

\section{Managementprozess}

\subsection{Managementprozess und --prioritäten}
Folgende Managementprozesse haben bei uns die höchsten Prioritäten:
\bigskip \\
Fertigstellung des Produktes: \\
Ein Ziel von uns ist die Fertigstellung des Produktes, welches vom Kunden gefordert ist. Dabei ist wichtig alle Mindestanforderungen, die der Kunde gefordert hat, erfolgreich umgesetzt wurden.\\
Dieses Ziel hat die höchste Priorität, da dies notwendig ist, um die Veranstaltung SWP 2 zu bestehen und eine Vorraussetzung aller anderen aufgeführten Ziele ist.\\
\bigskip \\
Qualität des Produktes: \\
Ein weiteres Ziel ist es, dem Produkt eine hohe Qualität zu geben. Dies ist notwendig, damit der Kunde zufrieden ist und das Produkt später evtl. verwendet wird. Dabei ist wichtig, das neben den Mindestanforderungen, weitere Funktionen vorhanden sind und die Benutzung einfach und benutzerfreundlich ist.\\
Dieses Ziel hat eine hohe Priorität, da dies notwendig ist um den Kunden zufrieden zu stellen und eine gute Note zu erreichen.\\
\bigskip \\
Weiterentwicklung des Produktes: \\
Es ist auch wichtig das Produkt so zu entwickeln, damit dieses später bei Bedarf von anderen weiterentwickelt werden kann. Dies erfordert eine Strukturierte Implementierung.\\
Dieses Ziel hat niedrige Priorität, da wir in erster Linie das Modul bestehen wollen
\bigskip \\
Kundenzufriedenheit:\\
Es ist sehr wichtig das der Kunde später zufrieden ist, was bedeutet, dass man die Mindestanforderungen erfüllt und darüber hinaus noch weitere Features einbindet, da nur so der Kunde wirklich zufrieden ist. \\
Dieses Ziel hat für uns mittlere Priorität, da wir in erster Linie die Mindestanforderungen schaffen wollen und nur wenn noch Zeit ist weitere Features einbinden. Dieses könnte aber noch notwendig sein um eine gute Note zu erreichen.\\
\bigskip \\
Kommunikation innerhalb der Gruppe: \\
Ein wichtiger Faktor ist noch die Kommunikation innerhalb der Gruppe. Wenn man sich nicht abspricht, kann es zu Schwierigkeiten kommen, wenn z.B. ein Gruppenmitglied seinen Teil nicht rechtzeitig schafft und die anderen aber davon ausgehen.\\
Dieses Ziel hat bei uns eine hohe Priorität, da ohne Kommunikation das Projekt mit hoher Wahrscheinlichkeit scheitert.\\
\bigskip \\
Klima innerhalb der Gruppe: \\
Ein gutes Gruppenklima heißt, dass innerhalb der Gruppe alle gut miteinander aus kommen und es keinen Streit gibt. Außerdem muss man den anderen Vertrauen können, dass sie immer rechtzeitig fertig werden und bei Problemen Bescheid geben.\\
Dies hat ebenfalls eine hohe Priorität, da gerade das Vertrauen und die Zuverlässigkeit sehr wichtig sind, damit alles rechtzeitig fertig wird.
\bigskip \\
Gute Note:\\
Ziel dieser Veranstaltung ist für uns das Projekt so gut wie möglich zu bestehen. Dabei sollte jeder sein bestes geben, damit am Ende das Maximum an Punkten für die Gruppe erreicht wird.\\
Dies hat bei uns eine hohe Priorität, da wir später einen möglichst guten Abschluss haben wollen.
\bigskip \\
Kunde entscheidet sich für unser Produkt:\\
Da der Kunde am Ende der Veranstaltung ein Produkt aussuchen wir, welches dann in der Bibliothek verwendet wird, wäre es möglich das er unser Produkt wählt.\\
Dieses Ziel hat bei uns eine niedrige Priorität, da wir in erster Linie gut abschneiden wollen, aber nicht darauf hinarbeiten, unbedingt das beste Produkt der Veranstaltung zu entwickeln, da dies zu zeitaufwendig wäre.


\subsection{Annahmen, Abhängigkeiten und Einschränkungen}
\subsubsection{Annahmen}
Mindestanforderungen werden nicht verändert: \\
Die erste Annahme ist, das der Kunde die Mindestanforderungen nicht verändert. Dies bedeutet, dass es keine Möglichkeit gibt andere Mindestanforderungen auszuhandeln und der Kunde auch keine neuen stellt.\\
\bigskip \\
Deadline wird nicht verschoben: \\
Noch eine Annahme ist, dass sich die Deadlines der verschiedenen Abgaben sich unter normalen Umständen nicht verändern. Dies bedeutet, das der Kunde diese nicht vorverlegt und wir diese nicht nach hinten verlegen können. \\
\bigskip \\
Erfolgreiche Teilnahme: \\
Eine weitere Annahme ist, dass alle Gruppenmitglieder die Veranstaltung erfolgreich bestehen wollen. Da sich alle für dieses Modul eingetragen haben, kann man davon ausgehen, dass alle ihr bestes geben um diese Veranstaltung zu bestehen.\\
\bigskip \\
Grundkenntnisse in Java: \\
Man kann auch annehmen, dass alle Mitglieder Grundkenntnisse in Java haben, da alle Gruppenmitglieder bereits Praktische Informatik 1 und 2 besucht haben.\\

\subsubsection{Abhängigkeiten}

Laptop: \\
Da jeder von uns ein Laptop besitzt, werden wir diesen hauptsächlich verwenden, da so jeder mobil ist und überall weiterarbeiten kann.\\
\bigskip \\
GitHub: \\
Zum Teilen der Dokumente verwenden wir GitHub. So kann jede Person einen Teil bearbeiten und die Dokumente können dann zusammengeführt werden. \\
\bigskip \\
Glassfish: \\
Als Server verwenden wir den Glassfish Server, auf dem unser Programm später arbeitet. \\
\bigskip \\
Mitglieder: \\
Da dies eine Gruppenarbeit ist, muss jedes Gruppenmitglied ihren Teil leisten, da die Arbeit auf 6 Leute ausgelegt ist. \\
\bigskip \\
Von den eben genannten Punkten ist das Projekt abhängig, da bei einem Ausfall der Punkte Schwierigkeiten auftreten können.

\subsubsection{Einschränkungen}
Weitere belegte Module: \\
Jeder von uns belegt noch weitere Module und hat deswegen nur eine gewisse Zeit für SWP 2. Hinzu kommt noch, dass wir Mitglieder haben, die in unterschiedlichen Semestern sind, wodurch es schwierig ist einen gemeinsamen Termin zu finden. \\

\subsection{Risikomanagement}\label{riskmanagement}

{\em Wenn Ihr Euch entschieden habt, bestimmte vorbeugende Maßnahmen 
     durchzuführen, solltet Ihr dies deutlich kennzeichnen. Hoffentlich
     haben diese Maßnahmen dann einen Einfluss auf Eintrittswahrscheinlichkeit oder Schadenshöhe (zum Beispiel
     ist die Eintrittswahrscheinlichkeit von komplettem Datenverlust durch regelmäßige Backups deutlich 
     geringer). Daher solltet Ihr für diese Fälle dann die verringerten Werte für Eintrittswahrscheinlichkeit, 
     Schadenshöhe und Risikopotential zusätzlich angeben. }

{\em Wie werden neue Risiken erkannt/erfasst? Wer ist für was
  zuständig? Wie ist der Informationsfluss? \ldots 

Dieser Teil ist ein
  wichtiger Schwerpunkt des Projektplans und sollte daher ausführlich
  behandelt werden.}

\begin{center}
\begin{tabular}{|c|c|c|c|} \hline
Risiko & EW (1-10) & SH (1-10) & RH\\ \hline
Krankheitsbedingter Ausfall eines Gruppenmitglieds & 5  & 4 & 20\\ \hline
Krankheitsbedingter Ausfall mehrerer Gruppenmitglieder & 2  & 7 & 14\\ \hline
Austritt eines Gruppenmitglieds & 4 & 5 & 20\\ \hline
Austritt mehrerer Gruppenmitglieder & 1 & 8 & 8\\ \hline
Inkompetenz eines Gruppenmitglieds & 2 & 7 & 14\\ \hline
Mangelhafte Kommunikation innerhalb der Gruppe & 4 & 6 & 24\\ \hline
Auflösung/Teilung der Gruppe & 2 & 10 & 20\\ \hline
Unstimmigkeiten in der Gruppe & 2 & 5 & 10\\ \hline
Mangelnde Motivation in der Gruppe & 6 & 5 & 30\\ \hline
Zeitmangel & 6 & 6 & 36\\ \hline
Probleme mit neuen Technologien & 5 & 3 & 15\\ \hline
Ausfall von GitHub & 1 & 8 & 8\\ \hline
Ausfall des Glassfish-servers & 1 & 8 & 8\\ \hline
\end{tabular}
\end{center}
EW = Eintrittswahrscheinlichkeit(Skala 1:gering - 10:hoch)\\
SH = Schadenshöhe (Skala 1:gering - 10:hoch)\\
RH = Risikohöhe (EW * SH)\\
\bigskip \\
\textbf{Krankheitsbedingter Ausfall eines/mehrerer Gruppenmitglieds/er:}\\
Aufgrund von Krankheiten fallen eine oder mehrere Personen aus und können nicht mehr richtig oder für eine gewisse Zeit überhaupt nicht mehr mitarbeiten. Dadurch kommt auf die restliche Gruppe mehr Arbeit zu.\\
\textbf{Maßnahmen:}\\
1. Gruppenmitglied benachrichtigt die anderen Mitglieder so früh wie möglich, damit diese sich darauf einstellen können.\\
2. Die Gruppe sucht Gespräch mit dem Tutor wenn mehrere Personen ausfallen.\\
\bigskip \\
\textbf{Austritt eines/mehrerer Gruppenmitglieds/er:}\\
Aufgrund von Zeitmangel, Studienabbruch und anderen Gründen kann es jederzeit passieren, dass Gruppenmitglieder aus der Gruppe austreten. Dadurch müssen die anderen Personen dann entsprechend mehr arbeiten, was zu Problemen führen kann.\\
\textbf{Maßnahmen:}\\
1. Bei einem Austritt aus der Gruppe gibt das Mitglied den anderen sofort Bescheid, damit diese sich rechtzeitig auf die Mehrarbeit einstellen können.\\
2. Sollten mehrere Mitglieder austreten, Gespräch mit dem Tutor suchen um gegebenfalls die Anforderungen zu senken.\\
3. Im Zeitplan vor den Deadlines immer ein wenig Zeit überlassen, um durch einen plötzlichen Austritt die Abgabe noch rechtzeitig zu schaffen.\\
\bigskip \\
\textbf{Inkompetenz eines Gruppenmitglieds:}\\
Es kann passieren das ein Gruppenmitglied Inkompetent ist und somit nicht in der Lage die ihm zugeteilten Aufgaben zu lösen. Dabei kann es passieren, dass 

\subsection{Projektüberwachung}\label{3.4-controlling}

Um das Projekt zu überwachen wird mindestens einmal die Woche ein Treffen stattfinden, wo überprüft wird, wer wie weit ist. Außerdem wird es einen permanenten Austausch über Skype geben, wo nachgefragt wird, ob es noch Probleme gibt. \\
Es wird auch für jede Phase einen Phasenleiter geben, der dafür zuständig ist, den Zeitplan im Auge zu behalten. Diesem Phasenleiter muss dann jedes Gruppenmitglied regelmäßig Bescheid geben, wie weit die Teilaufgabe bereits bearbeitet ist.\\
Sollte es Probleme bzw. Verzögerungen geben, werden diese im wöchentlichen treffen angesprochen und gemeinsam gelöst, indem z.B die in Verzögerung geratene Aufgabe unter den Mitgliedern aufgeteilt wird.\\
Bei größeren Probleme kann ein Treffen spontan einberufen werden oder es wird direkt bei Skype angesprochen und da versucht zu lösen. Dabei ist wichtig, dass die Probleme bzw. Verzögerungen immer von der Gruppe erledigt werden, damit es nicht zur Verärgerung kommt, wenn der Phasenleiter dies alleine entscheidet.

\subsection{Mitarbeiter}
{\em Kompetenzen der und Anforderungen an die Mitarbeiter.}

%%%%%%%%%%%%%%%%%%%%%%%%%%%%%%%%%%%%%%%%%%%%%%%%%%%%%%%%%%%%%%%%%%%%%%%%

\section{Technische Prozesse}
\subsection{Methoden, Werkzeuge und Techniken}
\subsubsection{Entwicklungsplattform}
Folgende Werkzeuge werden im Entwicklungsprozess von uns benutzt:
\begin{itemize}

\item{Eclipse\footnote{\url{http://www.eclipse.org/}} ist unsere Entwicklungsumgebung (beinhaltet AndroidSDK\footnote{\url{https://developer.android.com/sdk/index.html}} für die Androidentwicklung}
\item{Maven\footnote{\url{http://maven.apache.org/}} ist unser Build-Management Tool}
\item{GlassFish 3.1\footnote{\url{http://glassfish.java.net/}} ist unser Application-Server}
\item{jUnit\footnote{\url{http://junit.org/}} ist unser Framework zum testen}
\item{GitHub\footnote{\url{http://github.com/}} zur Versionsverwaltung}
\item{GanttProject\footnote{\url{http://www.ganttproject.biz/}}} für Gantt-Diagramme
\item{\LaTeX{}\footnote{\url{http://www.latex-project.org/}}} zur Dokumentenerstellung

\end{itemize}
\todo{Sind das alle?}

\subsubsection{Entwicklungsmethode}

\todo{Abhängigkeit von Prozessmodell, Mittwoch abstimmen}

\subsubsection{Programmiersprache und Bibliotheken}
Die Programmiersprache wird Java  (mindestens Version 5) sein. Außerdem wird in geringen Umfang HTML und XML benutzt.\\
Ob und welche Bibliotheken genutzt werden, kann zu diesem Zeitpunkt(\emph{Abgabe: 20.10.2013}) nicht gesagt werden.\\
\todo{Android SDK}
\\
Sobald wir Bibliotheken nutzen, wird dieser Punkt aktualisiert.
\subsection{Dokumentationsplan}
Wir werden als Ergebnis verschiedene Dokumentationen vorweisen können. Diese sind:

\begin{itemize}
\item{Nutzerhandbuch}
\item{Installationsanleitung}
\item{Dokumentation des Quellcodes}
\end{itemize}

\subsubsection{Codingstyle}
Unsere Implementierungen werden sich an die \emph{Code Conventions for the Java Programming Language}\footnote{\url{http://www.oracle.com/technetwork/java/codeconv-138413.html}} halten.\\
\todo{*.tex Dateien}

\subsubsection{Kommentarsprache}
Die Sprache in der unsere Kommentare verfasst sind, wird Deutsch sein. Dies verhindert mögliche Missverständnisse innerhalb unserer Gruppe.

\subsubsection{JavaDoc}
Wir benutzen JavaDoc zur Dokumentation unseres Quellcodes. Dieses lässt eine einfache Erstellung von HTML-Dokumentationsdateien zu.
Zur zukünftigen Wartung wird bis auf triviale Codezeilen, der komplette Code in JavaDoc dokumentiert.
\subsubsection{Begleitende Dokumentation}
\todo{Weitere Erläuterung zur Doku?}
\subsection{Unterstützende Projektfunktionen}
In Abschnitt \todo{Abschnitt nennen} werden unsere Phasenleiter benannt, die für die jeweiligen Phasen verantwortlich sind und als Ansprechpartner und Leiter dienen.\\
Unsere Projektdateien stehen jederzeit auf \emph{GitHub}\footnote{\url{asd}} zur Verfügung. Außerdem werden regelmäßig von \emph{jedem} Gruppenmmitglied Datensicherungen, in Form von Updates und Backups des Repositorys auf dem eigenen Rechner stattfinden.\\

%{\em Gibt es Maßnahmen zur Qualitätssicherung? Wer ist zuständig?
%  Wieviel Zeit ist dafür vorgesehen?}

\todo{Qualitätssicherung}

%%%%%%%%%%%%%%%%%%%%%%%%%%%%%%%%%%%%%%%%%%%%%%%%%%%%%%%%%%%%%%%%%%%%%%%%

\section{Arbeitspakete, Zeitplan und Budget}

{\em Dieser Teil ist ein zweiter Schwerpunkt des Projektplans. Hier sollt Ihr die nächste Phase detailliert planen (siehe Arbeitspakete). Die weiteren Phasen sollen ebenfalls wenigstens grob geplant werden. Ein Gantt-Diagramm ist zwingend! 

Ihr sollt den Plan in der kommenden Phase auch tatsächlich benutzen -- und so
  Erfahrungen sammeln, was evtl. bei der Planung unberücksichtigt
  blieb. Bei der nächsten Zeitplanung (für die nächste Phase) bekommt
  Ihr dann evtl.\ eine noch bessere Planung hin.}

\subsection*{5.0.1 Anmerkungen und Annahmen}\label{aps}

Wir haben bisher nur die Phasen Projektplan und Anforderungsspezifikation vollständig in den Arbeitspaketen und Zuteilungen behandelt. Die übrigen Phasen des Projekts - Architekturbeschreibung, Implementierung und Test - können zu diesem Zeitpunkt noch nicht detailliert beschrieben werden, sondern erfolgen stattdessen in grobem Format. \\

Arbeitspunkte zusammen

\subsection{Arbeitspakete}\label{aps}

Im Folgenden sind die einzelnen Arbeitspakete des Abschnitts Projektplan aufgeführt.

\textbf{1.Arbeitspakete für Projektplan} \\
\textit{\textbf{Arbeitspaket 1.1}} \\
\begin{tabular}{|p{7.5cm}|p{7.5cm}|}
\hline
\textbf{Bezeichnung} & Vorbereitung für Projektplanerstellung\\\hline
\multicolumn{2}{|p{15cm}|}{\textbf{Beschreibung: }}  \\\hline
\textbf{Hauptverantwortlicher} & Prüfsumme\\\hline
\textbf{Abhängigkeit} & ....\\\hline
\textbf{Ressourcen} & ....\\\hline
\textbf{Aufwand und Gesamtdauer} & 10h, 120h\\\hline
\textbf{Beginn} &.... \\\hline
\textbf{Ende} & ....\\\hline
\multicolumn{2}{|p{15cm}|}{\textbf{Mindestanforderungen: }}  \\\hline
\end{tabular} \\\\

\textit{\textbf{Arbeitspaket 1.2}} \\
\begin{tabular}{|p{7.5cm}|p{7.5cm}|}
\hline
\textbf{Bezeichnung} & Vorbereitung für Projektplanerstellung\\\hline
\multicolumn{2}{|p{15cm}|}{\textbf{Beschreibung: }}  \\\hline
\textbf{Hauptverantwortlicher} & Prüfsumme\\\hline
\textbf{Abhängigkeit} & ...\\\hline
\textbf{Ressourcen} & GitHub,...\\\hline
\textbf{Aufwand und Gesamtdauer} & 10h, 120h\\\hline
\textbf{Beginn} & 13.10.2013\\\hline
\textbf{Ende} & 20.10.2013\\\hline
\multicolumn{2}{|p{15cm}|}{\textbf{Mindestanforderungen: }}  \\\hline
\end{tabular} \\\\


{\em Besonderen Wert legen wir auf die Granularität der APs. Diese
  sollten von 1-2 Personen in max. einer Woche Zeitdauer (kalendarisch, nicht
  Aufwand) bearbeitbar sein. Die Beschreibungen sollten so genau sein,
  dass der Bearbeiter damit genau weiß, was zu tun ist.}

\subsection{Zeitplan und Abhängigkeiten}

{\em Die Abhängigkeiten zwischen Arbeitspaketen oder Meilensteinen müssen genannt werden, sowie im
  Gantt-Diagramm eingezeichnet werden. Der kritische Pfad soll
  angegeben und/oder eingezeichnet werden!}

\subsection{Ressourcenanforderung}

{\em Jedem Arbeitspaket muss mind.\ ein Bearbeiter zugeordnet
  werden. Die Zuordnung der ganzen Gruppe sollte nur in Ausnahmefällen
  erfolgen -- und dann vermutlich begründet werden!}


%%%%%%%%%%%%%%%%%%%%%%%%%%%%%%%%%%%%%%%%%%%%%%%%%%%%%%%%%%%%%%%%%%%%%%%%
\section{Sonstige Elemente}
ENTFÄLLT
\subsection{Pläne für die Konvertierung von Daten}
ENTFÄLLT

\subsection{Managementpläne für Unterauftragsnehmer}
ENTFÄLLT
{\em Wenn Fremdbibliotheken benutzt werden\dots}

\subsection{Ausbildungspläne}
ENTFÄLLT
{\em Hierunter fallen z.B. auch interne Schulungen, die Ihr
  durchführen wollt.}

\subsection{Raumpläne}
ENTFÄLLT
\dots

\subsection{Installationspläne}
ENTFÄLLT
\dots

\subsection{Pläne für die Übergabe des Systems}
ENTFÄLLT
\dots


\end{document}
