\documentclass[fontsize=12pt,paper=a4,twoside]{scrartcl}

\newcommand{\grad}{\ensuremath{^{\circ}} }
\renewcommand{\strut}{\vrule width 0pt height5mm depth2mm}

\usepackage[utf8]{inputenc}
\usepackage[final]{pdfpages}
% obere Seitenränder gestalten können
\usepackage{fancyhdr}
\usepackage{moreverb}
% Graphiken als jpg, png etc. einbinden können
\usepackage{graphicx}
\usepackage{stmaryrd}
% Floats Objekte mit [H] festsetzen
\usepackage{float}
% setzt URL's schön mit \url{http://bla.laber.com/~mypage}
\usepackage{url}
% Externe PDF's einbinden können
\usepackage{pdflscape}
% Verweise innerhalb des Dokuments schick mit " ... auf Seite ... "
% automatisch versehen. Dazu \vref{labelname} benutzen
\usepackage[ngerman]{varioref}
\usepackage[ngerman]{babel}
\usepackage{ngerman}
% Bibliographie
\usepackage{bibgerm}
% Tabellen
\usepackage{tabularx}
\usepackage{supertabular}
\usepackage[colorlinks=true, pdfstartview=FitV, linkcolor=blue,
            citecolor=blue, urlcolor=blue, hyperfigures=true,
            pdftex=true]{hyperref}
\usepackage{bookmark}

% Damit Latex nicht zu lange Zeilen produziert:
\sloppy
%Uneinheitlicher unterer Seitenrand:
%\raggedbottom

% Kein Erstzeileneinzug beim Absatzanfang
% Sieht aber nur gut aus, wenn man zwischen Absätzen viel Platz einbaut
\setlength{\parindent}{0ex}

% Abstand zwischen zwei Absätzen
\setlength{\parskip}{1ex}

% Seitenränder für Korrekturen verändern
\addtolength{\evensidemargin}{-1cm}
\addtolength{\oddsidemargin}{1cm}

\bibliographystyle{gerapali}

% Lustige Header auf den Seiten
  \pagestyle{fancy}
  \setlength{\headheight}{70.55003pt}
  \fancyhead{}
  \fancyhead[LO,RE]{Software--Projekt 1\\ SoSe 2013
  \\Projektplan}
  \fancyhead[LE,RO]{Seite \thepage\\\slshape \leftmark\\\slshape \rightmark}

% neuer Befehl: \includegraphicstotab[..]{..}
% Verwendung analog wie \includegraphics
\newlength{\myx} % Variable zum Speichern der Bildbreite
\newlength{\myy} % Variable zum Speichern der Bildhöhe
\newcommand\includegraphicstotab[2][\relax]{%
% Abspeichern der Bildabmessungen
\settowidth{\myx}{\includegraphics[{#1}]{#2}}%
\settoheight{\myy}{\includegraphics[{#1}]{#2}}%
% das eigentliche Einfügen
\parbox[c][1.1\myy][c]{\myx}{%
\includegraphics[{#1}]{#2}}%
}% Ende neuer Befehl

%
% Und jetzt geht das Dokument los....
%

\begin{document}

% Lustige Header nur auf dieser Seite
  \thispagestyle{fancy}
  \fancyhead[LO,RE]{ }
  \fancyhead[LE,RO]{Universität Bremen\\FB 3 -- Informatik\\
  Prof. Dr. Rainer Koschke \\TutorIn: Dierk Lüdemann}
  \fancyfoot[C]{}

% Start Titelseite
  \vspace{3cm}

  \begin{minipage}[H]{\textwidth}
  \begin{center}
  \bf
  \Large
  Software--Projekt 1 2013\\
  \smallskip
  \small
  VAK 03-BA-901.02\\
  \vspace{3cm}
  \end{center}
  \end{minipage}
  \begin{minipage}[H]{\textwidth}
  \begin{center}
  \vspace{1cm}
  \bf
  \Large Projektplan\\
  \vfill
  \end{center}
  \end{minipage}
  \vfill
  \begin{minipage}[H]{\textwidth}
  \begin{center}
  \sf
  \begin{tabular}{lrr}
  David Brinkmann  & david.brinkmann@uni-bremen.de & 2696099\\
  Patrick Damrow & damsen@tzi.de & 2056170\\
  Habenicht, Arnaud & mephisto@uni-bremen.de & 2260395\\
  Daniel Pupat & s\_aydi4h@uni-bremen.de & 2703053 \\
  Michael Sauerweis & s\_rfkrpo@uni-bremen.de & 2699739 \\
  Leopold Siakeu & henrileopold@yahoo.fr & 2193984
  \end{tabular}
  \\ ~
  \vspace{2cm}
  \\
  \it Abgabe: 02. 05. 2013 --- Version 1.0\\ ~
  \end{center}
  \end{minipage}

% Ende Titelseite

% Start Leerseite

\newpage

  \thispagestyle{fancy}
  \fancyhead{}
  \fancyhead[LO,RE]{Software--Projekt \\  2013
  \\Projektplan}
  \fancyhead[LE,RO]{Seite \thepage\\\slshape \leftmark\\~}
  \fancyfoot{}
  \renewcommand{\headrulewidth}{0.4pt}
  \tableofcontents

\newpage

  \fancyhead[LE,RO]{Seite \thepage\\\slshape \leftmark\\\slshape \rightmark}


%%%%%%%%%%%%%%%%%%%%%%%%%%%%%%%%%%%%%%%%%%%%%%%%%%%%%%%%%%%%%%%%%%%%%%%%
\section*{Version und Änderungsgeschichte}

{\em Die aktuelle Versionsnummer des Dokumentes sollte eindeutig und gut zu
identifizieren sein, hier und optimalerweise auf dem Titelblatt.}

\begin{tabular}{ccl}
Version & Datum & Änderungen \\
\hline
0.1 & 27.04.2013 & Arbeitspakete und Abhängigkeiten\\
0.2 & 27.04.2013 & Risikomanagement\\
0.3 & 27.04.2013 & Managementprozess 2.2\\
0.4 & 27.04.2013 & Mitarbeiter hinzugefügt \\
1.0 & 27.04.2013 & Erste veröffentlichte Version. \\
1.1 & 08.05.2013 & Korrektur nach Abgabe
\end{tabular}


%%%%%%%%%%%%%%%%%%%%%%%%%%%%%%%%%%%%%%%%%%%%%%%%%%%%%%%%%%%%%%%%%%%%%%%%
\section{Einleitung}

\textit{Erstellt von: David Brinkmann und Patrick Damrow}


\subsection{Projektübersicht}

In SWP-1 ist ein Bibliothekssystem zu erstellen. Dieses System beinhaltet sowohl eine Website als auch eine Android App. Die Website richtet sich an Leser (Bibliotheksbenutzer) und Bibliothekare. Die App ist für die Leser gedacht. Bibliothekare können Bücher und Benutzer verwalten (anlegen und einsehen). Leser können Bücher einsehen und bewerten. Dieses System wird vermutlich in SWP-2 weiterentwickelt mit weiteren Anforderungen. Daher ist es ratsam, bereits jetzt an mögliche zukünftige Änderungen zu denken. (Quelle: \url{http://www.informatik.uni-bremen.de/st/Lehre/swp/mindestanforderungen.html})

\subsubsection{Hauptarbeitsaktivitäten und --produkte}
Implementierung einer Bibliotheksverwaltungssoftware in Android.


\subsubsection{Haupt--Meilensteine und grober Zeitplan}

\begin{description}
\item[M1 - 02.05.13] Projektplan abgeben \\
Jedes Mitglied muss seinen Teil fertig gestellt haben. Anschließend wurden alle Einzelteile zusammengeführt und von allen auf Korrektheit geprüft.
\item[M2 - 14.05.13] Anforderungsspezifikationen (Intern) \\
Jedes Mitglied hat seinen Teil der Anforderungsspezifikation fertiggestellt. Anschließend wurden die Teile zusammen geführt und von allen auf Korrektheit geprüft.
\item[M3 - 16.05.13] Abgabe der Anforderungsspezifikation \\
Meilenstein 2 muss bereits fertig sein. Abgabe der Anforderungsspezifikation via MEMS.
\item[M4 - 24.06.13] Architektur, Testplan, Tests (Intern)\\
Jedes Mitglied muss seine Aufgaben erfüllt haben. Teile wurden zusammengeführt und kontrolliert. Tests müssen implementiert sein.
\item[M5 - 27.06.13] Architekturbeschreibung, Testplan und Schnittstellentests fertig\\
Meilenstein 4 muss bereits erreicht worden sein. Tests wurden lauffähig implementiert. Abgabe via MEMS.
\item[M6 - 19.07.13] Vollständige Abgabe des Dokuments und Software \\
Implementierung in der Woche zum 19.07. Vollständige Abgabe der Dokumente und der lauffähigen Software.
\end{description}

\newpage
\subsubsection{Mitarbeiter}
\begin{table}[htbp]
\begin{tabular}{|c|c|c|}
\hline 
Name & Email\\ \hline
Brinkmann, David & david.brinkmann@uni-bremen.de & \includegraphicstotab[scale=0.45]{david.jpg} \\ \hline
Damrow, Patrick & damsen@tzi.de & \includegraphicstotab[scale=0.45]{patrick.jpg} \\ \hline
Habenicht, Arnaud & mephisto@informatik.uni-bremen.de & \includegraphicstotab[scale=0.45]{arnaud.jpg} \\\hline
Pupat, Daniel & s\_aydi4h@uni-bremen.de & \includegraphicstotab[scale=0.45]{daniel.jpg} \\ \hline
Sauerwein, Michael & s\_rfkrpo@uni-bremen.de & \includegraphicstotab[scale=0.45]{michael.jpg}\\ \hline
Siakeu, Leopold & henrileopold@yahoo.fr & \includegraphicstotab[scale=0.45]{leo.jpg} \\ \hline
\end{tabular}
\end{table}

\newpage
\subsection{Auszuliefernde Produkte}


\subsection{Referenzen}
% mit \nocite kann man Literatur auflisten, die im Text nicht explizit
% erwähnt ist. \nocite{*} zitiert dann das ganze .bib-File
%
% Die Bibliographie erzeugt man indem man erst
%
% pdflatex bericht.tex
% bibtex bericht
% pdflatex bericht.tex
% pdflatex bericht.tex
%
% benutzt
%\nocite{Knudsen1}
%\nocite{*}
%\bibliography{literatur}

% Das renewcommand verhindert dass für die Literatur eine section* angelegt wird.
% auftaucht
{\renewcommand\section[2]{}
\bibliography{referenzen}
}

\section{Managementprozess}
\textit{Erstellt von: Leopold Siakeu und Arnaud Habenicht}

\subsection{Managementprozess und --prioritäten}
\textit{Erstellt von: Patrick Damrow}\\
Die Prioritäten liegen in der Fertigstellung der vom Kunden gewünschten Anwendung zur
Verwaltung einer Bibliothek bzw. die Implementierung dieser Anwendung auf Android-fähigen Endbenutzergeräten. Die Zufriedenheit des Kunden steht hier im Vordergrund. Um dies zu gewährleisten, werden wir einen Qualitätsmanager bestimmen, der regelmäßig überprüfen soll, ob alle Anforderugen erfüllt sind. Darüber hinaus trifft sichdie Gruppe einmal wöchentlich Mittwochs für 2 Stunden von 12 - 14 Uhr an der Uni, um über die Qualität, den Fortschritt und auftretende Probleme zu sprechen. Zur Qualitätsgewährleistung der Dokumente treffen sich alle Mitglieder mindestens zwei Tage vor Abgabe des eweiligen Dokuments, um einen Zeitpuffer für Verbesserungen zu haben. \\

Für ein gutes Gruppenklima, welches zwingend notwendig für ein funktionierendes und produktives Team ist, planen wir ausseruniversitäre Aktivitäten. Dies soll für Vertrauen und Zusammengehörigkeit sorgen. Intern haben wir "Spielregeln" eingeführt um Unzuverlässigkeit zu unterbinden und somit das Vertrauen zusätzlich zu stärken. Die Kommunikation unserer Gruppe wird meist elektronisch über E-Mail von statten gehen. Teilweise kommunizieren wir über Telefon und/oder WhatsApp, sowie den wöchentlichen Meetings.

\subsection{Annahmen, Abhängigkeiten und Einschränkungen}
\textit{Erstellt von: Patrick Damrow}\\
Das Projekt beruht auf der Annahme, dass dem Kunden und Mitarbeitern mit dieser Software die Verwaltung der Bibliothek vereinfacht wird. Diese muss zweisprachig gestaltet sein.
Darüberhinaus nehmen wir an, das jedes Gruppenmitglied solide Java Grundkenntnisse hat, da alle Mitglieder der Gruppe mindestens Praktische Informatik 1 an der Universität Bremen bestanden haben. \\

Die meisten Gruppenmitglieder besitzen ein Smartphone oder Tablet mit einem Android Betriebssystem, was das Testen der Software erleichtert, da es direkt am Endgerät getestet werden kann. Falls diese Ausfallen sollten, benutzen wir den Emulator des AndroidSDK.
Das Problem der Abhängigkeit an einen festen Ort gebunden zu sein, lösen wir indem jedes Gruppenmitglied ein eigenes Notebook besitzt. Bei Ausfall können wir auf Rechner im Rechnerpool der Ebene 0 im MZH an der Uni Bremen zurückgreifen. \\

Als zusätliche Abhängigkeit wären noch die Gesundheit der Mitglieder, das SVN (Subversion Repository) und LaTex zu nennen. Unser Arbeitsstundenkontingent ist stark von der Gesundheit unserer Mitglieder abhängig, da ein langfristiger Ausfall viel mehr Arbeit für die anderen Mitglieder bedeuten würde. Das SVN ist wichtig, da wir sonst nicht die Arbeiten von 6 Personen an einem Dokument bewältigen könnten und LaTex wird von uns alas Textsatz-Software zur Erstelung unserer Dokumente verwendet.\\

Zeitliche Einschränkungen entsehe zusätlich dadurch, dass eineige Mitglieder Regelmäßig arbeiten gehen und andere universitäre Veranstaltungen besuchen.

Eine Einschränkung besteht in der Platzformabhängigkeit von Android, sowie der verfügbaren Zeit zur Fertigstellung dieses Projekts. Zudem ist die erfolgreiche Fertigstellung des Projekts Voraussetzung zum Bestehen des Moduls SWP-1.

\subsection{Risikomanagement}\label{riskmanagement}
 {\em "'Projekte, Prozesse und Produkte (IT-Infrastrukturen, Software) im IT-Bereich unterliegen –
das steht außer Zweifel – immer einem gewissen Risiko. Eine Vielzahl von Einflussfaktoren
kann dazu führen, dass das Erreichen der angestrebten Ziele gefährdet oder gar erhebliche
negative Folgewirkungen denkbar sind (etwa aus Fehlern oder Verzögerungen in der Ausführung
der IT-Prozesse oder der IT-Projekte)."' }   (Quelle: Tiemeyer (Hrsg.), 2011, Handbuch IT-Management, 4. Auflage)\\

Es wird versucht, mögliche Risiken im Voraus zu erkennen und wenn möglich, Maßnahmen zu ergreifen, welche entweder die Eintrittswahrscheinlichkeit und/oder die Schadenhöhe eindämmen. Des Weiteren wird während des Projekts stets auf neu aufkommende Risiken geachtet und auf dadurch entstehende Probleme frühzeitig reagiert. Die Kontrolle über Risiken währende des Projekts liegt in der Obhut des Phasenleiters, welcher gegebenenfalls Gegenmaßnahmen einleitet, Gruppentreffen organisiert oder in Arbeitsschritte interveniert.\\
Absehbare Risiken sind in der nachfolgenden Tabelle aufgeführt. Gegenmaßnahmen werden im Anschluss an diese erläutert.

\begin{center}
\begin{tabular}{|c|c|c|c|} \hline
Risiko & Eintrittswahrsch. (1-3) & Schadenshöhe (1-6) & Risikohöhe\\ \hline
Unzuverlässigkeit & 2  & 3 & 6\\ \hline
Personalausfall & 1 & 2 & 2\\ \hline
Inkompetenz & 1 & 4 & 4\\ \hline
Technik & 1 & 5 & 5\\ \hline
Kommunikation & 2 & 3 & 6\\ \hline
Zeitmangel & 3 & 4 & 12\\ \hline
Unstimmigkeiten & 2 & 4 & 8\\ \hline
Mangelnde Motivation & 3 & 3 & 9\\ \hline
\end{tabular}
\end{center}

{\em(1 niedrigste, 3/6 höchste)}
\bigskip \\
Maßnahmen gegen:
\begin{description} 
\item[Unzuverlässigkeit:] Regeln aufstellen, Konsequenzen (Abmahnungen), Ausschluss aus Gruppe \\ EW: 1 SH: 3
\begin{itemize}
\item \textit{\underline{Präventive Maßnahmen}}: Es wird von Anfang über Regeln besprochen und Regeln festgelegt. Alle müssen von getroffenen Regeln Bescheid wissen, denn deren Missbrauch Konsequenzen hat, wie zum Beispiel: Abmahnungen.
\item \textit{\underline{Reaktive Maßnahmen}}: Sollte ein oder mehrerer Mitglieder der Gruppe sich unvernünftig benehmen (Treffen ignorieren, Anforderungen nicht anhalten, u.s.w.) können die anderen Mitglieder ihn verwarnen. Falls sein Verhalten sich nicht verbessert, scheidet das Mitglied (nach Absprache der Gruppe, mit einem Tutor bzw. mit dem Professor) aus der Gruppe aus.\\
\end{itemize}
\item[Personalausfall:] Aufgaben neu verteilen, bei Ausschluss eines Mitgliedes Projektanforderungen verringern (Absprache mit Tutor) \\ EW: 1 SH: 1
\begin{itemize}
\item \textit{\underline{Präventive Maßnahmen}}: (kann nicht vorgebeugt werden).
\item \textit{\underline{Reaktive Maßnahmen}}: Sollte ein oder mehrerer Mitglieder an einer Krankheit bzw. unter Unfallfolgen leiden (Erkältung, Knochenbruch, ...), werden weitere Mitglied seine Aufgabe(n) vorläufig übernehmen. Eine Verteilung wird in der Gruppe besprochen.\\
\end{itemize}
\item[Inkompetenz:] Gegenseitige Hilfe, Tauschen der Arbeiten, Wissen aneignen \\ EW: 1 SH: 4
\begin{itemize}
\item \textit{\underline{Präventive Maßnahmen}}: Eine Umfrage wird durchgeführt um herauszufinden, wer was am besten kann. Dadurch kann man eine optimale Aufgabeverteilung planen.
\item \textit{\underline{Reaktive Maßnahmen}}: Sollte ein oder mehrerer Mitglieder die erwarteten Grundkenntnisse nicht vorweisen, müssen sie frühzeitig Bescheid geben, so dass sie von den anderen Team-Mitglieder unterstützt werden können, je nachdem wie groß die Notwendigkeit dazu ist, kann eine neue Verteilung geplant werden.\\
\end{itemize}

\item[Technik:] Ständige Backups, Ausweichen auf Uni-Rechner, Ausweichen auf alternative Software \\ EW: 1 SH: 3
\begin{itemize}
\item \textit{\underline{Präventive Maßnahmen}}: Alle Mitglieder müssen dafür sorgen, dass regelmäßige Backups durchgeführt werden. Sie müssen ebenfalls zur Sicherung der Arbeit eine Kopie auf einer beliebigen Datenträger speichern.
\item \textit{\underline{Reaktive Maßnahmen}}: Sollte ein Mitglied (von Zuhause aus, auf seinem Notebook) keinen Zugang zum Server haben, stehen ihm noch die Uni-Rechner zu Verfügung.\\
\end{itemize}
\item[Kommunikation:] Regelmäßige Treffen, Zusätzlicher Austausch über Mails/Telefon EW:1 SH: 3
\begin{itemize}
\item \textit{\underline{Präventive Maßnahmen}}: Regelmäßige Treffen müssen geplant werden, nicht nur im Rahmen der Arbeitsbesprechung, sondern auch um die gesellschaftliche bzw. die soziologische Ebene unter Mitglieder zu pflegen.
\item \textit{\underline{Reaktive Maßnahmen}}: Sollte es doch vorkommen, dass ein oder mehrerer Mitglieder aus irgendeinem Grund nicht zu Treffen erscheinen können, bekommen die Abwesenden per Mail/Telefon ein Feedback.\\
\end{itemize}
\item[Zeitmangel:] Vorhandene Zeit organisieren, Ausweichen auf Wochenenden, Kleine Zeitpuffer einplanen \\ EW: 2 SH: 4
\begin{itemize}
\item \textit{\underline{Präventive Maßnahmen}}: Um zu vermeiden, dass es der Gruppe an Zeit fehlt, wird mehr Zeit als nötig für eine Abgabe geplant, so dass bei Umständen noch mehr Zeit in Anspruch genommen wird, ohne dass es die Arbeit des Teams direkt beeinflusst bzw. sofort hindern.
\item \textit{\underline{Reaktive Maßnahmen}}: Beim Zeitmangel, werden zusätzliche Termine geplant (Überstunden am Wochenende sind nicht ausgeschlossen). Im schlimmsten Fall kann sich die für eine Arbeitspaket geplante Anzahl von Mitgliedern verdoppelt, um an Zeit wahrscheinlich aufzuholen.\\
\end{itemize}
\item[Unstimmigkeiten:] Diskussion in der Gruppe, neutrale Person entscheiden lassen\\ EW: 1 SH: 3
\begin{itemize}
\item \textit{\underline{Präventive Maßnahmen}}: Jede Mitglieder sollten auf sachlicher Ebene handeln. Das heißt nicht andere Ideen aus Gefühl (wie Stolz, Schuldgefühl, Freundschaft) ablehnen bzw. annehmen, sondern aus praktischer Sicht betrachten.
\item \textit{\underline{Reaktive Maßnahmen}}: Sollten sich Gruppenmitglieder über eine Idee nicht einigen können, entscheidet in einer Gruppenbesprechung die Mehrheit, welche Option die beste wäre (es wird davon ausgegangen, dass die vorgeschlagenen Ideen umsetzbar sind). Sollte in der Gruppe keine Entscheidung getroffen werden, schließt eine neutrale Person, die sich auskennt, die Debatte ab.\\
\end{itemize}
\item[Motivation:] Zeiten einplanen, sich mit der Gruppe zu treffen, ohne dabei am Projekt zu arbeiten. \\
EW: 2 SH: 3
\begin{itemize}
\item \textit{\underline{Präventive Maßnahmen}}: Eine positive Einstellung ist in der Gruppe empfohlen. Gegenseitige Rücksicht ist von Vorteil, so dass man diejenigen, den es an Kraft fehlt, ermuntern kann.
\item \textit{\underline{Reaktive Maßnahmen}}: Sollte es der Gruppe an Begeisterung und Motivation fehlen, werden Treffen ``aufs Bier'' organisiert, so dass die Mitglieder für eine Zeit lang abschalten können und dadurch sich gegenseitig wieder stärken.\\
\end{itemize}
\end{description}




\subsection{Projektüberwachung}\label{3.4-controlling}
Der Phasenleiter organisiert regelmäßige Treffen und informiert sich über den aktuellen Stand der einzelnen Mitarbeiter. Diesen Stand vergleicht er mit dem aufgestellten Zeitplan und ergreift nötigenfalls Maßnahmen zur Beschleunigung der noch anstehenden Arbeiten.
Die Kommunikation unter den Mitarbeitern stellt sicher, dass etwaige Defizite frühzeitig erkannt werden und gegenseitige Unterstützung gewährleistet wird, zur Einhaltung der vorgegebenen Zeiten.

\subsection{Mitarbeiter}
Vorausgesetzt werden solide Grundkenntnisse in Java, soweit diese im Modul PI-1 vermittelt wurden. Des Weiteren wird vorausgesetzt, dass sich jeder Mitarbeiter eigenständig in neue Themenbereiche einarbeiten kann. Außerdem sollte jeder Mitarbeiter in der Lage sein eigene Defizite zu erkennen und diese rechtzeitig der Gruppe mitteilen können.\\
Für die Teilbereiche (wie. z.B. Layout) sollte der Mitarbeiter über die entsprechenden Kenntnisse verfügen.
 
\section{Arbeitspakete, Zeitplan und Budget}
\textit{Erstellt von: Michael Sauerwein und Daniel Pupat}
\subsection{Arbeitspakete}\label{aps}

\begin{description}
\item[1 Initialer Projektplan] In dem Arbeitspaket wird der Projektplan erstellt, mit einer Einleitung, dem Managmentprozess und den Arbeitspaketen und einen Zeitplan. Phasenleiter dieses Projektes ist David Brinkmann, wobei alle zusammen daran arbeiten, da nur eine Aufgabe ist, in diesem Zeitraum zu erledigen ist. Dieses Projekt soll im Zeitraum vom 24.04 bis zum 2.05 andauern.
\item[2 Anwendungsspezifikation] Im Arbeitspaket Anwendugsspezifikation wird eine detaillierte Darstellung des Datenmodells, der Anwendungsfälle und der Aktionen geben. Phasenleiter dieses Pakets ist Patrick Damrow, wobei wieder alle Gruppenmitglieder an dem Projekt arbeiten. Dieses Arbeitspaket soll im Zeitraum vom 2.05 bis zum 16.05 fertiggestellt werden
\item[3 Architekturbeschreibung, Testplan und Schnelltests] In dem Arbeitspaket wird die Architektur der Implementierungen erstellt, sowie schon vorläufige Testfälle. Phasenleiter dieses Projektes ist Michael Sauerwein und es werden wieder alle Mitglieder an diesem Projekt arbeiten. Dieses Arbeitspaket wird vom 16.05 bis zum 27.6 fertiggestellt werden
\item[4.0] Arbeitspakete 4.1-4.4 werden alle seperat in der Implementierungsphase vom 15.07 bis zum 19.07 ablaufen. Der Phasenleiter der Implementierungsphase ist Daniel Pupat, der einen Überblick über die einzelnen Pakete haben soll. 
\item[4.1 Server:]Der Server soll zur Kommunikation und Datenaustausch mit dem Client dienen, sowie einen Zugriff auf die Datenbank haben. Der Server wird von David Brinkmann bearbeitet. Dieser Vorgang soll vom 15.07 bis zum 18.07 andauern.
\item[4.2 Client:]Der Client soll mit dem Server kommunizieren, in dem er Daten vom Server abgerufen bzw. mit dem Server austauscht. Der Client wird von Brice Arnaud Habenicht bearbeitet. Dieser Vorgang soll vom 15.07 bis zum 18.07 andauern.
\item[4.3 GUI:]Die GUI soll übersichtlich und Nutzerfreundlich sein.Die GUI wird von Patrick Damrow(50 \%) und Michael Sauerwein(50 \%) bearbeitet. Dieser Vorgang soll vom 15.07 bis zum 18.07 andauern. 
\item[4.4 Datenbank:] \hfill
\begin{itemize}
\item Es soll eine Datenbank von Büchern erstellt werden, die in der Bibliothek vorhanden bzw. ausgeliehen werden können.\item Es soll eine Datenbank von Nutzern erstellt werden, in welcher Kunden der Bibliothek enthalten sind. Die Datenbanken werden von Daniel Pupat bearbeitet. Dieser Vorgang soll vom 15.07 bis zum 18.07 andauern.
\end{itemize} 
\item[5 Tests:]In diesem Arbeitspaket sollen Tests geschrieben werden, um den Code auf Fehler zu prüfen und so aufkommende Probleme vorzubeugen und überprüfen, ob das Programm überhaupt funktioniert. Die Tests werden von Daniel Pupat, Michael Sauerwein und Leopold Siakeu bearbeitet, wobei Leopold der Phasenleiter ist. Die Tests dauern vom 18.07 bis zu 19.07. Die Software Implementierung muss vorher abgeschlossen sein.
\item[6.1 Dokumentation:]Bei der Dokumentation soll ein Handbuch erstellt werden, welches zur Beschreibung und Benutzung für Mitarbeiter und Nutzer der Bibliothek dienen soll. Die Dokumentation wird von David Brinkmann bearbeitet. Die Dokumentation soll im Zeitraum vom 18.07 bis zum 19.07 erstellt werden. Die Software Implementierung muss vorher abgeschlossen sein.
\item[6.2 Skript:]Bei dem Arbeitspaket soll ein Installationsskript, zur Installation des Programms auf einem Computer oder Android fähigen Geräten, ertellt werden. Das Skript wird von Brice Arnaud Habenicht angefertigt. Das Skript soll im Zeitraum vom 18.07 bis zum 19.07 angefertigt werden. Die Software Implementierung muss vorher abgeschlossen sein.
\end{description}


\subsection{Zeitplan und Abhängigkeiten}
Als jeweilige Meilensteine wählen wir die auf der Website angegebenen Abgabetermine.
Bevor die Architekturbeschreibung, Schnittstellentests und der Testplans erstellt werden können, muss die Anwendungsspezifikation ausgearbeitet sein. Die gesamte Planungsphase muss spätestens am 06.07.2013 abgeschlossen sein. In der Woche des Blockkurses müssen zuerst die Implementierungen abgeschlossen sein, bevor die Tests und das Installationsskript bearbeitet werden können.

\subsection{Ressourcenanforderung}
Bei den Planungsarbeiten sind alle Mitglieder der Gruppe beteiligt. Während des Blockkurses werden die Ressourcen gemäß des Diagrammes aufgeteilt. In der letzten, kritischen Phase arbeiten zwei Gruppen zu je drei Leuten. 

\begin{figure}[htbp]
\caption{Gantt1}
\label{gantt1}
\includegraphics[height=\textheight]{gantt1.png}
\end{figure}

\begin{figure}[htbp]
\caption{Gantt2}
\label{gantt2}
\includegraphics[height=\textheight]{gantt2.png}
\end{figure}


\end{document}
