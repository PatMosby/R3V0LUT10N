\documentclass[fontsize=12pt,paper=a4,twoside]{scrartcl}

\newcommand{\grad}{\ensuremath{^{\circ}} }
\renewcommand{\strut}{\vrule width 0pt height5mm depth2mm}

\usepackage[utf8]{inputenc}
\usepackage[final]{pdfpages}
% obere Seitenränder gestalten können
\usepackage{fancyhdr}
\usepackage{moreverb}
% Graphiken als jpg, png etc. einbinden können
\usepackage{graphicx}
\usepackage{stmaryrd}
% Floats Objekte mit [H] festsetzen
\usepackage{float}
% setzt URLs schön mit \url{http://bla.laber.com/~mypage}
\usepackage{url}
% Externe PDF's einbinden können
\usepackage{pdflscape}
% Verweise innerhalb des Dokuments schick mit " ... auf Seite ... "
% automatisch versehen. Dazu \vref{labelname} benutzen
\usepackage[ngerman]{varioref}
\usepackage[ngerman]{babel}
\usepackage{ngerman}
% Bibliographie
\usepackage{bibgerm}
% Tabellen
\usepackage{tabularx}
\usepackage{supertabular}
\usepackage[colorlinks=true, pdfstartview=FitV, linkcolor=blue,
            citecolor=blue, urlcolor=blue, hyperfigures=true,
            pdftex=true]{hyperref}
\usepackage{bookmark}

\newboolean{langversion} %Deklaration
\setboolean{langversion}{true} %Zuweisung ist 'false' für Blockkurs
\newcommand{\highlight}[1]{\textcolor{blue}{\textbf{#1}}}
\newcommand{\nurlangversion}[0]{%
\ifthenelse{\boolean{langversion}}{\highlight{Muss in SWP-2 ausgefüllt werden}}
{\highlight{Entfällt in SWP-1}}}

\newcommand{\swp}[0]{\ifthenelse{\boolean{langversion}}%
{Software-Projekt 2}{Software-Projekt 1}}
\newcommand{\jahr}[0]{2013}
\newcommand{\semester}[0]{\ifthenelse{\boolean{langversion}}{WiSe}{SoSe} \jahr}

% Damit Latex nicht zu lange Zeilen produziert:
\sloppy
%Uneinheitlicher unterer Seitenrand:
%\raggedbottom

% Kein Erstzeileneinzug beim Absatzanfang
% Sieht aber nur gut aus, wenn man zwischen Absätzen viel Platz einbaut
\setlength{\parindent}{0ex}

% Abstand zwischen zwei Absätzen
\setlength{\parskip}{1ex}

% Seitenränder für Korrekturen verändern
\addtolength{\evensidemargin}{-1cm}
\addtolength{\oddsidemargin}{1cm}

\bibliographystyle{gerapali}

% Lustige Header auf den Seiten
  \pagestyle{fancy}
  \setlength{\headheight}{70.55003pt}
  \fancyhead{}
  \fancyhead[LO,RE]{\swp\\ \semester{}
  \\Anforderungsspezifikation}
  \fancyhead[LE,RO]{Seite \thepage\\\slshape \leftmark\\\slshape \rightmark}

%
% Und jetzt geht das Dokument los....
%

\begin{document}

% Lustige Header nur auf dieser Seite
  \thispagestyle{fancy}
  \fancyhead[LO,RE]{ }
  \fancyhead[LE,RO]{Universität Bremen\\FB 3 -- Informatik\\
  Prof. Dr. Rainer Koschke \\TutorIn: Sabrina Wilske}
  \fancyfoot[C]{}

% Start Titelseite
  \vspace{3cm}

  \begin{minipage}[H]{\textwidth}
  \begin{center}
  \bf
  \Large
  \swp{} \jahr\\
  \smallskip
  \small
  VAK 03-BA-901.02\\
  \vspace{3cm}
  \end{center}
  \end{minipage}
  \begin{minipage}[H]{\textwidth}
  \begin{center}
  \vspace{1cm}
  \bf
  {\Large Anforderungsspezifikation}\\
  \vspace{3ex}
  \small IT\_R3V0LUT10N\\
  \vfill
  \end{center}
  \end{minipage}
  \vfill
  \begin{minipage}[H]{\textwidth}
  \begin{center}
  \sf
  \begin{tabular}{lrr}
  Sebastian Bredehöft & sbrede@tzi.de & 2751589\\
  Patrick Damrow & damsen@tzi.de & 2056170\\
  Tobias Dellert & tode@tzi.de & 2936941\\
  Tim Ellhoff & tellhoff@tzi.de & 2520913\\
  Daniel Pupat & dpupat@tzi.de & 2703053\\
  \end{tabular}
  \\ ~
  \vspace{2cm}
  \\
  \it Abgabe: 17. November 2013 --- Version 1.1\\ ~
  \end{center}
  \end{minipage}

% Ende Titelseite

% Start Leerseite

\newpage

  \thispagestyle{fancy}
  \fancyhead{}
  \fancyhead[LO,RE]{\swp{}\\ \semester{} \jahr{}
  \\Anforderungsspezifikation}
  \fancyhead[LE,RO]{Seite \thepage\\\slshape \leftmark\\~}
  \fancyfoot{}
  \renewcommand{\headrulewidth}{0.4pt}
  \tableofcontents

\newpage

  \fancyhead[LE,RO]{Seite \thepage\\\slshape \leftmark\\\slshape \rightmark}


%%%%%%%%%%%%%%%%%%%%%%%%%%%%%%%%%%%%%%%%%%%%%%%%%%%%%%%%%%%%%%%%%%%%%%%%
\section*{Version und Änderungsgeschichte}

{\em Die aktuelle Versionsnummer des Dokumentes sollte eindeutig und gut zu
identifizieren sein, hier und optimalerweise auf dem Titelblatt.}

\begin{tabular}{ccl}
Version & Datum & Änderungen \\
\hline
1.0 & TT.MM.JJJJ & Projektplan als \LaTeX Vorlage kopiert.\\
1.1 & 31.10.2013 & Charakteristika der Benutzer\\
1.2 & 01.11.2013 & System- und Hardwareschnittstellen \\
\end{tabular}


%%%%%%%%%%%%%%%%%%%%%%%%%%%%%%%%%%%%%%%%%%%%%%%%%%%%%%%%%%%%%%%%%%%%%%%%
\section{Einleitung}

Dieses Dokument spezifiziert die Anforderungen des auszuliefernden 
Produkts, welche in Zusammenarbeit mit dem Kunden der Oberschule 
Rockwinkel und den Verantwortlichen der Veranstaltung Software Projekt 2 
der Universität Bremen im Wintersemester 2013/14 erarbeitet wurden.

{\em Dieses Dokument dient als Vorlage für Eure
  Anforderungsspezifikation. Die Gliederung dieses
  Dokuments ist an die Struktur des IEEE-Standards 830.1998 angelehnt,
  weicht jedoch an einigen Stellen davon ab. Die Abweichungen sind
  im weiteren Verlauf dieses Dokuments dokumentiert. Weitere detaillierte
  Hinweise finden sich im IEEE-Standard 830.1998, der in Stud.IP      
  beziehungsweise über die Uni-Bibliothek in digitaler Form verfügbar ist
  \footnote{Bei \url{http://ieeexplore.ieee.org} im Suchfeld 'IEEE std    
   830-1998' eingeben. Funktioniert nur innerhalb des Uni-Netzes.}.}

\subsection{Zweck}
\nurlangversion

  {\em Was ist der Zweck dieser Anforderungsspezifikation? Wer sind
  die LeserInnen?}

\subsection{Rahmen}
\nurlangversion

  {\em Dieser Abschnitt soll einen groben Überblick über die zu
  erstellende Software geben: Welche Produkte sind zu erstellen (mit
  Namen)? Was tut die Software? Auch: Was tut sie nicht? Wozu soll die
  Software verwendet werden?  (Ziele etc.)}

\subsection{Definitionen, Akronyme und Abkürzungen}
\nurlangversion

  {\em Hier geht es vor allem um Begriffe aus der Anwendungsdomäne,
  d.h.\ aus der Welt des Kunden. Aber auch Begriffe, die dem Kunden
  evtl.\ fremd oder unklar sind, sollten erläutert werden.}


\subsection{Referenzen}
  {\em Neben sonstigen Quellen, die Ihr verwendet habt, können dies
  z.B.\ das Skript, dieses Beispieldokument, der zugrunde
  liegende IEEE-Standard und anderes sein}
  

\subsection{Übersicht über das Dokument}
\nurlangversion

  {\em Was enthält die Anforderungsspezifikation? Wie ist das Dokument
  organisiert?}


\section{Allgemeine Beschreibung}
\label{ch:AllgemeineBeschreibung}

\subsection{Ergebnisse der Ist-Analyse}
\nurlangversion

  {\em Hier sollten die Ergebnisse Eurer Ist-Analyse kurz
  zusammengefasst werden. Diese Beschreibung ist hilfreich, um die
  Motivation für die Anforderungen zu verstehen und um sie später
  nachzuvollziehen (z.B.\ dann wenn Anforderungen überarbeitet werden
  sollen, weil sich ihre Rahmenbedingungen geändert haben).
  
  Mögliche Inhalte: 
  \begin{itemize}
    \item Interview/Beobachtung des Kunden oder der Benutzer
    \item Analyse des bisherigen Systems und dessen Probleme 
    \item Analyse ähnlicher Systeme
    \item Auswertung der Benutzerbefragung
    \item Wie sollen die identifizierten Probleme vom neuen System adressiert werden?
  \end{itemize}
  
  N.B.: Dieser Abschnitt ist im IEEE-Standard nicht vorgesehen, aber dennoch
  sinnvoll.}

\newpage
\subsubsection{Erstes Kundengespräch vom 23.10.2013}

Am Mittwoch den 23.10.2013, um 9:00 Uhr begann unser erstes Kundengespräch.
Am Tag zuvor hat die Gruppe Ideen zu einem Fragekatalog gesammelt, der dann
am Mittwoch, kurz vor der Besprechung fertiggestellt wurde. Er beinhaltete
eine Auflistung aller Mindestanforderungen, zu denen Unklarheiten bezüglich 
des Realisierungsvorgangs notiert wurden. (Hier kommt der Ablauf vom Gespräch...)

Nach dem das Gespräch wie geplant um etwa 11:00 Uhr endete, fielen uns noch 
drei bis vier Unklarheiten auf, weswegen wir sofort im Anschluss noch einmal das Gespräch mit einer Mitarbeiterin suchten. Sie war so freundlich, um sich noch 
einmal Zeit zu nehmen und sogar noch einen Rundgang mit uns zu machen. Die noch 
offenen Fragen bezogen sich auf den Vorgang des Freischalten einer Rezension
und den Ort der Benachrichtigung, nachdem eine neue Rezension vom System vermerkt
wurde. Auch dreht sich eine Frage noch um die Unterschiede des Designs zwischen
der geplanten Android- und der BrowserApplikation. Die abschließende Frage war 
noch einmal bezüglich der Bibliotheksstruktur, was Standortbezeichnungen und
Kategorisierungen angeht. Zum Schluss hat man uns noch angeboten, bei Bedarf gerne 
noch einmal wieder zu kommen. Die uns bis dahin bewussten Verständnislücken
wurden zufriedenstellend ausgefüllt. 

\subsubsection{Interview mit einem Mitarbeiter der ...}
\nurlangversion

{\em Falls durchgeführt}

\subsection{Produktperspektive}
  
\subsubsection{Systemschnittstellen}

  {\em Schnittstellen zu anderen Systemen, z.B.\ Datenimport/-export,
  Konfigurationsdateien, anzubindende externe Dienste und deren Schnittstelle,
  Anbieten der eigenen Funktionalität als API o.ä.}
  
  Grundsätzlich wird ein bestehendes Computersystem (nebst typischen Ein- und Ausgabegeräten) 
  mit einem Betriebssystem vorausgesetzt, das mit den notwendigen Schnittstellen wie z.B. dem 
  Datenim- und -export umgehen kann.
  
  \textbf{CSV-Im-/Export:}\\
  Es gibt eine Funktion, mithilfe dieser CSV-Dateien importiert werden können. Diese kann nur vom 
  Administrator benutzt werden. Die Bücher werden anschließend in der Datenbank der Bibliothek 
  vorhanden sein. \\
  Es ist auch möglich, CSV Dateien zu exportieren, welche dann abgespeichert werden.

\subsubsection{Benutzerschnittstelle}

  {\em GUI-Design-Richtlinien und Interaktionsmechanismen (nicht
  Screenshots aller Dialoge --- die werden in Kapitel 3 gezeigt --- aber
  evtl.\ ein Screenshot, der einen groben Überblick und Eindruck des
  GUI-Designs gibt).}

Als Schnittstelle zwischen Benutzer und Softwaresystem dient eine Internetseite, dessen Oberfläche 
seiner GUI als Screenshot unten dargestellt ist. \\
Die GUI weist je nach Benutzerrechten unterschiedliche Funktionalitäten auf, da es einen Unterschied ist, 
ob sich ein Ausleiher ins System einloggt oder ein Administrator. \\
Der Benutzer des Systems kann somit über einen Webbrowser mithilfe der typischen Eingabegeräte wie 
Tastatur und Maus auf diese Funktionen zugreifen und somit mit dem System interagieren. \\
Als Ausgabegerät dient selbstredend ein handelsüblicher Monitor, der in puncto Auflösung oder Größe 
keine besonderen, sondern nur minimalen Anforderungen (typischerweise mindestens 640x480 Pixel) 
genügen muss, sowie die Möglichkeit, einen Drucker einzusetzen, um beispielsweise eine Liste 
auszudrucken. \\
Ausgabeinteraktionsmechanismen sind in erster Linie Text sowie wenige Grafiken. 

\subsubsection{Hardwareschnittstellen} \label{hardware}
Das Softwaresystem besitzt als Schnittstelle zur Hardware das Betriebssystem des Computers bzw. des 
Smartphones. \\
  Es sind keine über minimale Anforderungen in Bezug auf RAM
  \footnote{RAM = Random Access Memory = Hauptspeicher des Computers}, Festplattenspeicher, 
  Prozessoren oder sonstigen Hardwarespezifika hinaus erforderlich. Somit wird die Software auch auf 
  älteren, internetfähigen Computersystemen laufen. \\
 
\subsubsection{Softwareschnittstellen} \label{software}

Unser System wird grundsätzlich plattformunabhängig laufen. Voraussetzung ist, dass das Java Runtime 
Environment sowie das Hibernate Framework (siehe Tabelle am Ende von Punkt \ref{Tabelle}) installiert 
ist. \\

   \textbf{Computer:}\\
  Unser System soll auf einem Web-Browser laufen. Dabei sollte das System auf Windows laufen, welches 
  die verwendete Plattform des Kunden ist. Dabei ist wichtig, dass alle Betriebssysteme von Windows 
  2000 bis Windows 8 unterstützt werden, da der Kunde Windows 2000 verwendet. Ebenfalls sollte das 
  System Linux und MacOS unterstützen. \\
  
  \textbf{Smartphone:}\\
  Unser System unterstützt nur Geräte, auf denen Android läuft. Dabei muss  die Version 2.3 oder höher 
  vorliegen, da somit der größte Teil der Android Geräte verwendet werden kann.\\
Im Folgenden dient eine Tabelle der Veranschaulichung von erforderlichen Softwarekomponenten nebst 
Version.  \\ 

\label{Tabelle}
  \begin{tabular}{|l|l|l|l|}\hline
    \textbf{Name} & \textbf{Version} & \textbf{Hersteller} & \textbf{Quelle} \\\hline
    Java Runtime & 6 Update 37 & Oracle & \url{http://java.com} \\\hline
    Hibernate & 4.3.0.Beta1 Release& JBoss Community &  
   \url{http://www.hibernate.org/}\\\hline
    \ldots & & & \\\hline
  \end{tabular}

\subsubsection{Kommunikationsschnittstellen}
\nurlangversion

{\em Anforderungen an und Bandbreite von Kommunikationsnetzwerken, öffentliche
  oder auch private IP-Adressen?}

\subsubsection{Speicherbeschränkung}
Wie schon im Punkt \ref{hardware} beschrieben, gibt es keine Speicherbeschränkungen. Ein PC, auf dem, 
wie beim Kunden, Windows 2000 läuft, kann also problemlos verwendet werden. Das Softwaresystem 
beansprucht nicht viele Ressourcen in puncto RAM oder Festplattenspeicher.

\subsubsection{Operationen (Betriebsmodi)}
\nurlangversion

  {\em Welche Betriebsmodi gibt es? Warum? Welche Benutzerklasse darf
  was in welchem Betriebsmodus (Rechte)? Was ist der Zusammenhang
  zwischen Betriebsmodus und Sicherung/Wiederherstellung von Daten?}

\subsubsection{Möglichkeiten der lokalen Anpassung}
\nurlangversion

  {\em Was kann bei Auslieferung des Systems alles konfiguriert
  werden? Z.B. Pfade, Datenbankname, Server-IP usw. Hier ist nicht
  Internationalisierung gemeint!}


\subsection{Anwendungsfälle}
  {\em Auflistung und kurze Beschreibung aller relevanten
  Anwendungsfälle. Dies soll einen Überblick über alle Anwendungsfälle
  geben, die in 3.2 detailliert beschrieben werden.}
  
\begin{itemize}
  \item \textbf{1. Programm starten}\\
  Website wird aufgerufen/ App wird gestartet. 
  \item \textbf{2. Benutzer anmelden}\\
  Ein Benutzer meldet sich an.
  \item \textbf{3. Benutzer abmelden} \\
  Ein Benutzer meldet sich ab.
  \item \textbf{4. Start anzeigen}
  Startseite wird mit dem "Start-Button"         
  aufgerufen.
  \item \textbf{5 Leserprofil anzeigen}
  Profil des Lesers wird mithilfe des 
  Buttons "Profil" angezeigt.
  \item \textbf{6 Vormerkung bearbeiten}
  Eigene in der Profilsicht angezeigte     
  Vormerkungen werden bearbeitet.
  \item \textbf{7. Publikationen anzeigen}
  Liste der Publikationen wird mithilfe
  des Buttons "Publikationen" angezeigt.
  \item \textbf{8. Medium hinzufügen}
  Neues Medium kann durch klicken auf 
  "Medium hinzufügen" und 
  \item \textbf{9. Medium ändern}
  \item \textbf{10.Medium löschen}
  \item \textbf{11. CVS-Import}
  Importieren einer CVS-Datei für 
  die Publikationen.
  \item \textbf{12. CVS-Export}
  Exportieren einer CVS-Datei für die
  Publikationen.
  \item \textbf{13. Buch suchen}
  In das 'Suchen-Textfeld' wird 
  der Titel des gesuchten Buches eingegeben.
  \item \textbf{14. Einzelnes Buch anzeigen/ Detailansicht}
  Mithilfe des Klicks auf den Pfeil, wird 
  die Detailsicht aufgerufen.
  \item \textbf{15. Medium bewerten}
  Ein Buch wird in der Detailansicht 
  bewertet.
  \item \textbf{16. Medium ausleihen}
  Bibliothekar leiht Leser ein oder 
  mehrere Medien aus.
  \item \textbf{17 Mediumrückgabe}
  Bibliothekar nimmt zurückgegebene
  Medien durch einscannen a, Ausleihablauf
  wird kontrolliert und Medien und Leser 
  werden auf den neusten Stand gebracht.  
  \item \textbf{18 Medium rezensieren}
  Leser kann in der Detailansicht 
  das Buch kommentieren.
  \item \textbf{19. Medium vormerken}
  Medien können in der Publikationsübersicht    
  vorgemerkt werden.
  \item \textbf{20. Rezension freischalten}
  Rezension muss vor Veröffentlichung von einem
  Bibliothekar freigeschaltet werden.
  \item \textbf{21. Leserliste anzeigen}
  Durch klicken auf 'Leser' wird die Leser-  
  übersicht oder Liste angezeigt.
  \item \textbf{22. Leser hinzufügen}
  Nach Klicken auf 'Leser hinzufügen'
  kann ein neuer Leser eingerichtet
  werden.
  \item \textbf{23. Leser ändern}
  Leserdaten können von einem 
  Bibliothekar geändert werden.
  \item \textbf{24. Leser löschen}
  Ein Leserprofil kann von einem Bibliothekar 
  gelöscht werden.
  \item \textbf{25. CVS-Import}
  Importieren einer CVS-Datei für 
  die Leserliste
  \item \textbf{27. CVS-Export}
  Exportieren einer CVS-Datei für 
  die Leserliste.
  \item \textbf{28. Einzelnen Leser anzeigen/  
  Detailansicht}
  Durch Klicke auf den Pfeil wird die 
  Detailansicht angezeigt.
  \item \textbf{29  Leser sperren}
  Leser kann von einem Bibliothekar gesperrt
  werden.
  \item \textbf{30. Leser suchen}
  Mithilfe des 'Suchen-Textfeldes' kann
  in der Leserübersicht nach einem Leser 
  gesucht werden.
  \item \textbf{31. Administration öffnen}
  Durch Klicken auf den Button Administration 
  wird die Übersicht der Administration 
  angezeigt.
  \item \textbf{32. Bibliothekarliste anzeigen}
  Bibliothekarliste kann durch den Admin 
  aufgerufen werden.
  \item \textbf{33. Bibliothekar hinzufügen}
  Der Admin kann in dem Bereich Administration
  durch klicken des Buttons 'Bibliothekar
  hinzufügen' einen neuen Bibliothekar 
  einrichten.
  \item \textbf{34. Bibliothekar löschen}
  Der Admin kann einen Bibliothekar löschen.
  \item \textbf{35. Bibliothekar ändern}
  Profildaten eines Bibliothekars können durch 
  den Admin geändert werden.
  \item \textbf{36. Statistiken anzeigen}
  Statistiken können im Bereich der       
  Administration durch den Button
  'Statistiken anzeigen' aufgerufen werden.
  \item \textbf{37. Mahnungsliste anzeigen}
  Im Bereich der Administration kann die
  Mahnungsliste durch betätigen des 
  entsprechenden Butons angezeigt werden.
  \item \textbf{38. Mahnungsliste drucken}
  In der Anzeige der Mahnungsliste kann durch 
  den Button 'Drucken' die Mahnungsliste
  ausgedruckt werden.
  \item \textbf{39. Mahnungsdetails anzeigen}
  Durch Klicken auf den Pfeil in der 
  Übersicht der Mahnungsliste, können Details
  der Mahnungen eines bestimmten Leser 
  angezeigt werden.
  \item \textbf{40. Startseite bearbeiten}
  Die Startseite kann mit Nachrichten und
  Meldungen beschrieben werden.
  \item \textbf{41. Abgabedaten und Mahngebühren bearbeiten}
  Bibliothekare können Abgabedaten und
  Mahngebühren individuell anpassen.
 \end{itemize}


\subsection{Charakteristika der Benutzer (Daniel)}


\begin{table}[htbp]
\caption{Benutzer}
\label{benutzer}
\begin{tabular}{|p{2,5cm}||p{2,8cm}|p{2,8cm}|p{2,8cm}|p{2,8cm}|}
\hline 
\textbf{Name(fiktiv)} & Bert Bib & Arnold Admin & Silke Schüler & Bart Besucher\\ \hline
\textbf{Bild(fiktiv)} & & & & \\ \hline
\textbf{Rolle} & Bibliothekar & Administrator & Leiherin & uregistrierter Leiher \\ \hline
\textbf{Beruf} & Bibliothekar & Bibliothekar & Schülerin & Anwalt\\ \hline
\textbf{Alter} & 39 & 56 & 16 & 34\\ \hline
\textbf{Ziel} & Bibliothek verwalten & System verwalten & Bücher ausleihen & Bücher ausleihen \\ \hline
\textbf{Verwendung der Software} & Bücher und Nutzer verwalten & System und Bibliothekare verwalten 
& Bücher suchen, ausleihen etc. & keine\\ \hline
\end{tabular}
\end{table}

\textbf{Bert Bib:}\\
Bert Bib ist ein Bibliothekar in der Bibliothek und arbeitet dort. Er wohnt in Bremen und ist 39 Jahre alt. Er fährt jeden Morgen mit Auto zur Arbeit und braucht dafür 25 Minuten. Er ist Verheiratet und hat 2 Kinder, welche beide männlich sind und zur Grundschule gehen. Der Ältere geht in die 4.Klasse und der jüngere in die 1.Klasse. Er ist ein großer Fussballfan und sein Lieblingsverein ist Hannover 96, da er in Hannover geboren und aufgewachsen ist. Er arbeitet bereits seit 13 Jahren als Bibliothekar und ist seit 8 Jahren an der Schule Rockwinkel beschäftigt. Er ist mit dem momentanen System unzufrieden, da der Ausleihvorgang sehr aufwändig ist. Von der neuen Software erhofft er sich eine leichtere und schnellere Verwendung um die Bibliothek zu verwalten und Bücher auszuleihen. \\

\bigskip

\textbf{Arnold Admin:}\\
Arnold Admin ist ein Lehrer an der Schule Rockwinkel und ist als Administrator für die Software zuständig. Er hat Grundkenntnisse in Informatik und kennt sich guten mit Computern aus. Er lehrt Mathematik und Physik an der Oberschule. Er ist 56 Jahre alt und wohnt auch in Bremen. Er fährt jeden Tag mit Bus zur Schule und braucht dafür 15 Minuten. Arnold war dreimal verheiratet und ist zweimal geschieden. Er hat zwei Töchter aus erster Ehe, welche bereits Berufstätig sind. Aus der aktuellen Ehe hat er einen Sohn, welcher 12 Jahre alt ist und in die 7. Klasse geht. Er liebt Bücher über alles, weshalb er sich auch als Administrator für die Bibliothek gemeldet hat. Er ist auch dafür zuständig Bibliothekare einzustellen und zu entlassen. Die Software wird er benutzen, um neue Bibliothekare zu registrieren und zu löschen. Er muss zudem auch wöchentlich die Dateien sichern und ein Back-up machen.\\

\bigskip

\textbf{Silke Schüler:}\\
Silke Schüler ist eine Schülerin der Oberschule Rockwinkel und besucht die 11.Klasse. Sie ist eine durchschnittliche Schülerin, welche beliebt unter ihren Klassenkameraden ist. Sie hat einen Freund, welcher zur Zeit eine Ausbildung macht. Sie wohnt in Bremen bei ihren Eltern und ist 16 Jahre alt. Zur Schule fährt sie immer mit dem Fahrrad und braucht dafür 10 Minuten. Sie geht am Wochenende gerne in Diskotheken oder trifft sich mit ihren Freunden. Sie lernt am liebsten mit Fachbüchern über das jeweilige Thema und ist deshalb öfter mal in der Bibliothek anzutreffen. Sie wünscht sich schon seit längeren eine App für die Bibliothek, da sie viel Zeit mit ihren Smartphone verbringt und so schnell nach Büchern suchen kann. Da sie sehr vergesslich ist, ist für sie auch ein Vorteil, dass sie über die App schnell nachgucken kann, wann sie die Bücher abgeben muss.\\

\bigskip

\textbf{Bart Besucher:}\\
Bart Besucher ist 34 Jahre alt und arbeitet als Anwalt. Er wohnt in Delmenhorst und ist momentan noch verheiratet, lebt aber getrennt von seiner Frau. Er hat einen Sohn, welcher noch in den Kindergarten geht und 5 Jahre alt ist. Er ist früher an der Oberschule Rockwinkel zur Schule gegangen, weshalb er die Bibliothek noch regelmäßig besucht. Die Software würde für ihn eine leichtere Suche bedeuten, indem er auch schon zu Hause Bücher suchen kann, da er sehr beschäftigt ist und wenig Zeit hat.\\

\subsection{Einschränkungen}
\label{sec:Einschraenkungen}
  {\em Dinge, die die Entwurfsfreiheit einschränken, z.B.
  \begin{itemize}
   \item feste Vorgaben (z.B. Policies)
   \item gesetzliche Rahmenbedingungen
   \item Hardwarebeschränkungen
   \item festgelegte Schnittstellen zu anderen Anwendungen
   \item parallele Operationen (z.B. Multithreading)
   \item Prüfungs- und Steuerungsfunktionen
   \item Verlässlichkeitsanforderungen
   \item Kritikalität der Anwendung
   \item Sicherheit
  \end{itemize}
  }

\subsubsection{Rahmenbedingungen}
\nurlangversion

\subsubsection{Gesetzliche Rahmenbedingungen}
\nurlangversion
 
\subsubsection{Sicherheitskritische Aspekte}
\nurlangversion

\subsection{Annahmen und Abhängigkeiten}

Bis zur Auslieferung der Software wird sich der Kunde nicht ändern. Die Anforderungsspezifikation dient 
als eine Art Vertrag mit dem Kunden. Deshalb ist davon auszugehen, dass nach der Abgabe der 
Anforderungsspezifikation keine zusätzlichen Anforderungen hinzukommen. \\
Abgabetermine haben Deadlines und sind somit strikt einzuhalten.\\

Des Weiteren wird davon ausgegangen, dass sich die Nutzer der Software zwar mit dem System 
eingehend auseinandersetzen. Es wird jedoch auch für den ungeübten Nutzer leicht möglich sein, dieses 
zu verwenden. Der jeweilige Nutzer sollte zumindest schon mal mit einem Computer gearbeitet haben.\\

Zu Hardware- und Software-Abhängigkeiten geben die Punkte \ref{hardware} (Hardwareschnittstellen) 
und \ref{software} (Softwareschnittstellen) hinreichend Aufschluss.
 
\subsection{Ausblick}
\nurlangversion

  {\em Beschreibt hier knapp, welche Änderungen und Erweiterungen
  zukünftig (d.h.\ nach Auslieferung des Systems) zu erwarten sind.
  Diese Information ist wichtig für den Entwurf, um mögliche
  Änderungen frühzeitig im ersten Entwurf berücksichtigen zu können.
  Der Entwurf kann dann so gestaltet werden, dass die zukünftigen
  Anforderungen leicht realisierbar sind. Die zukünftigen
  Anforderungen sollten realistisch sein, ansonsten könnte ein unnötig
  allgemeiner und damit zu komplizierter Entwurf die Folge sein.  Auch
  dieser Abschnitt ist im IEEE-Standard nicht vorgesehen -- zumindest
  nicht explizit in Form eines eigenständigen Abschnitts. Dennoch
  handelt es sich um wertvolle Information, von der der Entwurf
  profitieren kann.}
  

\section{Detaillierte Beschreibung}
\label{ch:DetaillierteBeschreibung}
{\em Die externen Schnittstellen werden grob in Abschnitt 2
  beschrieben.  Wenn die grobe Beschreibung dort nicht genügt, kann
  sie hier detaillierter ausgeführt werden (wie vom IEEE-Standard
  vorgesehen).}

\subsection{Datenmodell}
\begin{itemize}
 \item \textbf{1. Person:}\\
  Stellt die Oberklasse aller Nutzer, Bibliothekare oder 
  des Admins dar.

\item \textbf{2. Admin:}\\
Die Klasse Admin oder Administrator erbt von Person und kann mithilfe der    Assoziationsklasse '"Bearbeiten/Löschen/Hinzufügen"' die Profildaten eines
Bibliothekars entweder bearbeiten oder eine kompletten Bibliothekar löschen 
oder  neu einrichten.
 
\item \textbf{3. Bibliothekar:}\\
Die Klasse Bibliothekar erbt von Person und kann Rezensionen freischalten und
kann Daten von sowohl Leihobjekt, der Assoziationsklasse Ausleihe, als auch 
dem Nutzer bearbeiten. Zusätzlich ist Bibliothekar an der Ausleihe beteiligt.

\item \textbf{4. Nutzer:}\\
Auch der Nutzer erbt von Person und stellt den Leser dar, der sowohl Exemplare
ausleihen und vormerken bzw. reservieren, als auch das Leihobjekt bewerten und 
rezensieren kann.

\item \textbf{5. Leihobjekt:}\\
Die Klasse Leihobjekt stellt ein beliebiges Medium dar, welches in der 
Bibliothek vorhanden ist. Gleichzeitig ist sie die Oberklasse für Buch,
CD und noch eine Ebene weiter auch für Exemplar. Das Leihobjekt wird vom Nutzer
bewertet und rezensiert und vom Bibliothekar erstellt oder entweder bearbeitet 
oder gelöscht.

\item \textbf{6. Buch:}\\
Das Buch erbt von Leihobjekt und ist gleichzeitig eine mögliche Oberklasse
für Exemplar. Zusätzlich steht es mit der Klasse Buchreihe über eine Aggregation
in Verbindung. 

\item \textbf{7. Buchreihe:}\\
Die Klasse Buchreihe steht mit der Klasse Buch über eine Aggregation in Verbindung.
Eine Buchreihe kann beliebig viele Buchobjekte besitzen.
 
\item \textbf{8. CD:}\\
CD erbt ebenfalls von Leihobjekt und ist eine mögliche Oberklasse für Exemplar.

\item \textbf{9. Exemplar:}\\
Das Exemplar ist das Objekt, welches an den Nutzer verliehen und von diesem reserviert oder vorgemerkt wird. Es hat eine individuelle ID mit der bei der Rückgabe, Medien
eindeutig dem entsprechenden Nutzer zugeordnet werden kann. Exemplar erbt entweder von
der Klasse Buch oder von der Klasse CD, niemals beide oder keinem von beiden. Ein Exemplar ist also immer entweder eine CD oder ein Buch.

\item \textbf{10. Ausleihe:}\\
Die Assoziationsklasse Ausleihe stellt den Vorgang des Ausleihens dar.
Es beschreibt die Verbindung zwischen dem Nutzer, dem Bibliothekar und dem Exemplar, indem es unter anderem die ID des verliehenen Exemplars und die zu beachtende Frist 
als Variablen bekommt. 


\item \textbf{11. Reservierung/Vormerkung:}\\
Diese Assoziationsklasse bekommt das aktuelle oder gewünschte Datum der vom Nutzer getätigten Vormerkung oder Reservierung zugeschrieben.

\item \textbf{12. Bewertung:}\\
Die Assoziationsklasse Bewertung stellt die Bewertung eines Nutzers zu einem Leihobjekt dar. Zum Festhalten der Bewertungshöhe erhält die Klasse Bewertung
eine Integer-Variable 'Bewertung'.

\item \textbf{13. Rezensieren:}\\
Assoziationsklasse. Ein Nutzer kann über ein Leihobjekt eine Rezension schreiben, die allerdings vor der Veröffentlichung von einem Bibliothekar freigeschaltet werden muss. Sie bekommt eine String-Variable für den vom Nutzer verfassten Text und ein Boolean, ob die Rezension freigegeben wurde.

\item \textbf{14. Bearbeiten/Löschen/Hinzufügen:}\\
Eine Assoziationsklasse. Ermöglicht Bibliothekare diese drei Funktionen an den Klassen 
Leihobjekt, Ausleihe und Nutzer anzuwenden. Auch beschreibt sie Fähigkeit des Admins,
Bibliothekare zu bearbeiten, zu löschen oder hinzuzufügen.
\end{itemize}
 

  {\em Das Datenmodell im Kontext des Pflichtenhefts ist {\glqq}die
  Darstellung von Informationen und deren Beziehungen in einem
  fachlogischen Konzept{\grqq}. Es soll hier gezeigt werden, welche
  Einheiten für das existierende System relevant sind und welche
  Beziehungen zwischen diesen Einheiten gelten. Es handelt sich
  hierbei noch nicht um ein Datenbankschema oder eine Spezifikation
  von Klassen für die Implementierung (Entwurf), sondern um die
  Modellierung der realen Welt. Das Datenmodell ist leitend für den
  Entwurf (weil alles darin beschrieben sich auch in der Software 
  wiederfinden wird), aber nimmt den Entwurf nicht schon vorweg.
  
  Das Datenmodell soll als UML-Klassendiagramm angegeben werden.
  Wichtig ist hierbei die korrekte Verwendung der UML: Klassen,
  Attribute, Generalisierung, Assoziation, Aggregation, Komposition,
  Multiplizitäten. Außerdem sollte das Diagramm sinnvoll und gut
  lesbar sein. Dazu gehört weiterhin eine kurze Beschreibung des
  Modells mit ergänzenden Informationen, insbesondere wenn die
  Relationen durch ihren Namen nicht selbsterklärend sind. Gebt
  unbedingt ein Mengengerüst für die Daten an: Wie viele Instanzen der
  wichtigsten Klassen werden erwartet? Erwartet Ihr Änderungen im
  Datenvolumen in der Zukunft?}


\subsection{Anwendungsfälle}
\begin{figure}[htbp]
\caption{Startseite}
\includegraphics[width=1\textwidth]{WebApp-Screens/Startscreen-loggedOut.png}
  \label{startseite}
\end{figure}

\begin{table}[htbp]
\label{1}
\begin{tabular}{|l|p{10cm}|}
\hline 
\textbf{1} & \textbf{Programm starten} \\ \hline
\textbf{Akteure} & Bert Bib, Arnold Admin, Silke Schüler, Bart Besucher\\ \hline
\textbf{Ziel} & Der Akteur möchte das Programm starten  \\ \hline
\textbf{Vorbedingungen} & keine \\ \hline
\textbf{Regulärer Ablauf} & 
1. Der Akteur startet das Programm, indem er die URL aufruft \\
&2. Das Programm startet und zeigt die Startseite \\ \hline
\textbf{Varianten} & keine \\ \hline
\textbf{Nachbedingungen} & Das Programm ist gestartet und der Benutzer kann dieses nun verwenden\\ 
\hline
\textbf{Fehler-/Ausnahmefälle} & Server ist nicht erreichbar \\ \hline
\end{tabular}
\end{table}

\begin{figure}[htbp]
\caption{Loginscreen}
\includegraphics[width=1\textwidth]{WebApp-Screens/Loginscreen.png}
  \label{login}
\end{figure}

\begin{table}[htbp]
\label{2}
\begin{tabular}{|l|p{10cm}|}
\hline 
\textbf{2} & \textbf{Benutzer anmelden} \\ \hline
\textbf{Akteure} & Bert Bib, Arnold Admin, Silke Schüler, Bart Besucher\\ \hline
\textbf{Ziel} & Der Akteur möchte sich im System anmelden  \\ \hline
\textbf{Vorbedingungen} & Das Programm wurde gestartet  \\ \hline
\textbf{Regulärer Ablauf} & 
1. Bib gibt seinen Benutzernamen und sein Passwort ein \\
&2. Bert Bib drückt auf den Button anmelden\\
&3. Der Startbildschirm erscheint wieder und Bert Bib kann nun alle Funktionen eines Bibliothekars 
verwenden\\ \hline
\textbf{Varianten} & 
1. Arnold Admin gibt seinen Benutzernamen und sein Passwort ein \\
&2. Arnold Admin drückt auf den Button 'Anmelden'\\
&3. Der Startbildschirm erscheint wieder und Arnold Admin kann nun alle Funktionen eines 
Administrators verwenden\\  
& \\
&1. Silke Schüler gibt ihren Benutzernamen und ihr Passwort ein \\
&2. Silke Schüler drückt auf den Button 'Anmelden'\\
&3. Der Startbildschirm erscheint wieder und Silke Schüler kann nun alle Funktionen eines registrierten 
Nutzers verwenden\\ \hline
\textbf{Nachbedingungen} & Die Personen sind nun angemeldet und können nun Funktionen abhängig 
vom Zugriffsrecht verwenden \\ \hline
\textbf{Fehler-/Ausnahmefälle} & 
1. Bart Besucher besitzt kein Benutzernamen oder Passwort, somit kann er sich nicht anmelden und hat 
keinen Zugriff auf die anderen Funktionen \\
&2. Es wird der falsche Nutzername oder das falsche Passwort eingegeben. Dann erscheint eine 
Fehlermeldung, welche dieses Problem beschreibt \\ \hline
\end{tabular}
\end{table}

\begin{table}[htbp]
\label{3}
\begin{tabular}{|l|p{10cm}|}
\hline 
\textbf{3} & \textbf{Benutzer abmelden} \\ \hline
\textbf{Akteure} & Bert Bib, Arnold Admin, Silke Schüler \\ \hline
\textbf{Ziel} & Der Akteur möchte sich vom System abmelden  \\ \hline
\textbf{Vorbedingungen} & Der Benutzer ist angemeldet  \\ \hline
\textbf{Regulärer Ablauf} & 
1. Ein Benutzer drückt auf den Button 'Abmelden' \\
&2. Das System meldet den Benutzer ab\\
\hline
\textbf{Varianten} & 
keine \\ \hline
\textbf{Nachbedingungen} & Es wird nun die Startseite angezeigt und der Benutzer ist abgemeldet\\
\hline
\textbf{Fehler-/Ausnahmefälle} & keine
\end{tabular}
\end{table}

\begin{figure}[htbp]
\caption{Startseite bei angemeldeten Benutzer}
\includegraphics[width=1\textwidth]{WebApp-Screens/Startscreen-loggedIn.png}
  \label{startlog}
\end{figure}

\begin{table}[htbp]
\label{4}
\begin{tabular}{|l|p{10cm}|}
\hline 
\textbf{4} & \textbf{Startseite anzeigen} \\ \hline
\textbf{Akteure} & Bert Bib, Arnold Admin, Silke Schüler, Bart Besucher\\ \hline
\textbf{Ziel} & Der Akteur möchte die Startseite des Systems aufrufen  \\ \hline
\textbf{Vorbedingungen} & Das Programm wurde gestartet  \\ \hline
\textbf{Regulärer Ablauf} & 
1. Ein Benutzer drückt auf den Button 'Start' \\
&2. Das System zeigt die Startseite an\\
\hline
\textbf{Varianten} & 
Anwendungsfall 1 \\ \hline
\textbf{Nachbedingungen} & Es wird nun die Startseite angezeigt \\ \hline
\textbf{Fehler-/Ausnahmefälle} & keine\\
\hline
\end{tabular}
\end{table}

\begin{figure}[htbp]
\caption{Publikationsscreen von Silke Schüler oder Bart Besucher}
\includegraphics[width=1\textwidth]{WebApp-Screens/PublicationsscreenLogOut.png}
  \label{pub}
\end{figure}

\begin{table}[htbp]
\label{4.1}
\begin{tabular}{|l|p{10cm}|}
\hline 
\textbf{4.1} & \textbf{Leserprofil anzeigen} \\ \hline
\textbf{Akteure} & Silke Schüler\\ \hline
\textbf{Ziel} & Der Akteur möchte sich das eigene Leserprofil anzeigen lassen  \\ \hline
\textbf{Vorbedingungen} & Der Akteur ist angemeldet  \\ \hline
\textbf{Regulärer Ablauf} & 
1. Ein Benutzer drückt auf den Button 'Profil' \\
&2. Das System zeigt die Profilseite an\\
\hline
\textbf{Varianten} & 
keine \\ \hline
\textbf{Nachbedingungen} & Es wird nun die Profilseite angezeigt \\ \hline
\textbf{Fehler-/Ausnahmefälle} & keine\\
\hline
\end{tabular}
\end{table}

\begin{table}[htbp]
\label{4.1.1}
\begin{tabular}{|l|p{10cm}|}
\hline 
\textbf{4.1.1} & \textbf{Vormerkung bearbeiten} \\ \hline
\textbf{Akteure} & Silke Schüler\\ \hline
\textbf{Ziel} & Der Akteur möchte die eigenen Vormerkungen bearbeiten  \\ \hline
\textbf{Vorbedingungen} & Der Benutzer hat sein Profil geöffnet  \\ \hline
\textbf{Regulärer Ablauf} & 
1. Ein Benutzer drückt auf den Button 'Vormerkungen' \\
&2. Das System zeigt eine Liste der Vormerkungen an\\
\hline
\textbf{Varianten} & 
keine \\ \hline
\textbf{Nachbedingungen} & Es wird nun die Vormerkungen angezeigt, die bearbeitet werden können.\\ 
\hline
\textbf{Fehler-/Ausnahmefälle} & keine\\
\hline
\end{tabular}
\end{table}

\begin{table}[htbp]
\label{5}
\begin{tabular}{|l|p{10cm}|}
\hline 
\textbf{5} & \textbf{Publikationen anzeigen} \\ \hline
\textbf{Akteure} & Bert Bib, Arnold Admin, Silke Schüler, Bart Besucher\\ \hline
\textbf{Ziel} & Der Akteur möchte die Liste der Publikationen aufrufen  \\ \hline
\textbf{Vorbedingungen} & Das Programm wurde gestartet  \\ \hline
\textbf{Regulärer Ablauf} & 
1. Ein Benutzer drückt auf den Button 'Publikationen' \\
&2. Das System zeigt die Publikationsliste an\\
\hline
\textbf{Varianten} & 
keine \\ \hline
\textbf{Nachbedingungen} & Es wird nun die Liste mit den Publikationen angezeigt \\ \hline
\textbf{Fehler-/Ausnahmefälle} & keine\\
\hline
\end{tabular}
\end{table}

\begin{table}[htbp]
\label{6}
\begin{tabular}{|l|p{10cm}|}
\hline 
\textbf{6} & \textbf{Buch hinzufügen} \\ \hline
\textbf{Akteure} & Bert Bib\\ \hline
\textbf{Ziel} & Der Akteur möchte ein neues Buch hinzufügen \\ \hline
\textbf{Vorbedingungen} & Der Akteur ist als Bibliothekar angemeldet und hat die Publikationsliste 
aufgerufen  \\ \hline
\textbf{Regulärer Ablauf} & 
1. Der Benutzer drückt auf den Button 'Hinzufügen' \\
&2. Das System zeigt das Formular für das Hinzufügen eines Buches an\\
&3. Der Benutzer drückt den Button 'Speichern'\\
\hline
\textbf{Varianten} & 
keine \\ \hline
\textbf{Nachbedingungen} & Das Buch wurde gespeichert und ist in die Datenbank aufgenommen 
worden\\ \hline
\textbf{Fehler-/Ausnahmefälle} & 1. falsches ISBN-Format wurde eingeben\\
&2. Pflichtfelder wurden nicht eingegeben\\
\hline
\end{tabular}
\end{table}

\begin{table}[htbp]
\label{7}
\begin{tabular}{|l|p{10cm}|}
\hline 
\textbf{7} & \textbf{Buch ändern} \\ \hline
\textbf{Akteure} & Bert Bib\\ \hline
\textbf{Ziel} & Der Akteur möchte ein Daten eines Buches ändern \\ \hline
\textbf{Vorbedingungen} & Der Akteur ist als Bibliothekar angemeldet und hat die Detailsicht eines 
Buches aufgerufen  \\ \hline
\textbf{Regulärer Ablauf} & 
1. Der Benutzer drückt auf den Button 'Ändern' \\
&2. Das System zeigt das Formular für das Hinzufügen eines Buches an\\
&3. Der Benutzer drückt den Button 'Änderung speichern'\\
\hline
\textbf{Varianten} & 
keine \\ \hline
\textbf{Nachbedingungen} & Die Änderungen wurden gespeichert und sind in die Datenbank 
aufgenommen worden\\ \hline
\textbf{Fehler-/Ausnahmefälle} & 1. falsches ISBN-Format wurde eingeben\\
&2. Pflichtfelder wurden nicht eingegeben\\
\hline
\end{tabular}
\end{table}

\begin{table}[htbp]
\label{8}
\begin{tabular}{|l|p{10cm}|}
\hline 
\textbf{8} & \textbf{Buch löschen} \\ \hline
\textbf{Akteure} & Bert Bib\\ \hline
\textbf{Ziel} & Der Akteur möchte ein Buch löschen \\ \hline
\textbf{Vorbedingungen} & Der Akteur ist als Bibliothekar angemeldet und hat die Publikationsliste 
aufgerufen  \\ \hline
\textbf{Regulärer Ablauf} & 
1. Der Benutzer markiert die zu löschenden Bücher\\
&2. Der Benutzer drückt auf den Button 'Löschen' \\
\hline
\textbf{Varianten} & 
1. Der Benutzer befindet sich in der Detailsicht eines Buches\\
&2. Der Benutzer drückt auf den Button 'Löschen' \\ \hline
\textbf{Nachbedingungen} & Das Buch wurde gelöscht \\ \hline
\textbf{Fehler-/Ausnahmefälle} & \\
\hline
\end{tabular}
\end{table}

\begin{table}[htbp]
\label{9}
\begin{tabular}{|l|p{10cm}|}
\hline 
\textbf{9} & \textbf{CVS-Import} \\ \hline
\textbf{Akteure} & Bert Bib\\ \hline
\textbf{Ziel} & Der Akteur möchte eine CVS-Datei für Bücher importieren \\ \hline
\textbf{Vorbedingungen} & Der Akteur ist als Bibliothekar angemeldet und hat die Publikationsliste 
aufgerufen \\ \hline
\textbf{Regulärer Ablauf} & 
1. Der Benutzer drückt auf den Button CVS-Import \\
&2. Der Benutzer kann nun eine CVS-Datei auswählen\\
&3. Der Benutzer drückt den Button 'Importieren'\\
\hline
\textbf{Varianten} & 
keine \\ \hline
\textbf{Nachbedingungen} & Die CVS-Datei wurde hochgeladen und in der Datenbank ergänzt\\\hline
\textbf{Fehler-/Ausnahmefälle} & 1. falsches Datei-Format\\
\hline
\end{tabular}
\end{table}

\begin{table}[htbp]
\label{10}
\begin{tabular}{|l|p{10cm}|}
\hline 
\textbf{10} & \textbf{CVS-Export} \\ \hline
\textbf{Akteure} & Bert Bib\\ \hline
\textbf{Ziel} & Der Akteur möchte eine CVS-Datei von der Datenbank exportieren \\ \hline
\textbf{Vorbedingungen} & Der Akteur ist als Bibliothekar angemeldet und hat die Publikationsliste 
aufgerufen \\ \hline
\textbf{Regulärer Ablauf} & 
1. Der Benutzer drückt auf den Button CVS-Export \\
&2. Der Benutzer kann nun den Speicherort und Name für eine CVS-Datei auswählen\\
&3. Der Benutzer drückt den Button 'Exportieren'\\
\hline
\textbf{Varianten} & 
keine \\ \hline
\textbf{Nachbedingungen} & Die CVS-Datei wurde exportiert und gespeichert\\ \hline
\textbf{Fehler-/Ausnahmefälle} & keine\\
\hline
\end{tabular}
\end{table}

\begin{table}[htbp]
\label{11}
\begin{tabular}{|l|p{10cm}|}
\hline 
\textbf{11} & \textbf{Buch suchen} \\ \hline
\textbf{Akteure} & Bert Bib, Silke Schüler, Bart Besucher, Arnold Admin\\ \hline
\textbf{Ziel} & Der Akteur möchte ein Buch suchen \\ \hline
\textbf{Vorbedingungen} & keine \\ \hline
\textbf{Regulärer Ablauf} & 
1. Der Benutzer gibt den Suchbegriff in das Suchfeld ein und drückt 'Eingabe' \\
&2. Eine Liste von Büchern mit passendem Suchbegriff wird angezeigt\\
\hline
\textbf{Varianten} & 
keine \\ \hline
\textbf{Nachbedingungen} & Eine Liste von Büchern mit passendem Suchbegriff wird angezeigt\\ \hline
\textbf{Fehler-/Ausnahmefälle} & Zum eingegeben Suchbegriff existieren keine Daten\\
\hline
\end{tabular}
\end{table}

\begin{table}[htbp]
\label{12}
\begin{tabular}{|l|p{10cm}|}
\hline 
\textbf{12} & \textbf{Einzelnes Buch anzeigen/ Detailansicht} \\ \hline
\textbf{Akteure} & Bert Bib, Silke Schüler, Bart Besucher, Arnold Admin\\ \hline
\textbf{Ziel} & Der Akteur möchte sich Details zu einem Buch anzeigen lassen \\ \hline
\textbf{Vorbedingungen} & Die Publikationsliste oder die Suchliste wurde aufgerufen \\ \hline
\textbf{Regulärer Ablauf} & 
1. Der Benutzer klickt auf den Button 'Details' bei einem Buch in der Liste \\
&2. Die Detailseite des Buches wird angezeigt\\
\hline
\textbf{Varianten} & 
keine \\ \hline
\textbf{Nachbedingungen} & Die Detailseite eines Buches wird angezeigt\\ \hline
\textbf{Fehler-/Ausnahmefälle} & keine\\
\hline
\end{tabular}
\end{table}

\newpage

\begin{table}[htbp]
\label{13}
\begin{tabular}{|l|p{10cm}|}
\hline 
\textbf{13} & \textbf{Buch bewerten} \\ \hline
\textbf{Akteure} & Silke Schüler\\ \hline
\textbf{Ziel} & Der Akteur möchte ein Buch bewerten \\ \hline
\textbf{Vorbedingungen} & Die Detailansicht eines Buches wurde aufgerufen \\ \hline
\textbf{Regulärer Ablauf} & 
1. Der Benutzer klickt auf den Button 'Bewerten' und kann nun in einem Dropdownmenü eine Punktzahl 
auswählen\\
&2. Der Benutzer drückt den Button 'Buch bewerten'\\
\hline
\textbf{Varianten} & 
keine \\ \hline
\textbf{Nachbedingungen} & Das Buch wurde vom Akteur bewertet und lässt sich kein zweites Mal 
bewerten\\ \hline
\textbf{Fehler-/Ausnahmefälle} & Das Buch wurde schon einmal bewertet\\
\hline
\end{tabular}
\end{table}

\begin{table}[htbp]
\label{14}
\begin{tabular}{|l|p{10cm}|}
\hline 
\textbf{14} & \textbf{Buch ausleihen} \\ \hline
\textbf{Akteure} & Bert Bib, Silke Schüler\\ \hline
\textbf{Ziel} & Silke Schüler möchte ein Buch ausleihen \\ \hline
\textbf{Vorbedingungen} & Bert Bib ist im System als Bibliothekar angemeldet und Silke Schüler ist vor 
Ort \\ \hline
\textbf{Regulärer Ablauf} & 
1. Silke Schüler gibt Buch (Bücher) und ihren Bibliotheksausweis zum Einscannen an Bert Bib\\
&2. Bert Bib scannt erst den Ausweis\\
&3. Nun scannt Bert Bib die Bücher ein\\
&4. Die Liste der auszuleihenden Bücher wird mit dem Ausleiher angezeigt\\
&5. Bert Bib drückt auf den Button 'Ausleihen'\\
\hline
\textbf{Varianten} & 
keine \\ \hline
\textbf{Nachbedingungen} & Die Bücher stehen im System als 'ausgeliehen an Silke Schüler'\\ \hline
\textbf{Fehler-/Ausnahmefälle} & Silke Schüler ist gesperrt und kann keine Bücher ausleihen\\
\hline
\end{tabular}
\end{table}

\begin{table}[htbp]
\label{14.1}
\begin{tabular}{|l|p{10cm}|}
\hline 
\textbf{14.1} & \textbf{Buchrückgabe} \\ \hline
\textbf{Akteure} & Bert Bib, Silke Schüler\\ \hline
\textbf{Ziel} & Der Akteur will Bücher zurückgeben \\ \hline
\textbf{Vorbedingungen} & Bücher sind ausgeliehen \\ \hline
\textbf{Regulärer Ablauf} & 
1. Ein Akteur gibt abzugebene Bücher dem Bibliothekaren \\
&2. Der Bibliothekar scannt die Bücher ein\\
&3. Der Bibliothekar drückt auf den Button 'Bücher zurückgeben'\\
\hline
\textbf{Varianten} & 
Mahngebühren werden bezahlt \\ \hline
\textbf{Nachbedingungen} & Die Bücher stehen im System als zurückgegeben\\ \hline
\textbf{Fehler-/Ausnahmefälle} & keine\\
\hline
\end{tabular}
\end{table}

\begin{table}[htbp]
\label{14.1}
\begin{tabular}{|l|p{10cm}|}
\hline 
\textbf{14.1} & \textbf{Buchrückgabe} \\ \hline
\textbf{Akteure} & Bert Bib, Silke Schüler\\ \hline
\textbf{Ziel} & Der Akteur will Bücher zurückgeben \\ \hline
\textbf{Vorbedingungen} & Bücher sind ausgeliehen \\ \hline
\textbf{Regulärer Ablauf} & 
1. Ein Akteur gibt abzugebene Bücher dem Bibliothekaren \\
&2. Der Bibliothekar scannt die Bücher ein\\
&3. Der Bibliothekar drückt auf den Button 'Bücher zurückgeben'\\
\hline
\textbf{Varianten} & 
Mahngebühren werden bezahlt \\ \hline
\textbf{Nachbedingungen} & Die Bücher stehen im System als zurückgegeben\\ \hline
\textbf{Fehler-/Ausnahmefälle} & keine\\
\hline
\end{tabular}
\end{table}

\begin{table}[htbp]
\label{15}
\begin{tabular}{|l|p{10cm}|}
\hline 
\textbf{15} & \textbf{Buch rezensieren} \\ \hline
\textbf{Akteure} & Silke Schüler\\ \hline
\textbf{Ziel} & Der Akteur will ein Buch rezensieren \\ \hline
\textbf{Vorbedingungen} & Der Akteur befindet sich auf der Detailsicht eines Buches \\ \hline
\textbf{Regulärer Ablauf} & 
1. Der Akteur drückt auf 'Buch rezensieren' \\
&2. Der Akteur schreibt seine Rezension in das entsprechende Feld\\
&3. Der Button 'Rezension abschicken' wird gedrückt\\
\hline
\textbf{Varianten} & 
keine \\ \hline
\textbf{Nachbedingungen} & Die Rezension wird abgeschickt und der Bibliothekar muss diese nun 
freischalten\\ \hline
\textbf{Fehler-/Ausnahmefälle} & Es wurde nichts in das Bedienfeld eingegeben und dann abgeschickt.\\
\hline
\end{tabular}
\end{table}

\begin{table}[htbp]
\label{16}
\begin{tabular}{|l|p{10cm}|}
\hline 
\textbf{16} & \textbf{Buch vormerken} \\ \hline
\textbf{Akteure} & Silke Schüler\\ \hline
\textbf{Ziel} & Der Akteur will ein Buch vormerken \\ \hline
\textbf{Vorbedingungen} & Der Akteur befindet sich auf der Detailsicht eines Buches \\ \hline
\textbf{Regulärer Ablauf} & 
1. Der Akteur drückt auf 'Buch vormerken' \\
&2. Das Buch wurde vorgemerkt\\
\hline
\textbf{Varianten} & 
keine \\ \hline
\textbf{Nachbedingungen} & Das Buch wurde vorgemerkt und erscheint nun auf der Profilseite\\ \hline
\textbf{Fehler-/Ausnahmefälle} & Das Buch wurde bereits vorgemerkt und kann somit nicht noch einmal 
vorgemerkt werden\\
\hline
\end{tabular}
\end{table}

\begin{table}[htbp]
\label{17}
\begin{tabular}{|l|p{10cm}|}
\hline 
\textbf{17} & \textbf{Rezension freischalten} \\ \hline
\textbf{Akteure} & Bert Bib\\ \hline
\textbf{Ziel} & Der Bibliothekar will eine Rezension überprüfen und gegebenenfalls freischalten \\ \hline
\textbf{Vorbedingungen} & Es wurde eine Rezension geschrieben und der Bibliothekar hat diese zur 
Überprüfung erhalten. \\ \hline
\textbf{Regulärer Ablauf} & 
1. Der Akteur liest sich die Rezension durch \\
&2. Der Bibliothekar schaltet die Rezension frei\\
\hline
\textbf{Varianten} & 
1. Der Akteur ließt sich die Rezension durch \\
&2. Der Bibliothekar lehnt die Rezension ab \\ \hline
\textbf{Nachbedingungen} & Die Rezension wurde angenommen und freigeschaltet oder abgelehnt\\ 
\hline
\textbf{Fehler-/Ausnahmefälle} & \\
\hline
\end{tabular}
\end{table}

\begin{table}[htbp]
\label{18}
\begin{tabular}{|l|p{10cm}|}
\hline 
\textbf{18} & \textbf{Leserliste anzeigen} \\ \hline
\textbf{Akteure} & Bert Bib\\ \hline
\textbf{Ziel} & Der Akteur möchte die Liste der Leser aufrufen  \\ \hline
\textbf{Vorbedingungen} & Das Programm wurde gestartet  \\ \hline
\textbf{Regulärer Ablauf} & 
1. Ein Benutzer drückt auf den Button 'Leserliste' \\
&2. Das System zeigt die Leserliste an\\
\hline
\textbf{Varianten} & 
keine \\ \hline
\textbf{Nachbedingungen} & Es wird nun die Liste mit den Lesern angezeigt \\ \hline
\textbf{Fehler-/Ausnahmefälle} & keine\\
\hline
\end{tabular}
\end{table}

\begin{table}[htbp]
\label{19}
\begin{tabular}{|l|p{10cm}|}
\hline 
\textbf{19} & \textbf{Leser hinzufügen} \\ \hline
\textbf{Akteure} & Bert Bib\\ \hline
\textbf{Ziel} & Der Akteur möchte ein neuen Leser hinzufügen \\ \hline
\textbf{Vorbedingungen} & Der Akteur ist als Bibliothekar angemeldet und hat die Leserliste aufgerufen\\
\hline
\textbf{Regulärer Ablauf} & 
1. Der Benutzer drückt auf den Button 'Hinzufügen' \\
&2. Das System zeigt das Formular für das Hinzufügen eines Lesers an\\
&3. Der Bibliothekar füllt das Formular aus\\
&4. Der Benutzer drückt den Button 'Speichern'\\
\hline
\textbf{Varianten} & 
keine \\ \hline
\textbf{Nachbedingungen} & Der Leser wurde gespeichert und ist in die Datenbank aufgenommen 
worden\\ \hline
\textbf{Fehler-/Ausnahmefälle} & 1. Leser existiert bereits (alle Angaben stimmen überein)\\
&2. Pflichtfelder wurden nicht eingegeben\\
\hline
\end{tabular}
\end{table}

\begin{table}[htbp]
\label{20}
\begin{tabular}{|l|p{10cm}|}
\hline 
\textbf{20} & \textbf{Leser ändern} \\ \hline
\textbf{Akteure} & Bert Bib\\ \hline
\textbf{Ziel} & Der Akteur möchte die Daten eines Lesers ändern \\ \hline
\textbf{Vorbedingungen} & Der Akteur ist als Bibliothekar angemeldet und hat die Detailsicht eines 
Lesers aufgerufen  \\ \hline
\textbf{Regulärer Ablauf} & 
1. Der Benutzer drückt auf den Button 'Ändern' \\
&2. Das System zeigt das Formular für das Hinzufügen eines Lesers an\\
&3. Der Bibliothekar ändert das Formular entsprechend\\
&4. Der Benutzer drückt den Button 'Änderung speichern'\\
\hline
\textbf{Varianten} & 
keine \\ \hline
\textbf{Nachbedingungen} & Die Änderungen wurden gespeichert und sind in die Datenbank 
aufgenommen worden\\ \hline
\textbf{Fehler-/Ausnahmefälle} & 1. Leser existiert bereits\\
&2. Pflichtfelder wurden nicht eingegeben\\
\hline
\end{tabular}
\end{table}

\begin{table}[htbp]
\label{21}
\begin{tabular}{|l|p{10cm}|}
\hline 
\textbf{21} & \textbf{Leser löschen} \\ \hline
\textbf{Akteure} & Bert Bib\\ \hline
\textbf{Ziel} & Der Akteur möchte ein Leser löschen \\ \hline
\textbf{Vorbedingungen} & Der Akteur ist als Bibliothekar angemeldet und hat die Leserliste 
aufgerufen  \\ \hline
\textbf{Regulärer Ablauf} & 
1. Der Benutzer markiert die zu löschenden Leser\\
&2. Der Benutzer drückt auf den Button 'Löschen' \\
\hline
\textbf{Varianten} & 
1. Der Benutzer befindet sich in der Detailsicht eines Lesers\\
&2. Der Benutzer drückt auf den Button 'Löschen' \\ \hline
\textbf{Nachbedingungen} & Der Leser wurde gelöscht \\ \hline
\textbf{Fehler-/Ausnahmefälle} & \\
\hline
\end{tabular}
\end{table}

\begin{table}[htbp]
\label{22}
\begin{tabular}{|l|p{10cm}|}
\hline 
\textbf{22} & \textbf{CVS-Import} \\ \hline
\textbf{Akteure} & Bert Bib\\ \hline
\textbf{Ziel} & Der Akteur möchte eine CVS-Datei für Leser importieren \\ \hline
\textbf{Vorbedingungen} & Der Akteur ist als Bibliothekar angemeldet und hat die Leserliste aufgerufen\\
\hline
\textbf{Regulärer Ablauf} & 
1. Der Benutzer drückt auf den Button 'CVS-Import' \\
&2. Der Benutzer kann nun eine CVS-Datei auswählen\\
&3. Der Benutzer drückt den Button 'Importieren'\\
\hline
\textbf{Varianten} & 
keine \\ \hline
\textbf{Nachbedingungen} & Die CVS-Datei wurde hochgeladen und in der Datenbank ergänzt\\ \hline
\textbf{Fehler-/Ausnahmefälle} & 1. falsches Datei-Format\\
\hline
\end{tabular}
\end{table}

\begin{table}[htbp]
\label{23}
\begin{tabular}{|l|p{10cm}|}
\hline 
\textbf{23} & \textbf{CVS-Export} \\ \hline
\textbf{Akteure} & Bert Bib\\ \hline
\textbf{Ziel} & Der Akteur möchte eine CVS-Datei von der Datenbank exportieren \\ \hline
\textbf{Vorbedingungen} & Der Akteur ist als Bibliothekar angemeldet und hat die Leserliste 
aufgerufen \\ \hline
\textbf{Regulärer Ablauf} & 
1. Der Benutzer drückt auf den Button 'CVS-Export' \\
&2. Der Benutzer kann nun den Speicherort und Namen für eine CVS-Datei auswählen\\
&3. Der Benutzer drückt den Button 'Exportieren'\\
\hline
\textbf{Varianten} & 
keine \\ \hline
\textbf{Nachbedingungen} & Die CVS-Datei wurde exportiert und gespeichert\\ \hline
\textbf{Fehler-/Ausnahmefälle} & keine\\
\hline
\end{tabular}
\end{table}

\begin{table}[htbp]
\label{24}
\begin{tabular}{|l|p{10cm}|}
\hline 
\textbf{24} & \textbf{Einzelnen Leser anzeigen/ Detailansicht} \\ \hline
\textbf{Akteure} & Bert Bib\\ \hline
\textbf{Ziel} & Der Akteur möchte sich Details zu einem Leser anzeigen lassen \\ \hline
\textbf{Vorbedingungen} & Die Leserliste wurde aufgerufen \\ \hline
\textbf{Regulärer Ablauf} & 
1. Der Benutzer klickt auf den Button 'Details' bei einem Leser in der Liste \\
&2. Die Detailseite des Lesers wird angezeigt\\
\hline
\textbf{Varianten} & 
keine \\ \hline
\textbf{Nachbedingungen} & Die Detailseite eines Lesers wird angezeigt\\ \hline
\textbf{Fehler-/Ausnahmefälle} & keine\\
\hline
\end{tabular}
\end{table}

\begin{table}[htbp]
\label{24.1}
\begin{tabular}{|l|p{10cm}|}
\hline 
\textbf{24.1} & \textbf{Leser sperren} \\ \hline
\textbf{Akteure} & Bert Bib, Silke Schüler\\ \hline
\textbf{Ziel} & Der Bibliothekar sperrt einen Leser \\ \hline
\textbf{Vorbedingungen} & Die Detailsicht eines Lesers wurde aufgerufen \\ \hline
\textbf{Regulärer Ablauf} & 
1. Ein Leser gibt die ausgeliehenen nicht wieder zurück\\
&2. Der Bibliothekar sperrt den Nutzer\\
\hline
\textbf{Varianten} & 
1. Ein Leser verwendet seinen Account nicht ordnungsgemäß\\
&2. Der Bibliothekar sperrt den Nutzer\\ \hline
\textbf{Nachbedingungen} & Der Benutzer wurde gesperrt und kann keine Bücher mehr vormerken oder 
ausleihen\\ \hline
\textbf{Fehler-/Ausnahmefälle} & keine\\
\hline
\end{tabular}
\end{table}

\begin{table}[htbp]
\label{25}
\begin{tabular}{|l|p{10cm}|}
\hline 
\textbf{25} & \textbf{Leser suchen} \\ \hline
\textbf{Akteure} & Bert Bib\\ \hline
\textbf{Ziel} & Der Akteur möchte ein Leser suchen \\ \hline
\textbf{Vorbedingungen} & Die Leserliste wurde aufgerufen \\ \hline
\textbf{Regulärer Ablauf} & 
1. Der Bibliothekar gibt den Suchbegriff in das Suchfeld ein und drückt 'Eingabe' \\
&2. Eine Liste von Lesern mit passendem Suchbegriff wird angezeigt\\
\hline
\textbf{Varianten} & 
keine \\ \hline
\textbf{Nachbedingungen} & Eine Liste von Lesern mit passendem Suchbegriff wird angezeigt\\ \hline
\textbf{Fehler-/Ausnahmefälle} & Zum eingegeben Suchbegriff existieren keine Daten\\
\hline
\end{tabular}
\end{table}

\begin{table}[htbp]
\label{26}
\begin{tabular}{|l|p{10cm}|}
\hline 
\textbf{26} & \textbf{Administration öffnen} \\ \hline
\textbf{Akteure} & Bert Bib, Arnold Admin\\ \hline
\textbf{Ziel} & Der Akteur will die Administratorseite anzeigen lassen \\ \hline
\textbf{Vorbedingungen} & Der Akteur ist im System angemeldet \\ \hline
\textbf{Regulärer Ablauf} & 
1. Der Akteur klickt auf den Button 'Administration' \\
\hline
\textbf{Varianten} & 
keine \\ \hline
\textbf{Nachbedingungen} & Der Akteur befindet sich nun auf der Administrationsseite\\ \hline
\textbf{Fehler-/Ausnahmefälle} & keine\\
\hline
\end{tabular}
\end{table}

\newpage

\begin{table}[htbp]
\label{27}
\begin{tabular}{|l|p{10cm}|}
\hline 
\textbf{27} & \textbf{Bibliothekarliste anzeigen} \\ \hline
\textbf{Akteure} & Arnold Admin\\ \hline
\textbf{Ziel} & Der Akteur will die Liste der Bibliothekare einsehen \\ \hline
\textbf{Vorbedingungen} & Der Akteur ist im System angemeldet \\ \hline
\textbf{Regulärer Ablauf} & 
1. Der Akteur klickt auf den Button 'Administration' \\
&2. Der Akteur klickt auf den Button 'Bibliothekare'\\
\hline
\textbf{Varianten} & 
keine \\ \hline
\textbf{Nachbedingungen} & Der Akteur befindet sich nun auf der Seite, die Bibliothekare in einer 
Liste anzeigen\\ \hline
\textbf{Fehler-/Ausnahmefälle} & keine\\
\hline
\end{tabular}
\end{table}

\begin{table}[htbp]
\label{28}
\begin{tabular}{|l|p{10cm}|}
\hline 
\textbf{28} & \textbf{Bibliothekar hinzufügen} \\ \hline
\textbf{Akteure} & Arnold Admin\\ \hline
\textbf{Ziel} & Der Akteur will einen neuen Bibliothekaren hinzufügen \\ \hline
\textbf{Vorbedingungen} & Der Akteur ist als Admin angemeldet und hat die Bibliothekarsliste 
aufgerufen \\ \hline
\textbf{Regulärer Ablauf} & 
1. Der Akteur klickt auf den Button 'Hinzufügen' \\
&2. Der Admin füllt das Formular aus und klickt auf 'Speichern'\\
\hline
\textbf{Varianten} & 
keine \\ \hline
\textbf{Nachbedingungen} & Der Akteur befindet sich nun auf der Seite, die die Bibliothekarsliste 
anzeigt\\ \hline
\textbf{Fehler-/Ausnahmefälle} & keine\\
\hline
\end{tabular}
\end{table}


\begin{table}[htbp]
\label{29}
\begin{tabular}{|l|p{10cm}|}
\hline 
\textbf{29} & \textbf{Bibliothekar löschen} \\ \hline
\textbf{Akteure} & Arnold Admin\\ \hline
\textbf{Ziel} & Der Akteur will einen Bibliothekaren löschen \\ \hline
\textbf{Vorbedingungen} & Der Akteur ist als Admin angemeldet und hat die Bibliothekarsliste 
aufgerufen\\\hline
\textbf{Regulärer Ablauf} & 
1. Der Akteur klickt auf einen Bibliothekaren \\
&2. Der Admin klickt nun auf der Detailseite auf 'Löschen'\\
\hline
\textbf{Varianten} & 
keine \\ \hline
\textbf{Nachbedingungen} & Der Akteur befindet sich nun auf der Seite, die die Bibliothekarsliste 
anzeigt\\ \hline
\textbf{Fehler-/Ausnahmefälle} & keine\\
\hline
\end{tabular}
\end{table}

\begin{table}[htbp]
\label{30}
\begin{tabular}{|l|p{10cm}|}
\hline 
\textbf{30} & \textbf{Bibliothekar ändern} \\ \hline
\textbf{Akteure} & Arnold Admin\\ \hline
\textbf{Ziel} & Der Akteur will Daten eines Bibliothekaren ändern \\ \hline
\textbf{Vorbedingungen} & Der Akteur ist als Admin angemeldet und hat die Bibliothekarsliste 
aufgerufen \\ \hline
\textbf{Regulärer Ablauf} & 
1. Der Akteur klickt auf einen Bibliothekaren \\
&2. Der Akteur klickt auf der Detailseite auf 'Ändern '\\
&3. Der Akteur füllt das Formular aus und klickt auf 'Speichern'\\
\hline
\textbf{Varianten} & 
keine \\ \hline
\textbf{Nachbedingungen} & Der Akteur befindet sich nun auf der Seite, die die Bibliothekarliste 
anzeigt\\ \hline
\textbf{Fehler-/Ausnahmefälle} & keine\\
\hline
\end{tabular}
\end{table}


\begin{table}[htbp]
\label{31}
\begin{tabular}{|l|p{10cm}|}
\hline 
\textbf{31} & \textbf{Statistik anzeigen} \\ \hline
\textbf{Akteure} & Bert Bib\\ \hline
\textbf{Ziel} & Der Akteur will die Statistiken einsehen \\ \hline
\textbf{Vorbedingungen} & Der Akteur ist als Bibliothekar angemeldet und hat die Administrationsseite 
aufgerufen \\ \hline
\textbf{Regulärer Ablauf} & 
1. Der Akteur klickt auf den Button 'Statistiken' \\
\hline
\textbf{Varianten} & 
keine \\ \hline
\textbf{Nachbedingungen} & Der Akteur befindet sich nun auf der Seite, die die Statistiken anzeigt\\
\hline
\textbf{Fehler-/Ausnahmefälle} & keine\\
\hline
\end{tabular}
\end{table}


\begin{table}[htbp]
\label{32}
\begin{tabular}{|l|p{10cm}|}
\hline 
\textbf{32} & \textbf{Mahnungsliste anzeigen} \\ \hline
\textbf{Akteure} & Bert Bib\\ \hline
\textbf{Ziel} & Der Akteur will die Mahnungsliste einsehen \\ \hline
\textbf{Vorbedingungen} & Der Akteur ist als Bibliothekar angemeldet und hat die Administrationsseite 
aufgerufen \\ \hline
\textbf{Regulärer Ablauf} & 
1. Der Akteur klickt auf den Button 'Mahnungen' \\
\hline
\textbf{Varianten} & 
keine \\ \hline
\textbf{Nachbedingungen} & Der Akteur befindet sich nun auf der Seite, die die Mahnungsliste 
anzeigt\\ \hline
\textbf{Fehler-/Ausnahmefälle} & keine\\
\hline
\end{tabular}
\end{table}

\begin{table}[htbp]
\label{33}
\begin{tabular}{|l|p{10cm}|}
\hline 
\textbf{33} & \textbf{Mahnungsliste drucken} \\ \hline
\textbf{Akteure} & Bert Bib\\ \hline
\textbf{Ziel} & Der Akteur will die Mahnungsliste ausdrucken \\ \hline
\textbf{Vorbedingungen} & Der Akteur ist als Bibliothekar angemeldet und hat die Mahnungsliste 
aufgerufen \\ \hline
\textbf{Regulärer Ablauf} & 
1. Der Akteur klickt auf den Button 'Drucken' \\
\hline
\textbf{Varianten} & 
Einzelne Mahnungen werden ausgewählt damit nur diese ausgedruckt werden \\ \hline
\textbf{Nachbedingungen} & Der Akteur befindet sich nun auf der Seite, die die Mahnungsliste 
anzeigt\\ \hline
\textbf{Fehler-/Ausnahmefälle} & Probleme beim Drucken\\
\hline
\end{tabular}
\end{table}

\begin{table}[htbp]
\label{34}
\begin{tabular}{|l|p{10cm}|}
\hline 
\textbf{34} & \textbf{Mahnungsdetails anzeigen} \\ \hline
\textbf{Akteure} & Bert Bib\\ \hline
\textbf{Ziel} & Der Akteur will sich Details zu einer Mahnung anschauen \\ \hline
\textbf{Vorbedingungen} & Der Akteur ist als Bibliothekar angemeldet und hat die Mahnungsliste 
aufgerufen \\ \hline
\textbf{Regulärer Ablauf} & 
1. Der Akteur klickt auf eine Mahnung \\
\hline
\textbf{Varianten} & 
keine \\ \hline
\textbf{Nachbedingungen} & Der Akteur befindet sich nun auf der Seite, die die Mahnungsdetails 
anzeigt\\ \hline
\textbf{Fehler-/Ausnahmefälle} & keine\\
\hline
\end{tabular}
\end{table}

\begin{table}[htbp]
\label{35}
\begin{tabular}{|l|p{10cm}|}
\hline 
\textbf{35} & \textbf{Startseite bearbeiten} \\ \hline
\textbf{Akteure} & Bert Bib\\ \hline
\textbf{Ziel} & Der Akteur will die Startseite bearbeiten \\ \hline
\textbf{Vorbedingungen} & Der Akteur ist als Bibliothekar angemeldet und hat die 
Administrationsseite aufgerufen\\\hline
\textbf{Regulärer Ablauf} & 
1. Der Akteur klickt auf den Button 'Startseite bearbeiten' \\
&2. Der Akteur bearbeitet die Startseite nach seinen Wünschen und klickt auf 'Speichern'\\
\hline
\textbf{Varianten} & 
keine \\ \hline
\textbf{Nachbedingungen} & Der Akteur befindet sich nun auf der (neuen) Startseite \\ \hline
\textbf{Fehler-/Ausnahmefälle} & keine\\
\hline
\end{tabular}
\end{table}

\begin{table}[htbp]
\label{36}
\begin{tabular}{|l|p{10cm}|}
\hline 
\textbf{36} & \textbf{Abgabedaten und Mahngebühren bearbeiten} \\ \hline
\textbf{Akteure} & Bert Bib\\ \hline
\textbf{Ziel} & Der Akteur will Abgabedaten und Mahngebühren bearbeiten \\ \hline
\textbf{Vorbedingungen} & Der Akteur ist als Bibliothekar angemeldet und hat die Administrationsseite 
aufgerufen \\ \hline
\textbf{Regulärer Ablauf} & 
1. Der Akteur klickt auf den Button 'Abgabedaten/Mahngebühren bearbeiten' \\
&2. Der Akteur bearbeitet die Daten und/oder Gebühren und klickt auf 'Speichern'\\
\hline
\textbf{Varianten} & 
keine \\ \hline
\textbf{Nachbedingungen} & Der Akteur befindet sich auf der gleichen Seite und kann die neuen 
Daten/Gebühren sehen\\ \hline
\textbf{Fehler-/Ausnahmefälle} & keine\\
\hline
\end{tabular}
\end{table}

\newpage
\subsection{Aktionen}
  {\em Hier sollten die gleichen Aktionen wie in den Anwendungsfällen
  genannt und genauer beschrieben werden. Mit anderen Worten: Die
  Anwendungsfälle müssen vollständig durch Ausführung von Aktionen aus
  dieser Liste durchführbar sein. Im Prinzip muss es z.B.\ für jeden
  Button/Menüpunkt/Link eine Aktion geben. Dabei ist zu beachten:
  \begin{itemize}
    \item Die Namen sollten sinnvoll und eindeutig sein.

    \item Die Parameter der Aktionen sollen angegeben werden. Hier
    sollen sprechende Namen verwendet werden. Eventuell müssen die
    Parameter auch genauer erläutert werden.

    \item Es müssen maximale Ausführungszeiten für jede Operation
    angegeben werden.
    
  \item Die Gruppierung und Sortierung sollte sinnvoll sein
    (z.B. alphabetisch).
  \end{itemize}

  Wenn Ihr z.B.\ irgendwo in Eurer GUI ein Suchfeld habt, in das Ihr
  den Namen eines Kunden eintragen könnt, und einen Button, welcher die
  Suche startet, dann wird es vermutlich eine Aktion {\bf Kunde
    suchen(name)} geben. Dies ist eine Funktion, die Euer System
  bereitstellt und die durch Anklicken des Buttons ausgelöst wird. Der
  Anwendungsfall {\bf Kunde suchen} verwendet dann diese Aktion,
  enthält aber zusätzlich die Beschreibung der Interaktion mit dem
  System.
  
  Dieser Abschnitt ist im Standard im Prinzip vorgesehen, weil hierzu
  grundsätzlich eine Aussage gemacht werden muss. Die Aktionen sind
  letztlich die Produktfunktionen, während die Anwendungsfälle die
  Interaktion zwischen Akteuren und System beschreiben. }

  \begin{table}[htbp]
  \label{a1}
  \begin{tabular}{|l|p{10cm}|}
  \hline 
  \textbf{1} & \textbf{Abgabedaten und Mahngebühren bearbeiten} \\ \hline
  \textbf{Beschreibung} & Hier kann der Bibliothekar die Abgabedaten und Mahngebühren bearbeiten\\ \hline
  \textbf{Parameter} & Der Zeitraum der Ausleihdauer und die Höhe der Gebühren \\ \hline
  \textbf{Ausführungszeit} & 1s\\ \hline
  \end{tabular}
  \end{table}
  
  \begin{table}[htbp]
  \label{a2}
  \begin{tabular}{|l|p{10cm}|}
  \hline 
  \textbf{2} & \textbf{Abmelden} \\ \hline
  \textbf{Beschreibung} & Über den Button meldet sich der Nutzer ab\\ \hline
  \textbf{Parameter} & keine \\ \hline
  \textbf{Ausführungszeit} & 1s\\ \hline
  \end{tabular}
  \end{table}
  
  \begin{table}[htbp]
  \label{a3}
  \begin{tabular}{|l|p{10cm}|}
  \hline 
  \textbf{3} & \textbf{Administration öffnen} \\ \hline
  \textbf{Beschreibung} & Die Seite der Administration öffnet sich\\ \hline
  \textbf{Parameter} & keine \\ \hline
  \textbf{Ausführungszeit} & 2s\\ \hline
  \end{tabular}
  \end{table}
  
  \begin{table}[htbp]
  \label{a4}
  \begin{tabular}{|l|p{10cm}|}
  \hline 
  \textbf{4} & \textbf{Anmelden} \\ \hline
  \textbf{Beschreibung} & Der Nutzer meldet sich im System an\\ \hline
  \textbf{Parameter} & Nutzername und Passwort \\ \hline
  \textbf{Ausführungszeit} & 3s\\ \hline
  \end{tabular}
  \end{table}
  
  Bibliothekar ändern
  Bibliothekar hinzufügen
  Bibliothekar löschen
  Bibliothekarsliste anzeigen
  Buch ändern
  Buch ausleihen
  Buch bewerten
  Buch hinzufügen
  Buch löschen
  Buch Rezension bestätigen
  Buch suchen
  Buch vormerken
  Buchrückgabe
  CVS-Import
  CVS-Export
  Einzelnes Buch anzeigen/ Detailansicht
  Leser ändern
  Leser hinzufügen
  Leser löschen
  Leser sperren
  Leser suchen
  Leserliste anzeigen
  Leserprofil anzeigen
  Mahnungsdetails anzeigen
  Mahnungsliste anzeigen
  Mahnungsliste drucken
  Publikationen anzeigen
  Rezension freischalten
  Start anzeigen
  Startseite bearbeiten
  Statistiken anzeigen
  
  
\subsection{Entwurfseinschränkungen}
\nurlangversion

{\em Wurde bereits in \ref{sec:Einschraenkungen} behandelt und muss
  daher hier nicht wiederholt werden. Falls aber eine detailliertere
  Beschreibung notwendig wäre, wäre hier der geeignete Ort.}
  

\subsection{Softwaresystemattribute}
  {\em Hier werden die sogenannten „nichtfunktionalen Anforderungen“
  spezifiziert. Dazu gehören beispielsweise:
  \begin{itemize}
    \item Performanz:
    \item Zuverlässigkeit (Korrektheit, Robustheit, Ausfallsicherheit):
    \item Verfügbarkeit:
    \item Sicherheit:
     Sicherheit bezüglich der persönlichen Daten wird zum Teil durch den 
     passwortgeschützten Login gewährleistet, wobei jeder Nutzer ein individuelles 
     Passwort besitzt. Auf Profildaten haben nur der bestimmte eingeloggte          
     Leser und die Mitarbeiter Zugriff. Zusätzlich werden vor dem Versenden von Daten, 
     diese via SSL verschlüsselt, was die Datensicherheit in unserem System 
     garantiert.
    \item Wartbarkeit:
    \item Portabilität:
    
  \end{itemize}
}

{\em Die spezifizierten Systemattribute müssen hinreichend konkret und
  überprüfbar formuliert werden.}



\subsection{Weitere Anforderungen}
\nurlangversion

{\em In diesem Abschnitt können weitere relevante Anforderungen
  beschrieben werden, die in keine der oben genannten Abschnitte
  passen.}

\section{Anhang}
\nurlangversion

{\em Hier können weitere detailliertere Ergebnisse aus der Ist-Analyse
  oder andere Informationen, die zur Erstellung der Spezifikation
  gedient haben (z.B. Papierprototypen), angefügt werden.}

\end{document}
