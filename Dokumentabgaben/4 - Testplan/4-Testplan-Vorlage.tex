\documentclass[fontsize=12pt,paper=a4,twoside]{scrartcl}

\newcommand{\grad}{\ensuremath{^{\circ}} }
\renewcommand{\strut}{\vrule width 0pt height5mm depth2mm}

\usepackage[utf8]{inputenc}
\usepackage[final]{pdfpages}
% obere Seitenränder gestalten können
\usepackage{fancyhdr}
\usepackage{moreverb}
% Graphiken als jpg, png etc. einbinden können
\usepackage{graphicx}
\usepackage{stmaryrd}
% Floats Objekte mit [H] festsetzen
\usepackage{float}
% setzt URL's schön mit \url{http://bla.laber.com/~mypage}
\usepackage{url}
% Externe PDF's einbinden können
\usepackage{pdflscape}
% Verweise innerhalb des Dokuments schick mit " ... auf Seite ... "
% automatisch versehen. Dazu \vref{labelname} benutzen
\usepackage[ngerman]{varioref}
\usepackage[ngerman]{babel}
\usepackage{ngerman}
% Bibliographie
\usepackage{bibgerm}
% Tabellen
\usepackage{tabularx}
\usepackage{supertabular}
\usepackage[colorlinks=true, pdfstartview=FitV, linkcolor=blue,
            citecolor=blue, urlcolor=blue, hyperfigures=true,
            pdftex=true]{hyperref}
\usepackage{bookmark}


\newboolean{langversion} %Deklaration
\setboolean{langversion}{true} %Zuweisung ist 'false' für Blockkurs
\newcommand{\highlight}[1]{\textcolor{blue}{\textbf{#1}}}
\newcommand{\nurlangversion}[0]{%
\ifthenelse{\boolean{langversion}}{\highlight{Muss in SWP-2 ausgefüllt werden}}{\highlight{Entfällt in SWP-1}}}

% erstes Argument: SWP-2, zweites SWP-1
\newcommand{\variante}[2]{\ifthenelse{\boolean{langversion}}{#1}{#2}}


% Damit Latex nicht zu lange Zeilen produziert:
\sloppy
%Uneinheitlicher unterer Seitenrand:
%\raggedbottom

% Kein Erstzeileneinzug beim Absatzanfang
% Sieht aber nur gut aus, wenn man zwischen Absätzen viel Platz einbaut
\setlength{\parindent}{0ex}

% Abstand zwischen zwei Absätzen
\setlength{\parskip}{1ex}

% Seitenränder für Korrekturen verändern
\addtolength{\evensidemargin}{-1cm}
\addtolength{\oddsidemargin}{1cm}

\bibliographystyle{gerapali}

% Lustige Header auf den Seiten
  \pagestyle{fancy}
  \setlength{\headheight}{70.55003pt}
  \fancyhead{}
  \fancyhead[LO,RE]{Software--Projekt 1\\  2013
  \\Testplan}
  \fancyhead[LE,RO]{Seite \thepage\\\slshape \leftmark\\\slshape \rightmark}

%
% Und jetzt geht das Dokument los....
%

\begin{document}

% Lustige Header nur auf dieser Seite
  \thispagestyle{fancy}
  \fancyhead[LO,RE]{ }
  \fancyhead[LE,RO]{Universität Bremen\\FB 3 -- Informatik\\
  Prof. Dr. Rainer Koschke \\TutorIn: Euer/Eure TutorIn}
  \fancyfoot[C]{}

% Start Titelseite
  \vspace{3cm}

  \begin{minipage}[H]{\textwidth}
  \begin{center}
  \bf
  \Large
  Software--Projekt 1 2013\\
  \smallskip
  \small
  VAK 03-BA-901.02\\
  \vspace{3cm}
  \end{center}
  \end{minipage}
  \begin{minipage}[H]{\textwidth}
  \begin{center}
  \vspace{1cm}
  \bf
  \Large Testplan\\
  \vfill
  \end{center}
  \end{minipage}
  \vfill
  \begin{minipage}[H]{\textwidth}
  \begin{center}
  \sf
  \begin{tabular}{lrr}
  xxxxxx xxxxxxx & xxxxxxxx@tzi.de & 1234567\\
  xxxx xxxxxxxx & xxxx@tzi.de & 2345678\\
  \end{tabular}
  \\ ~
  \vspace{2cm}
  \\
  \it Abgabe: TT. Monat JJJJ --- Version 1.1\\ ~
  \end{center}
  \end{minipage}

% Ende Titelseite

% Start Leerseite

\newpage

  \thispagestyle{fancy}
  \fancyhead{}
  \fancyhead[LO,RE]{Software--Projekt \\  2013
  \\Testplan}
  \fancyhead[LE,RO]{Seite \thepage\\\slshape \leftmark\\~}
  \fancyfoot{}
  \renewcommand{\headrulewidth}{0.4pt}
  \tableofcontents

\newpage

  \fancyhead[LE,RO]{Seite \thepage\\\slshape \leftmark\\\slshape \rightmark}


%%%%%%%%%%%%%%%%%%%%%%%%%%%%%%%%%%%%%%%%%%%%%%%%%%%%%%%%%%%%%%%%%%%%%%%%
\section*{Version und Änderungsgeschichte}

{\em Die aktuelle Versionsnummer des Dokumentes sollte eindeutig und gut zu
identifizieren sein, hier und optimalerweise auf dem Titelblatt.}

\begin{tabular}{ccl}
Version & Datum & Änderungen \\
\hline
1.0 & TT.MM.JJJJ & Dokumentvorlage als initiale Fassung kopiert \\
1.1 & TT.MM.JJJJ & .... \\
\end{tabular}


%%%%%%%%%%%%%%%%%%%%%%%%%%%%%%%%%%%%%%%%%%%%%%%%%%%%%%%%%%%%%%%%%%%%%%%%
\section{Einführung}\label{c01}

\subsection{Zweck}
\nurlangversion

  {\em Was ist der Zweck dieses Testplans? Wer sind
  die LeserInnen?}

\subsection{Umfang}
\nurlangversion

\subsection{Beziehungen zu anderen Dokumenten}
\nurlangversion


\subsection{Aufbau der Testbezeichner}
\label{sec:aufb-der-testb}

\emph{Jeder Testfall muss einen eindeutigen Bezeichner haben. Der könnte
hierarchisch aufgebaut sein. Hier wird dieses Namensschema einheitlich festgelegt.}

%%
%% Dokumentation der Testergebnisse 
%%
\subsection{Dokumentation der Testergebnisse}
\nurlangversion

{\em In welcher Form werden die Testergebnisse dokumentiert
($\Rightarrow$ Testprotokoll)? Wie werden gefundene Fehler dokumentiert
(Bugtracking)?}


%%
%% Definitionen und Referenzen 
%%
\subsection{Definitionen und Akronyme}
\label{c00b}


\subsection{Referenzen}

\bibliographystyle{plain}
\bibliography{literatur}


% Systemüberblick
\section{Systemüberblick}\label{c02}
\nurlangversion

{\em Dieser Abschnitt soll einen Überblick über das System geben, und
  zwar insbesondere hinsichtlich jener Komponenten, die durch
  Komponententests geprüft werden sollen. Dazu ist die richtige
  Granularität der Komponenten wichtig (einzelne Klassen, ganze Pakete
  oder Systeme etc.). Für den Integrationstest sind die Abhängigkeiten
  der Komponenten voneinander von besonderer Wichtigkeit.}

{\em Der Überblick sollte leicht verständlich sein und sich auf die
  wesentlichen benötigten Informationen konzentrieren. Hier sind
  evtl.\ auch Referenzen auf die Architekturbeschreibung angebracht.}

{\em Es sollten die Hofmeister-Konzepte verwendet
  werden.\ Benutzt am besten Diagramme aus der Architekturbeschreibung.}

\emph{Dieser Abschnitt ist nicht einfach nur Copy\&Paste der
  Architekturbeschreibung. Es gilt einerseits, Redundanzen soweit wie
  möglich zu vermeiden, und andererseits, dieses Dokument so
  selbsterklärend wie möglich zu machen.}

\subsection{Module der Anwendungsschicht und deren Funktionen}
\label{mod-controller}
\nurlangversion

{\em Hier solltet ihr dann die einzelnen Module aus Sicht des
Tests weiter verfeinern. Auch wäre noch eine Grafik angebracht, die
die Abhängigkeiten der Subsysteme/Module zeigt, sowie eine kurze
Beschreibung dazu.}


\clearpage

\section{Merkmale}

{\em Hier sollen die zu testenden Merkmale, aber auch die nicht zu
  testenden Merkmale (mit Begründung) aufgelistet werden. Welche
  Kombinationen von Merkmalen sind relevant?}

\subsection{Zu testende Merkmale}\label{c04}



\subsubsection{Funktionale Anforderungen} 

{\em Welche sind besonders wichtig?}

\subsection{Nicht zu testende Merkmale}\label{c05}

% Abnahme- und Testkriterien
\section{Abnahme- und Testendekriterien}\label{c07}

{\em Wann wird das Testen beendet? Die angegebenen Kriterien müssen
  objektiv prüfbar sein.}



% Vorgehensweise
\section{Vorgehensweise}\label{c06}

\subsection{Komponenten- und Integrationstest}


{\em Hier findet sich das konkrete Vorgehen bei der Integration: Welche
  Klassen werden zunächst zusammen getestet, welche kommen dann hinzu?
 Das kann man z.B.\ geeignet in Form eines Baumes aufzeigen.}


\subsection{Funktionstest}



\section{Aufhebung und Wiederaufnahme}\label{c08}


\section{Hardware- und Softwareanforderungen}\label{c09}
\nurlangversion

{\em Hier soll das zu prüfende Material spezifiziert werden. Auf
  welchen Rechnern werden die Tests durchgeführt? Welche anderen
  Hilfsmittel oder Ressourcen sind dafür nötig?}

\subsection{Hardware}

\nurlangversion

\subsection{Software}
\nurlangversion

% Testfälle
\section{Testfälle}\label{c10}

{\em Dies ist der wichtigste (und vermutlich umfangreichste) Teil des
  Testplans. Hier wird genau aufgelistet, {\bf was wie} und {\bf von
  wem} getestet wird. Das spätere Testen besteht dann einfach aus
  einer Durchführung dieser Tests.}

{\em Zu jedem Testfall muss es eine {\bf Testfallspezifikation}
  geben (außer Komponententests, siehe unten). Schwerpunkt sollen
  hier Integrations- und Leistungstests sein. Welche Arten von
  Integrationstests und Leistungstests seht Ihr vor und wie wollt Ihr
  diese genau ausgestalten?}

\subsection{Komponententest}\label{c10-0}

{\em Auf die genaue Spezifikation der Komponententests könnt Ihr
  verzichten -- diese werden durch die JUnit-Testfälle gegeben. Uns
  genügt hierzu eine Beschreibung, welche Komponenten/Klassen wie,
  wann und durch wen getestet werden sollen. Die zugehörigen
  JUnit-Tests werden separat abgegeben (Black-Box jetzt, White-Box
  später bei der Abgabe der Implementierung) und müssen im Testplan
  nicht als Code aufgeführt werden.}

\begin{table}[h]
\centering
\begin{tabular}{|l|p{3cm}|p{3cm}|l|}
\hline
Klasse & Implementierer & Tester & Testart \\
\hline
ReferenceManagement & Klaus  & Hans    & Blackbox \\
UserManagement      & Anna   & Klaus   & Blackbox \\
\dots               & \dots  & \dots   & \dots \\
ReferenceTable      & Heinz  & Bert    & Whitebox \\
\hline
\end{tabular}
\caption{Komponententests}
\label{tab:komponententests}
\end{table}

{\em Achtet darauf, dass der Tester ein anderer ist als der
  Implementierer. Jeder soll mindestens eine Klasse als Black-Box und
  eine Klasse als White-Box testen!}

{\em Hier könnten später noch die Kontrollflussdiagramme für die
  Whiteboxtests eingefügt werden.}

\clearpage
\subsection{Integrationstest}\label{c10a}

{\em Die Integrationstests werden hier genauer beschrieben: Welche
  Klassen sind beteiligt? Wie ist der Zustand des Systems vor Beginn?
  Welche Eingaben werden getätigt? Welche Ergebnisse/welches Verhalten
  wird erwartet?}

{\em Aufbau einer Testfallspezifikation:
\begin{enumerate}
\item eindeutiger Testfallbezeichner (entspricht Namenskonvention aus Abschnitt~\ref{sec:aufb-der-testb})
\item Testobjekte: welche Komponenten werden getestet?
\item Eingabespezifikationen (Eingaben des Testfalls)
\item Ausgabespezifikationen (erwartete Ausgaben)
\item Umgebungserfordernisse (notwendige Software- und
  Hardwareplattform sowie Testtreiber und -rümpfe)
\item besondere prozedurale Anforderungen (Einschränkungen wie
  Zeitvorgaben, Belastung oder Eingreifen durch den Operator)
\item Abhängigkeiten zwischen Testfällen
\end{enumerate}
}

\subsection{Funktionstest}\label{c10b}
\nurlangversion

\variante{}{\emph{Auch wenn dieser Teil in SWP-1 nicht gefordert ist,
    so müsst Ihr Euch dennoch über den Funktionstest Gedanken machen
    und ihn planen. In SWP-1 ist der Plan für den Funktionstest
    letztlich nichts anderes als das Drehbuch Eurer
    Abschlusspräsentation, die den Akzeptanztest darstellt.}}

{\em Die Funktionstests basieren auf den Anwendungsfällen. Diese
  müssen hier nun konkretisiert werden, um einen Testfall zu
  erhalten. Es müssen also konkrete Werte für Ein- und Ausgabe
  festgelegt werden. Auch das Verhalten im Fehlerfall sollte getestet
  werden.}

{\em Die Anwendungsfälle müssen nicht alle wiederholt werden. Sie
  werden schließlich in der Anforderungsspezifikation
  spezifiziert. Hier genügt eine tabellarische Auflistung aller zu
  testenden Anwendungsfälle, deren konkrete Ein- und Ausgaben sowie
  die Umsetzung (automatisiert oder manuell und falls manuell, durch
  welche Art von Benutzer?). Insbesondere müsst Ihr an dieser Stelle
  klären, welche Varianten von Anwendungsfällen getestet werden. Falls
  Varianten oder gar ganze Anwendungsfälle nicht getestet werden
  sollen, dann wird hier eine plausible Begründung erwartet.}


\subsection{Leistungstest}\label{c10c}
\nurlangversion

{\em Die Leistungstests prüfen die Leistungsanforderungen, wie z.B.\ Reaktionszeiten, 
das Verhalten unter extremen Bedingungen, bei großen Datenmengen etc\ldots}

\subsubsection{Härtetest}
\nurlangversion


\subsubsection{Volumentest}
\nurlangversion


\subsubsection{Sicherheitstest}
\nurlangversion

\subsubsection{Erholungstest}
\nurlangversion

{\em Hier wird geprüft, ob sich das System von Fehlerzuständen auch wieder erholt.
 Es werden also gezielt Fehler provoziert, um die Korrektheit der Systemreaktion herauszufinden.}


\section{Testzeitplan}
\label{sec:testzeitplan}
\nurlangversion

\end{document}
