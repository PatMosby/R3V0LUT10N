\documentclass[fontsize=12pt,paper=a4,twoside]{scrartcl}

\newcommand{\grad}{\ensuremath{^{\circ}} }
\renewcommand{\strut}{\vrule width 0pt height5mm depth2mm}

\usepackage{qtree}

\usepackage[utf8]{inputenc}
\usepackage[final]{pdfpages}
% obere Seitenränder gestalten können
\usepackage{fancyhdr}
\usepackage{moreverb}
% Graphiken als jpg, png etc. einbinden können
\usepackage{graphicx}
\usepackage{stmaryrd}
% Floats Objekte mit [H] festsetzen
\usepackage{float}
% setzt URL's schön mit \url{http://bla.laber.com/~mypage}
\usepackage{url}
% Externe PDF's einbinden können
\usepackage{pdflscape}
% Verweise innerhalb des Dokuments schick mit " ... auf Seite ... "
% automatisch versehen. Dazu \vref{labelname} benutzen
\usepackage[ngerman]{varioref}
\usepackage[ngerman]{babel}
\usepackage{ngerman}
% Bibliographie
\usepackage{bibgerm}
% Tabellen
\usepackage{tabularx}
\usepackage{supertabular}
\usepackage[colorlinks=true, pdfstartview=FitV, linkcolor=blue,
            citecolor=blue, urlcolor=blue, hyperfigures=true,
            pdftex=true]{hyperref}
\usepackage{bookmark}


\newboolean{langversion} %Deklaration
\setboolean{langversion}{true} %Zuweisung ist 'false' für Blockkurs
\newcommand{\highlight}[1]{\textcolor{blue}{\textbf{#1}}}
\newcommand{\nurlangversion}[0]{%
\ifthenelse{\boolean{langversion}}{\highlight{Muss in SWP-2 ausgefüllt werden}}{\highlight{Entfällt in SWP-1}}}

% erstes Argument: SWP-2, zweites SWP-1
\newcommand{\variante}[2]{\ifthenelse{\boolean{langversion}}{#1}{#2}}


% Damit Latex nicht zu lange Zeilen produziert:
\sloppy
%Uneinheitlicher unterer Seitenrand:
%\raggedbottom

% Kein Erstzeileneinzug beim Absatzanfang
% Sieht aber nur gut aus, wenn man zwischen Absätzen viel Platz einbaut
\setlength{\parindent}{0ex}

% Abstand zwischen zwei Absätzen
\setlength{\parskip}{1ex}

% Seitenränder für Korrekturen verändern
\addtolength{\evensidemargin}{-1cm}
\addtolength{\oddsidemargin}{1cm}

\bibliographystyle{gerapali}

% Lustige Header auf den Seiten
  \pagestyle{fancy}
  \setlength{\headheight}{70.55003pt}
  \fancyhead{}
  \fancyhead[LO,RE]{Software--Projekt 2\\  2013
  \\Testplan}
  \fancyhead[LE,RO]{Seite \thepage\\\slshape \leftmark\\\slshape \rightmark}

%
% Und jetzt geht das Dokument los....
%

\begin{document}

% Lustige Header nur auf dieser Seite
  \thispagestyle{fancy}
  \fancyhead[LO,RE]{ }
  \fancyhead[LE,RO]{Universität Bremen\\FB 3 -- Informatik\\
  Prof. Dr. Rainer Koschke \\TutorIn: Sabrina Wilske}
  \fancyfoot[C]{}

% Start Titelseite
  \vspace{3cm}

  \begin{minipage}[H]{\textwidth}
  \begin{center}
  \bf
  \Large
  Software--Projekt 2 2013\\
  \smallskip
  \small
  VAK 03-BA-901.02\\
  \vspace{3cm}
  \end{center}
  \end{minipage}
  \begin{minipage}[H]{\textwidth}
  \begin{center}
  \vspace{1cm}
  \bf
  \Large Testplan\\
  \vspace{3ex} \small IT\_R3V0LUT10N\\
  \vfill
  \end{center}
  \end{minipage}
  \vfill
  \begin{minipage}[H]{\textwidth}
  \begin{center}
  \sf
  \begin{tabular}{lrr}
 			Sebastian Bredehöft & sbrede@tzi.de & 2751589\\
 			Patrick Damrow & damsen@tzi.de & 2056170\\
 			Tobias Dellert & tode@tzi.de & 2936941\\
 			Tim Ellhoff & tellhoff@tzi.de & 2520913\\
 			Daniel Pupat & dpupat@tzi.de & 2703053\\
  \end{tabular}
  \\ ~
  \vspace{2cm}
  \\
  \it Abgabe: 22. Dezemebr 2013 --- Version 1.0\\ ~
  \end{center}
  \end{minipage}

% Ende Titelseite

% Start Leerseite

\newpage

  \thispagestyle{fancy}
  \fancyhead{}
  \fancyhead[LO,RE]{Software--Projekt \\  2013
  \\Testplan}
  \fancyhead[LE,RO]{Seite \thepage\\\slshape \leftmark\\~}
  \fancyfoot{}
  \renewcommand{\headrulewidth}{0.4pt}
  \tableofcontents

\newpage

  \fancyhead[LE,RO]{Seite \thepage\\\slshape \leftmark\\\slshape \rightmark}


%%%%%%%%%%%%%%%%%%%%%%%%%%%%%%%%%%%%%%%%%%%%%%%%%%%%%%%%%%%%%%%%%%%%%%%%
\section*{Version und Änderungsgeschichte}

{\em Die aktuelle Versionsnummer des Dokumentes sollte eindeutig und gut zu
identifizieren sein, hier und optimalerweise auf dem Titelblatt.}

\begin{tabular}{ccl}
Version & Datum & Änderungen \\
\hline
1.0 & 18.12.2013 & Dokumentvorlage als initiale Fassung kopiert \\
\end{tabular}


%%%%%%%%%%%%%%%%%%%%%%%%%%%%%%%%%%%%%%%%%%%%%%%%%%%%%%%%%%%%%%%%%%%%%%%%
\section{Einführung (Sebastian)}\label{c01}

\subsection{Zweck}
Der Testplan bietet einen Überblick über die geplanten Tests und dient u.a. als Anleitung für die Tester. Die Software soll dabei ausführlich auf Funktionalität getestet werden. 

Im Testplan wird festgelegt, wie man welche Komponenten testet. Dazu wird außerdem definiert, welchen Umfang die Tests haben sollen und wann ein Test erfolgreich ist und wann nicht.

Während der Implementierungsphase werden wir uns nach dem Testplan richten und ihn gegebenenfalls weiterführen und vervollständigen.


\subsection{Umfang}
Der Testplan entspricht der vereinfachten Form des \emph{IEEE Standard for Software Test Documentation 829-1998}. 

\subsection{Beziehungen zu anderen Dokumenten}
Dieser Testplan bezieht sich zum einen auf die Anforderungsspezifikation, da dort die Systemeigenschaften und Systemattribute spezifiziert wurden. Die Testfälle werden auf Grundlage der dortigen Anwendungsfälle entwickelt.

Außerdem gibt es Referenzen zur Architekturbeschreibung, da in dieser die Module und Komponenten definiert wurden, die in diesem Dokument getestet werden sollen.


\subsection{Aufbau der Testbezeichner}
\label{sec:aufb-der-testb}

Der Aufbau der Testbezeichner richtet sich nach folgendem Schema:
\begin{itemize}
\item Die ersten beiden Buchstaben geben die Art des Tests vor. Dabei unterscheiden wir zwischen vier verschiedenen Testarten:
\begin{itemize}
	\item Komponententests = KT
	\item Integrationstests = IT
	\item Funktionstests = FT
	\item Leistungstests = LT
\end{itemize}
\item Die Nummer steht für die jeweilige Testfallnummer

\item Optional: in alphabetischer Reihenfolge werden hier Variationen oder untergeordnete Testfälle definiert
\end{itemize}
Nach diesem Schema sieht ein Testbezeichner nun folgendermaßen aus:

\textbf{\emph{IT-3-A}}: Integrationstest, Nr. 3, Variante 1

%%
%% Dokumentation der Testergebnisse 
%%
\subsection{Dokumentation der Testergebnisse}

Zu jedem Testfall wird ein kurzes Testprotokoll angefertigt. Dieses beinhaltet den Ablauf des Testfalls und die möglichen Komplikationen, die bei der Durchführung entstehen können. Dann werden die Resultate des Testfalls bestimmt und eventuell gefundene Fehler beschrieben.

%%
%% Definitionen und Referenzen 
%%
\subsection{Definitionen und Akronyme}
\label{c00b}


\subsection{Referenzen}

\bibliographystyle{plain}
\bibliography{literatur}


% Systemüberblick
\section{Systemüberblick (Sebastian)}\label{c02}
Das System besteht aus der Server- und der Clientkomponente. Die konzeptionelle Sicht der Architekturbeschreibung (vgl. Abschnitt 3 der Architekturbeschreibung) dient als Grundlage für den Testplan, da dort die verschiedenen Komponenten  beschrieben werden.

Auf der Serverseite gibt es die Komponenten \texttt{Communication, BusinessLogic} und \texttt{Persistence} (vgl. Abbildung 3: Konzeptionelle Sicht Server; Architekturbeschreibung).

Die Clientseite besteht aus den Komponenten \texttt{Communication, Model} und \texttt{User Interface} (vgl. Abbildung 4: Konzeptionelle Sicht Client; Architekturbeschreibung).

Da starke Abhängigkeiten zwischen all diesen Komponenten bestehen, ist es wichtig, dass diese Komponenten fehlerfrei funktionieren. 


{\em Es sollten die Hofmeister-Konzepte verwendet
  werden.\ Benutzt am besten Diagramme aus der Architekturbeschreibung.}

\emph{Dieser Abschnitt ist nicht einfach nur Copy\&Paste der
  Architekturbeschreibung. Es gilt einerseits, Redundanzen soweit wie
  möglich zu vermeiden, und andererseits, dieses Dokument so
  selbsterklärend wie möglich zu machen.}

\subsection{Module der Anwendungsschicht und deren Funktionen}
\label{mod-controller}
\nurlangversion

{\em Hier solltet ihr dann die einzelnen Module aus Sicht des
Tests weiter verfeinern. Auch wäre noch eine Grafik angebracht, die
die Abhängigkeiten der Subsysteme/Module zeigt, sowie eine kurze
Beschreibung dazu.}

\subsubsection{Server}


\subsubsection{Client}


\clearpage

\section{Merkmale (Daniel)}

Zu testende Merkmale sind in erster Linie Funktionen, die in den Mindestanforderungen enthalten sind. Dabei ist zu beachten, dass die Funktionen des Lesers für die Website und der App getestet werden müssen, während sich die anderen Testmerkmale auf die Website beziehen. Beide sind im Folgenden aufgelistet:\\

1. Benutzerrechte\\
\begin{itemize}
\item[1.1]Administrator 
\item[1.2]Bibliothekar 
\item[1.3]Leser 
\item[1.4]Gast
\end{itemize}
\bigskip
2. Administrator\\
\begin{itemize}
\item[2.1]Bibliothekar hinzufügen
\item[2.2]Bibliothekar löschen
\item[2.3]Bibliothekar ändern
\end{itemize}
\bigskip
3. Bibliothekar\\
\begin{itemize}
\item[3.1]Medium hinzufügen
\item[3.2]Medium löschen
\item[3.3]Medium ändern
\item[3.4]Medium ausleihen
\item[3.5]Medium zurücknehmen
\item[3.6]Abgabedaten und Mahngebühren ändern
\item[3.7]Vormerkungen anzeigen
\item[3.8]Übersicht über verliehene Bücher(Versäumnisse, Mahngebühren)
\item[3.9]Statistiken anzeigen
\end{itemize}
\bigskip
4.Leser\\
\begin{itemize}
\item[4.1]Medium suchen
\item[4.2]Medium anzeigen
\item[4.3]Medium vormerken
\end{itemize}
\bigskip
Dazu kommt noch der Gast, der unangemeldet nur suchen kann und man sollte die Unterklassen von Medium noch einzeln testen, ob diese die geforderten Attribute und Funktionen enthalten.

\subsubsection{Funktionale Anforderungen} 

Besonders wichtig sind die Funktionen des Bibliothekars (3.1-3.5). Die sollten gut getestet werden, da es beim Ausleihvorgang nicht zu Problemen kommen soll, sodass irgendwo Bücher verschwinden oder Ausleiher oder Bücher verwechselt werden. Auch wichtig ist die Benutzerunterscheidung, damit Leser nicht irgendwas löschen oder hinzufügen. Außerdem sollte man die Suche und das Anzeigen der Bücher ausgiebig testen, da dies die Hauptfunktionen des Lesers sind.

\subsection{Nicht zu testende Merkmale}\label{c05}
Nicht zu testende Merkmale sind in erster Linie alle trivialen Funktionen. Zudem brauchen bereits implementierte Funktionen, wie \texttt{Leser verwalten, Backup} und \texttt{Restore, Buchaufkleber drucken, Leserausweise drucken, Import und Export von Buch- und Leserdaten} nicht mehr getestet werden, sofern man diese nicht verändert. Da wir \texttt{Buch verwalten} noch verändern, da wir mit mehreren Medientypen arbeiten, ist dieses wie in \ref{c04} beschrieben noch zu testen.

% Abnahme- und Testkriterien
\section{Abnahme- und Testendekriterien(Daniel \& Sebastian)}\label{c07}
Fehler werden in eine Kategorie eingeordnet und erhalten entsprechende Fehlerwerte. Aus diesen Fehlerwerten ergeben sich Prioritäten, die die Reihenfolge der Fehlerbehandlung angibt. Das Testen wird beendet, wenn der berechnete Fehlerwert aller Fehler pro 1000 Zeilen Code unter dem Wert 10 liegt und die Software nicht beeinträchtigt wird, d.h. es keinen Fehler der Fehlerklasse \texttt{Mittel} oder höher gibt.

\textbf{Testabdeckung}
Die Testabdeckung soll so hoch wie möglich sein. Für ein stabiles System spricht, das die Testabdeckung in systemkritischen Bereichen soweit vollständig ist. Jeder Fehler in diesem Bereich kann das System zum Absturz bringen und muss somit verhindert werden. In anderen Bereichen die das laufende System bei einem Fehler weniger beinträchtigen wird die Testabdeckung nicht so vollständig sein, wie in kritischen Bereichen.

\textbf{Fehlerbewertung:}\\
Die nachfolgende Tabelle spezifiziert die Auswirkung eines Fehlers, durch die man diese nach Priorität einordnen kann.\\

\begin{tabularx}{\textwidth}{|p{2cm}|p{11.53cm}|c|}
\hline
	\textbf{Fehlerkl.\footnote{=Fehlerklasse}} & \textbf{Beschreibung} & \textbf{Wert}\\
\hline
	Leicht & Unwesentliche Fehler, die den Programmablauf nicht beeinträchtigen, aber trotzdem behandelt werden sollten. & 1\\
\hline
	Mittel & Fehler in dieser Art haben Auswirkungen auf den Programmablauf. Dieser beeinträchtigt aber nicht die grundlegenden Funktionen. & 10\\
\hline
	Schwer & Fehler der Klasse ,,Schwer'' beeinträchtigen die Funktionsfähigkeit des Systems sehr stark und müssen sofort behandelt werden. & 20\\
\hline
	Fatal & Diese Fehler machen den Programmablauf unmöglich und können zum Absturz des Systems führen.  & 100\\
\hline
\end{tabularx}



% Vorgehensweise
\section{Vorgehensweise (Sebastian)}\label{c06}

\subsection{Komponenten- und Integrationstest}

Die beiden Komponenten Server und Client werden bezüglich der Integrationstests zunächst unabhängig voneinander getestet und erst bei wenn das sichere Laufen der einzelnen Komponente gewährleistet ist, wird das System als ganzes getestet.

\subsection*{Server}
Es wird zuerst die Persistenz mit \texttt{eu.it\_r3v.bibcommon} und \texttt{eu.it\_r3v.persistence} getestet:

{\qtreeshowframes
\Tree [.eu.it\_r3v.persistence eu.it\_r3v.bibcommon
]}\\
\\
\\
\\
Nun folgt das Zusammenspiel mit der BusinessLogic \texttt{eu.it\_r3v.businesslogic}:\\

{\qtreeshowframes
\Tree [.eu.it\_r3v.persistence [.eu.it\_r3v.businesslogic ] {eu.it\_r3v.bibcommon} ]}\\
\\
\\
\\
Darauf folgt nun die Kommunikation \texttt{eu.it\_r3v.communication} mit den vorigen Komponenten:

{\qtreeshowframes
\Tree [.eu.it\_r3v.persistence [.eu.it\_r3v.businesslogic {eu.it\_r3v.communication} ] {eu.it\_r3v.bibcommon} ]}\\
\\
\\
\\
Somit ist der Server als ganzes zu testen, da jede einzelnen Komponenten mit ihren jeweiligen Abhängigkeiten getestet wurden.
\subsection*{Client}
Hier sind nun zwei verschiedene Komponenten für die Darstellung auf dem jeweiligen Endgerät vorhanden:\\

Einmal \texttt{GUI} welche zusammen mit dem Model getestet wird und sich an die Browserdarstellung richtet:

{\qtreeshowframes
\Tree [.GUI [.Model ]]}\\
\\
\\
\\
Und einmal \texttt{Android App} mit dem Model, welche sich an mobile Geräte richtet:

{\qtreeshowframes
\Tree [.GUI [.Model ]]}\\
\\
\\
\\
Diese beiden Tests werden zusammengefasst zu \texttt{User Interface} und zusammen mit der nächsten Ebene, dem Model getestet:

{\qtreeshowframes
\Tree [.UserInterface [.Model ]]}\\
\\
\\
\\
Darauf folgt nun der Test mit der Komponente \texttt{Communication}:

{\qtreeshowframes
\Tree [.UserInterface [.Model [.Communication ]]]}\\
\\
\\
\\
Nun ist das Zusammenspiel der Komponenten des Clients gewährleistet.
\subsection*{Server + Client}
Um die kompletten Systemkomponenten zu testen werden jetzt der komplette Client und der komplette Server zusammen getestet:

{\qtreeshowframes
\Tree [.eu.it\_r3v.persistence [.eu.it\_r3v.businesslogic [.eu.it\_r3v.communication [.Model [.UserInterface GUI AndroidApp ] ] ] ] eu.it\_r3v.bibcommon ]

\subsection{Funktionstest}

Die Funktionstest sind durch jene Anwendungsfälle aus der Anforderungsspezifikation vorgegeben. Jede dieser Funktionen muss durch Tests gedeckt sein.


\section{Aufhebung und Wiederaufnahme (Daniel)}\label{c08}

Wir werden Tests unterbrechen, wenn ein gewisser Wert überschritten wird, welcher über die Tabelle \ref{testende}berechnet wird. In diesem Fall werden wir sofort wieder mit der Implementierung anfangen. Da wir mit der Bottom-up Strategie testen, werden wir bei Fehlern in der unteren Schicht einen niedrigeren Wert nehmen.\\
Bei Fehlern der Data setzen wir einen Wert von 10, bei Fehlern in der Logik einen Wert von 20 und bei Fehlern, welche die GUI betreffen, einen von 40 und bei den restlichen Faktoren einen von 100.\\
Sollten die Fehler behoben sein, testen wir noch einmal alle Komponenten, die mit den veränderten interagieren.

\section{Hardware- und Softwareanforderungen (Daniel)}\label{c09}


\subsection{Hardware}

Als Hardware stehen uns unsere Notebooks und Smartphones, sowie die Unirechner zur Verfügung. Dabei haben wir alle geforderten Betriebssysteme mindestens einmal auf unseren Notebooks installiert, sodass wir auf jeden Gerät testen können. Wir besitzen ebenfalls einen PC, der über Windows 2000 läuft, darüber testen wir auch noch, da dies den Rechnern des Kunden entspricht. Da nur Android Unterstützung gefordert ist, werden wir die App über unseren vorhandenen Smartphones, die Android haben, testen.

\subsection{Software}

Als Software benutzen wir in der Eclipse Umgebung JUnit-Tests. Diese werden in Form von BlackBox- und WhiteBox-Tests implementiert. Die App werden wir mithilfe eines Android Emulators testen.

% Testfälle
\section{Testfälle}\label{c10}

{\em Dies ist der wichtigste (und vermutlich umfangreichste) Teil des
  Testplans. Hier wird genau aufgelistet, {\bf was wie} und {\bf von
  wem} getestet wird. Das spätere Testen besteht dann einfach aus
  einer Durchführung dieser Tests.}

{\em Zu jedem Testfall muss es eine {\bf Testfallspezifikation}
  geben (außer Komponententests, siehe unten). Schwerpunkt sollen
  hier Integrations- und Leistungstests sein. Welche Arten von
  Integrationstests und Leistungstests seht Ihr vor und wie wollt Ihr
  diese genau ausgestalten?}

\subsection{Komponententest}\label{c10-0}

{\em Auf die genaue Spezifikation der Komponententests könnt Ihr
  verzichten -- diese werden durch die JUnit-Testfälle gegeben. Uns
  genügt hierzu eine Beschreibung, welche Komponenten/Klassen wie,
  wann und durch wen getestet werden sollen. Die zugehörigen
  JUnit-Tests werden separat abgegeben (Black-Box jetzt, White-Box
  später bei der Abgabe der Implementierung) und müssen im Testplan
  nicht als Code aufgeführt werden.}

\begin{table}[h]
\centering
\begin{tabular}{|l|p{3cm}|p{3cm}|l|}
\hline
Klasse & Implementierer & Tester & Testart \\
\hline
ReferenceManagement & Klaus  & Hans    & Blackbox \\
UserManagement      & Anna   & Klaus   & Blackbox \\
\dots               & \dots  & \dots   & \dots \\
ReferenceTable      & Heinz  & Bert    & Whitebox \\
\hline
\end{tabular}
\caption{Komponententests}
\label{tab:komponententests}
\end{table}

{\em Achtet darauf, dass der Tester ein anderer ist als der
  Implementierer. Jeder soll mindestens eine Klasse als Black-Box und
  eine Klasse als White-Box testen!}

{\em Hier könnten später noch die Kontrollflussdiagramme für die
  Whiteboxtests eingefügt werden.}

\clearpage
\subsection{Integrationstest}\label{c10a}

{\em Die Integrationstests werden hier genauer beschrieben: Welche
  Klassen sind beteiligt? Wie ist der Zustand des Systems vor Beginn?
  Welche Eingaben werden getätigt? Welche Ergebnisse/welches Verhalten
  wird erwartet?}

{\em Aufbau einer Testfallspezifikation:
\begin{enumerate}
\item eindeutiger Testfallbezeichner (entspricht Namenskonvention aus Abschnitt~\ref{sec:aufb-der-testb})
\item Testobjekte: welche Komponenten werden getestet?
\item Eingabespezifikationen (Eingaben des Testfalls)
\item Ausgabespezifikationen (erwartete Ausgaben)
\item Umgebungserfordernisse (notwendige Software- und
  Hardwareplattform sowie Testtreiber und -rümpfe)
\item besondere prozedurale Anforderungen (Einschränkungen wie
  Zeitvorgaben, Belastung oder Eingreifen durch den Operator)
\item Abhängigkeiten zwischen Testfällen
\end{enumerate}
}

\subsection{Funktionstest}\label{c10b}
\nurlangversion

\variante{}{\emph{Auch wenn dieser Teil in SWP-1 nicht gefordert ist,
    so müsst Ihr Euch dennoch über den Funktionstest Gedanken machen
    und ihn planen. In SWP-1 ist der Plan für den Funktionstest
    letztlich nichts anderes als das Drehbuch Eurer
    Abschlusspräsentation, die den Akzeptanztest darstellt.}}

{\em Die Funktionstests basieren auf den Anwendungsfällen. Diese
  müssen hier nun konkretisiert werden, um einen Testfall zu
  erhalten. Es müssen also konkrete Werte für Ein- und Ausgabe
  festgelegt werden. Auch das Verhalten im Fehlerfall sollte getestet
  werden.}

{\em Die Anwendungsfälle müssen nicht alle wiederholt werden. Sie
  werden schließlich in der Anforderungsspezifikation
  spezifiziert. Hier genügt eine tabellarische Auflistung aller zu
  testenden Anwendungsfälle, deren konkrete Ein- und Ausgaben sowie
  die Umsetzung (automatisiert oder manuell und falls manuell, durch
  welche Art von Benutzer?). Insbesondere müsst Ihr an dieser Stelle
  klären, welche Varianten von Anwendungsfällen getestet werden. Falls
  Varianten oder gar ganze Anwendungsfälle nicht getestet werden
  sollen, dann wird hier eine plausible Begründung erwartet.}


\subsection{Leistungstest}\label{c10c}
\nurlangversion

{\em Die Leistungstests prüfen die Leistungsanforderungen, wie z.B.\ Reaktionszeiten, 
das Verhalten unter extremen Bedingungen, bei großen Datenmengen etc\ldots}

\subsubsection{Härtetest}
\nurlangversion


\subsubsection{Volumentest}
\nurlangversion


\subsubsection{Sicherheitstest}
\nurlangversion

\subsubsection{Erholungstest}
\nurlangversion

{\em Hier wird geprüft, ob sich das System von Fehlerzuständen auch wieder erholt.
 Es werden also gezielt Fehler provoziert, um die Korrektheit der Systemreaktion herauszufinden.}


\section{Testzeitplan}
\label{sec:testzeitplan}
\nurlangversion

\end{document}
