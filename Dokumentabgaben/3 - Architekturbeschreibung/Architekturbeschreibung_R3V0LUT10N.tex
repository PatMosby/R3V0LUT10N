\documentclass[fontsize=12pt,paper=a4,twoside]{scrartcl}

% Dokumentenpräambel
% hier alle Packages die genutzt werden mittels \usepackage einbinden
\usepackage{booktabs}

\usepackage[utf8]{inputenc}

\usepackage[final]{pdfpages}

\usepackage[ngerman]{babel} 

% für Deutschunterstützung und neue Rechtschreibung
\usepackage[ngerman]{varioref} 

% Verweise innerhalb des Dokuments schick mit " ... auf Seite ... " automatisch versehen. Dazu \vref{labelname} benutzen
\usepackage{ngerman}

% obere Seitenränder gestalten können
\usepackage{fancyhdr}

% Graphiken als jpg, png etc. einbinden können
\usepackage{graphicx}

% Unterstützung für etliche Symbole
\usepackage{stmaryrd}

% Floats Objekte mit [H] festsetzen
\usepackage{float}

% setzt URL's schön mit \url{http://bla.laber.com/~mypage}
\usepackage{url}

% Externe PDF's einbinden können
\usepackage{pdflscape}

% Bibliographie
\usepackage{bibgerm}

% Tabellen
\usepackage{tabularx}

\usepackage{supertabular}

\usepackage[colorlinks=true, pdfstartview=FitV, linkcolor=blue, citecolor=blue, urlcolor=blue, hyperfigures=true, pdftex=true]{hyperref}

\usepackage{bookmark}

%
% hier Formatierungsbefehle einfügen
%
% Damit Latex nicht zu lange Zeilen produziert
\sloppy

% Uneinheitlicher unterer Seitenrand
\raggedbottom

% Pfad zu den Grafiken fürs Dokument. Ordner muss im gleichen Verzeichniss liegen
\graphicspath{{graphics/},{graphics/ScreensWebsite/}}

% Gafikendungen die genutzt werden
\DeclareGraphicsExtensions{.png,.jpg}

% Seitenränder für Korrekturen setzen
\addtolength{\evensidemargin}{-1cm}\addtolength{\oddsidemargin}{1cm}

% Kein Erstzeileneinzug beim Absatzanfang. Sieht nur gut aus, wenn man zwischen Absätzen viel Platz einbaut.
\setlength{\parindent}{0ex} 

% Abstand zwischen zwei Absätzen
\setlength{\parskip}{1ex}

% hier werden neue Befehle deklariert
% Header für alle Seiten
\pagestyle{fancy} \setlength{ 
\headheight}{70.55003pt} \fancyhead{} \fancyhead[LO,RE]{Software-Projekt 2\\
2013/14 \\Architekturbeschreibung} \fancyhead[LE,RO]{Seite \thepage\\\slshape \leftmark\\
\slshape\rightmark}

%
% Ab hier beginnt das Dokument
%
\begin{document}

% Lustige Header nur auf dieser Seite
\thispagestyle{fancy} \fancyhead[LO,RE]{ } \fancyhead[LE,RO]{Universität Bremen\\FB 3 -- Informatik\\
Prof. Dr. Rainer Koschke \\TutorIn: Sabrina Wilske} \fancyfoot[C]{}

% Start Titelseite
\vspace{3cm} 
\begin{minipage}
	[H]{ 
	\textwidth} 
	\begin{center}
		\bf \Large Software-Projekt 2 2013/14\\
		\smallskip \small VAK 03-BA-901.02\\
		\vspace{3cm} 
	\end{center}
\end{minipage}
\begin{minipage}
	[H]{ 
	\textwidth} 
	\begin{center}
		\vspace{1cm} \bf \Large Architekturbeschreibung\\
		\vspace{3ex} \small IT\_R3V0LUT10N\\
		\vfill 
	\end{center}
\end{minipage}
\vfill 
\begin{minipage}
	[H]{ 
	\textwidth} 
	\begin{center}
		\sf 
		\begin{tabular}
			{lrr} Sebastian Bredehöft & sbrede@tzi.de & 2751589\\
			Patrick Damrow & damsen@tzi.de & 2056170\\
			Tobias Dellert & tode@tzi.de & 2936941\\
			Tim Ellhoff & tellhoff@tzi.de & 2520913\\
			Daniel Pupat & dpupat@tzi.de & 2703053\\
		\end{tabular}
		\\
		~ \vspace{2cm} \\
		\it Abgabe: 22. Dezember 2013 -- Version 1.1 vom 08.12.2013 \\
		~ 
	\end{center}
\end{minipage}

% Ende Titelseite
% Eine Leerseite
\newpage

\thispagestyle{fancy} \fancyhead{} \fancyhead[LO,RE]{Software-Projekt 2 \\
2013/14 \\Architekturbeschreibung} \fancyhead[LE,RO]{Seite \thepage\\\slshape \leftmark\\~} \fancyfoot{} 
\renewcommand{\headrulewidth}{0.4pt} 
\tableofcontents
\newpage
% Abbildungsverzeichnis 
\listoffigures

% Tabellenverzeichnis
\listoftables

\clearpage

\fancyhead[LE,RO]{Seite \thepage\\
\slshape \leftmark\\
\slshape\rightmark}
%%%%%%%%%%%%%%%%%%%%%%%%%%%%%%%%%%%%%%%%%%%%%%%%%%%%%%%%%%%%%%%%%%%%%%%%
\section*{Version und Änderungsgeschichte}

{\em Die aktuelle Versionsnummer des Dokumentes sollte eindeutig und gut zu
identifizieren sein, hier und optimalerweise auf dem Titelblatt.}

\begin{tabular}{ccl}
Version & Datum & Änderungen \\
\hline
1.0 & 25.11.2013 & Dokumentvorlage als initiale Fassung kopiert \\
1.1 & 08.12.2013 & Einflussfaktoren \\
\end{tabular}


%%%%%%%%%%%%%%%%%%%%%%%%%%%%%%%%%%%%%%%%%%%%%%%%%%%%%%%%%%%%%%%%%%%%%%%%
\section{Einführung}

\subsection{Zweck}

  {\em Was ist der Zweck dieser Architekturbeschreibung? Wer sind
  die LeserInnen?}

\subsection{Status}


  
\subsection{Definitionen, Akronyme und Abkürzungen}


\subsection{Referenzen}

\subsection{Übersicht über das Dokument}



\section{Globale Analyse}
\label{sec:globale_analyse}

\subsection{Einflussfaktoren}
\label{sec:einflussfaktoren}

Die Einflussfaktoren werden im Folgenden unterteilt in:

\begin{itemize}
\item{Organisatorische Faktoren}
\item{Technische Faktoren}
\item{Produktfaktoren}
\end{itemize}

\subsubsection{Organisatorische Faktoren}
\label{sec:orgfaktoren}

\begin{table}[H]
\centering
\caption{Organisatorische Faktoren}
\begin{tabular}{|l|l|} \hline
\textbf{O1} & \textbf{Time-To-Market} \\ \hline
\textbf{O2} & \textbf{Auslieferung von Produktfunktionen} \\ \hline
\textbf{O3} & \textbf{Budget} \\ \hline
\textbf{O4} & \textbf{Kenntnisse in Java und Android} \\ \hline
\textbf{O5} & \textbf{Kenntnisse in J-Unit} \\ \hline
\textbf{O6} & \textbf{Anzahl der Entwickler}\\ \hline
\end{tabular}
\end{table}

\begin{table}[H]
\begin{tabular}{|p{3cm}|p{12cm}|}\hline
\textbf{O1} & \textbf{Time-To-Market}\\ \hline \hline
Faktor & Auslieferungsdatum 23.02.2014\\ \hline
Flexibilität und Veränderlichkeit & Die Deadline kann nicht verändert werden\\ \hline
Auswirkungen & Die Software muss zum Abgabedatum lauffähig sein\\ \hline
\end{tabular}
\end{table}

\begin{table}[H]
\begin{tabular}{|p{3cm}|p{12cm}|}\hline
\textbf{O2} & \textbf{Auslieferung von Produktfunktionen}\\ \hline \hline
Faktor & Alle Mindestanforderungen\\ \hline
Flexibilität und Veränderlichkeit & Es müssen alle Mindestanforderungen erfüllt sein, sind jedoch vom Kunden oder beim verlassen eines Gruppenmitglieds veränderbar\\ \hline
Auswirkungen & Architektur muss alle Mindestanforderungen abdecken, es muss darauf geachtet werden, dass diese sich im Verlauf noch ändern\\ \hline
\end{tabular}
\end{table}

\begin{table}[H]
\begin{tabular}{|p{3cm}|p{12cm}|}\hline
\textbf{O3} & \textbf{Budget} \\ \hline \hline
Faktor & Kein finanzielles Budget\\ \hline
Flexibilität und Veränderlichkeit & Es werden keine finanziellen Unterstützungen für das Produkt geben \\ \hline
Auswirkungen & Es können keine Kostenpflichtigen Dienste in Anspruch genommen werden\\ \hline
\end{tabular}
\end{table}

\begin{table}[H]
\begin{tabular}{|p{3cm}|p{12cm}|}\hline
\textbf{O4} & \textbf{Kenntnisse in Java und Android} \\ \hline \hline
Faktor & Kenntnisse der Entwickler in Java und Android\\ \hline
Flexibilität und Veränderlichkeit & Kenntnisse sind nicht flexibel, es muss in Java programmiert werden und über Smartphone laufen. Die Kenntnisse können sich im Laufe ändern, durch neue Erfahrungen und neu erworbenen Kenntnissen\\ \hline
Auswirkungen & Bei wenig Kenntnissen muss mehr Zeit eingeplant werden um sich diese anzueignen\\ \hline
\end{tabular}
\end{table}

\begin{table}[H]
\begin{tabular}{|p{3cm}|p{12cm}|}\hline
\textbf{O5} & \textbf{Kenntnisse in J-Unit} \\ \hline \hline
Faktor & Kenntnisse in J-Unit Tests\\ \hline
Flexibilität und Veränderlichkeit & Da Tests mit J-Unit gefordert werden, sind diese nicht verhandelbar oder Flexibel\\ \hline
Auswirkungen & Bei unzureichenden Tests kann es später beim Programm zu Problemen kommen, da Fehler spät oder gar nicht erkannt werden\\ \hline
\end{tabular}
\end{table}

\begin{table}[H]
\begin{tabular}{|p{3cm}|p{12cm}|}\hline
\textbf{O6} & \textbf{Anzahl der Entwickler}\\ \hline \hline
Faktor & Die Anzahl der Entwickler\\ \hline
Flexibilität und Veränderlichkeit & Es können keine neuen Gruppenmitglieder dazukommen, es können aber jederzeit Gruppenmitglieder wegfallen \\ \hline
Auswirkungen & Wenn Gruppenmitglieder wegfallen, müssen die restlichen mehr Arbeiten und mehr Zeit einplanen. Auch müssen Projektplan und Architektur neu angepasst werden.\\ \hline
\end{tabular}
\end{table}

\subsubsection{Technische Faktoren}
\label{sec:techfaktoren}

\begin{table}[H]
\centering
\caption{Technische Faktoren}
\begin{tabular}{|l|l|} \hline
\textbf{T1} & \textbf{Software funktioniert unter Windows, Linux und MacOS} \\ \hline
\textbf{T2} & \textbf{Software funktioniert als App(Andriod 2.3 oder höher)}\\ \hline
\textbf{T3} & \textbf{SQL-Datenbank} \\ \hline
\textbf{T4} & \textbf{Mehrere parallele Nutzer} \\ \hline
\textbf{T5} & \textbf{Client-Server System} \\ \hline
\textbf{T6} & \textbf{Benutzerschnittstelle} \\ \hline
\textbf{T7} & \textbf{Implementierungssprache Java} \\ \hline
\textbf{T8} &  \textbf{Beschränkungsfreiheit für Fremdbibliotheken}\\ \hline
\end{tabular}
\end{table}

\begin{table}[H]
\begin{tabular}{|p{3cm}|p{12cm}|}\hline
\textbf{T1} & \textbf{Software funktioniert unter Windows, Linux und MacOS} \\ \hline \hline
Faktor & Die Software muss auf den Betriebssystemen Windows, Linux und MacOS laufen\\ \hline
Flexibilität und Veränderlichkeit & nicht Flexibel, da dies zu den Mindestanforderungen gehört. Veränderungen können jederzeit vom Kunden vorgenommen werden.  \\ \hline
Auswirkungen & Die Entwickler müssen sich mit allen Betriebsprogrammen befassen und sichergehen, dass es auf allen funktioniert\\ \hline
\end{tabular}
\end{table}


\begin{table}[H]
\begin{tabular}{|p{3cm}|p{12cm}|}\hline
\textbf{T2} & \textbf{Software funktioniert als App (Andriod 2.3 oder höher)} \\ \hline \hline
Faktor & Die Software muss als Android App auf einem Smartphone laufen\\ \hline
Flexibilität und Veränderlichkeit & nicht Flexibel, da dies zu den Mindestanforderungen gehört. Veränderungen können jederzeit vom Kunden vorgenommen werden.  \\ \hline
Auswirkungen & Die Software muss wie gefordert als App auf einem Android-Smartphone laufen\\ \hline
\end{tabular}
\end{table}


\begin{table}[H]
\begin{tabular}{|p{3cm}|p{12cm}|}\hline
\textbf{T3} & \textbf{SQL-Datenbank} \\ \hline \hline
Faktor & Software läuft über eine relationale Datenbank\\ \hline
Flexibilität und Veränderlichkeit & Flexibel jedoch muss eine Datenbank mit SQL oder SQL-ähnlichen abfragen verwendet werden  \\ \hline
Auswirkungen & Es muss eine relationale Datenbank für die serverseitige Persistenz benutzt werden. Es muss eine Datenbank mit SQL oder SQL-ähnlichen abfragen verwendet werden\\ \hline
\end{tabular}
\end{table}

\begin{table}[H]
\begin{tabular}{|p{3cm}|p{12cm}|}\hline
\textbf{T4} & \textbf{Mehrere parallele Nutzer} \\ \hline \hline
Faktor & Es greifen mehrere Nutzer zur gleichen Zeit auf die Software zu\\ \hline
Flexibilität und Veränderlichkeit & Es ist uns überlassen, wie viele Nutzer zur gleichen Zeit auf das System zugreifen dürfen  \\ \hline
Auswirkungen & Die Software muss darauf ausgelegt sein, mehrere Nutzer zur gleichen Zeit zu verwalten\\ \hline
\end{tabular}
\end{table}

\begin{table}[H]
\begin{tabular}{|p{3cm}|p{12cm}|}\hline
\textbf{T5} & \textbf{Client-Server System} \\ \hline \hline
Faktor & Die Software arbeitet über ein Client-Server System\\ \hline
Flexibilität und Veränderlichkeit & Da wir übers Internet auf den Server zugreifen müssen, ist es notwendig ein Server-Client System zu verwenden \\ \hline
Auswirkungen & Die Implementierung wird in Server und Client aufgeteilt(siehe \ref{sec:konzeptionell}) Übers Internet werden die Daten zwischen Server und Client ausgetauscht\\ \hline
\end{tabular}
\end{table}

\begin{table}[H]
\begin{tabular}{|p{3cm}|p{12cm}|}\hline
\textbf{T6} & \textbf{Benutzerschnittstelle} \\ \hline \hline
Faktor & Es sollte eine übersichtliche und ansprechende GUI geben\\ \hline
Flexibilität und Veränderlichkeit & Die Gestaltung der GUI ist uns überlassen \\ \hline
Auswirkungen & Für eine Benutzerfreundliche Gestaltung sind gute Kenntnisse in XHTML notwendig\\ \hline
\end{tabular}
\end{table}

\begin{table}[H]
\begin{tabular}{|p{3cm}|p{12cm}|}\hline
\textbf{T7} & \textbf{Implementierungssprache Java} \\ \hline \hline
Faktor & Die Software muss in Java 5 oder höher geschrieben werden\\ \hline
Flexibilität und Veränderlichkeit & Nicht Flexibel, da dies zu den Mindestanforderungen gehört\\ \hline
Auswirkungen & Die Software muss in Java geschrieben werden, daher müssen alle Entwickler diese Sprache beherrschen \\ \hline
\end{tabular}
\end{table}

\begin{table}[H]
\begin{tabular}{|p{3cm}|p{12cm}|}\hline
\textbf{T8} & \textbf{Beschränkungsfreiheit für Fremdbibliotheken}\\ \hline \hline
Faktor & Fremdbibliotheken dürfen für den Einsatz in Forschung\\
& und Lehre keine Beschränkungen aufweisen \\ \hline
Flexibilität und Veränderlichkeit & Nicht Flexibel, da dies zu den Mindestanforderungen gehört\\ \hline
Auswirkungen & Es darf keine Software oder Bibliothek verwendet werden, die Kostenpflichtig ist \\ \hline
\end{tabular}
\end{table}

\subsubsection{Produkt Faktoren}
\label{sec:produktfaktoren}

\begin{table}[H]
\centering
\caption{Produkt Faktoren}
\begin{tabular}{|l|l|} \hline
\textbf{P1} & \textbf{Mindestanforderung} \\ \hline
\textbf{P2} &  \textbf{Performanz}\\ \hline
\textbf{P3} &  \textbf{Benutzerrechte} \\ \hline
\textbf{P4} &  \textbf{Fehlererkennung} \\ \hline
\end{tabular}
\end{table}

\begin{table}[H]
\begin{tabular}{|p{3cm}|p{12cm}|}\hline
\textbf{P1} & \textbf{Mindestanforderung} \\ \hline \hline
Faktor & Das Produkt muss alle Mindestanforderungen enthalten\\ \hline
Flexibilität und Veränderlichkeit & Alle Anforderungen müssen zum Bestehen erfüllt werden. Die Anforderungen können vom Kunden oder Dozenten verändert werden oder die Anforderungen werden bei einem Austritt eines Mitglieds verringert.\\ \hline
Auswirkungen & Es müssen alle Mindestanforderungen implementiert werden \\ \hline
\end{tabular}
\end{table}


\begin{table}[H]
\begin{tabular}{|p{3cm}|p{12cm}|}\hline
\textbf{P2} &  \textbf{Performanz}\\ \hline \hline
Faktor & Möglichst schnelle Ausführungszeiten\\ \hline
Flexibilität und Veränderlichkeit & Flexibel, da nichts davon in den Mindestanforderungen steht\\ \hline
Auswirkungen & Es sollte bei der Implementierung auf einen schnellen Datenaustausch zwischen Server und Client geachtet werden \\ \hline
\end{tabular}
\end{table}

\begin{table}[H]
\begin{tabular}{|p{3cm}|p{12cm}|}\hline
\textbf{P3} &  \textbf{Benutzerrechte} \\ \hline \hline
Faktor & Es gibt verschiedene Benutzer mit unterschiedlichen Rechten\\ \hline
Flexibilität und Veränderlichkeit & Nicht Flexibel, da dies vom Kunden gefordert wird\\ \hline
Auswirkungen & Es müssen unterschiedliche Benutzer implementiert werden, die unterschiedliche Rechte haben und diese auch nicht überschreiten dürfen \\ \hline
\end{tabular}
\end{table}

\begin{table}[H]
\begin{tabular}{|p{3cm}|p{12cm}|}\hline
\textbf{P4} &  \textbf{Fehlererkennung} \\ \hline \hline
Faktor & Fehler sollten von der Software erkannt werden und entsprechend behandelt werden\\ \hline
Flexibilität und Veränderlichkeit & Flexibel, da dies nicht ausdrücklich vom Kunden gefordert wird\\ \hline
Auswirkungen & Fehler müssen erkannt werden und durch Exception muss es dann entsprechend Korrigiert werden. Die Software sollte weiter laufen  \\ \hline
\end{tabular}
\end{table}


\subsection{Probleme und Strategien}
\label{sec:strategien}

Folgenden Probleme haben wir identifiziert:\\

\begin{table}[H]
\centering
\begin{tabular}{|l|l|}\hline
Nummer & Faktoren\\ \hline
1 & Zeitprobleme\\ \hline
2 & Mangelnde Kenntnisse in Java\\ \hline
3 & Mangelnde Kenntnisse in Android\\ \hline
4 & Mangelnde Kenntnisse von Datenbanksystemen\\ \hline
5 & Unzureichende Softwaretests\\ \hline
6 & Ausfall eines Gruppenmitglieds\\ \hline
7 & Mehrere parallele Nutzer\\ \hline
8 & Performanz\\ \hline
9 & unterschiedliche Benutzerrechte\\ \hline
\end{tabular}
\end{table}

Diese versuchen wir mit folgenden Strategien zu überbrücken:\\

\begin{table}[H]
\begin{tabular}{|p{\textwidth}|}\hline
1 Zeitprobleme\\ \hline
Es gibt einen festgesetzten Abgabetermin, der eingehalten werden muss\\ \hline
\textbf{Einflussfaktoren}\\
\begin{itemize}
\item O1 Time-To-Market
\item O2 Auslieferung von Produktfunktionen
\item O4 Kenntnisse in Java  und Android
\item O5 Kenntnisse in J-Unit
\item O6 Anzahl der Entwickler
\item P1 Mindestanforderungen
\end{itemize}\\ \hline
\textbf{Lösung}\\
\begin{itemize}
\item Strategie 1: Modularisierung für paralleles Arbeiten \leavevmode\newline
Durch Modularisierung können mehrere Entwickler zur gleichen Zeit am Projekt arbeiten und die Module unabhängig voneinander implementieren. Diese werden dann später zusammengesetzt 
\item Strategie 2: Bibliotheken Benutzen \leavevmode\newline
Es werden bereits vorhandene Java Bibliotheken verwendet, dies spart Zeit, da man dann nicht alles neu schreiben muss.
\end{itemize}
Es werden beide Strategien verwendet.\\ \hline
\end{tabular}
\end{table}

\begin{table}[H]
\begin{tabular}{|p{\textwidth}|}\hline
2 Mangelnde Kenntnisse in Java\\ \hline
Es werden Vorkenntnisse in Java vorausgesetzt, ohne diese könnte es zu großen Problemen kommen, da ohne ausreichende Kenntnisse das Programm nicht realisiert werden kann\\ \hline
\textbf{Einflussfaktoren}\\
\begin{itemize}
\item O1 Time-To-Market
\item O2 Auslieferung von Produktfunktionen
\item O4 Kenntnisse in Java und Android
\item O6 Anzahl der Entwickler
\item T1 Software funktioniert unter Windows, Linux und MacOS
\item T5 Client-Server System
\item T7 Implementierungssprache Java
\item P1 Mindestanforderungen
\end{itemize}\\ \hline
\textbf{Lösung}\\
\begin{itemize}
\item Strategie 1: Modularisierung \leavevmode\newline
Der Code wird in verschiedene Module aufgeteilt, wenn ein Gruppenmitglied nicht genügend Kenntnisse besitzt, kann dieses Modul von einem anderen Mitglied neu erstellt werden und der inkompetente Entwickler kann keinen Schaden auf andere Module auswirken.
\item Strategie 2: Aufteilen in Server und Client \leavevmode\newline
Die Implementierung wird unter den Entwicklern so aufgeteilt, dass ein Teil den Client und ein Teil den Server macht, so müssen sich die Gruppenmitglieder nicht Kenntnisse in beiden Teilen aneignen.
\end{itemize}
Es werden beide Strategien verwendet.\\ \hline
\end{tabular}
\end{table}


\begin{table}[H]
\begin{tabular}{|p{\textwidth}|}\hline
3 Mangelnde Kenntnisse in Android\\ \hline
Es werden Kenntnisse in Android vorausgesetzt, da eine App entwickelt werden muss. \\ \hline
\textbf{Einflussfaktoren}\\
\begin{itemize}
\item O1 Time-To-Market
\item O2 Auslieferung von Produktfunktionen
\item O4 Kenntnisse in Java und Android
\item O6 Anzahl der Entwickler
\item T2 Software funktioniert als App(Andriod 2.3 oder höher)
\item T5 Client-Server System
\item T6 Benutzerschnittstelle
\item T7 Implementierungssprache Java
\item P1 Mindestanforderungen
\end{itemize}\\ \hline
\textbf{Lösung}\\
\begin{itemize}
\item Strategie 1: Modularisierung \leavevmode\newline
Der Code wird in verschiedene Module aufgeteilt, wenn ein Gruppenmitglied nicht genügend Kenntnisse besitzt, kann dieses Modul von einem anderen Mitglied neu erstellt werden und der inkompetente Entwickler kann keinen Schaden auf andere Module auswirken.
\item Strategie 2: Bearbeitung von Gruppenmitgliedern mit Android-Erfahrung \leavevmode\newline
Wir werden die Implementierung einen Gruppenmitglied überlassen, welches bereits Erfahrung mit Android hat. So müssen sich die anderen nicht in Android einarbeiten und können sich bei Fragen an diesen wenden.
\end{itemize}
Es werden beide Strategien verfolgt, sollte das Gruppenmitglied mit Android zeitlich oder fachlich nicht klarkommen, wird ein weiteres Gruppenmitglied sich mit Android beschäftigen. \\ \hline
\end{tabular}
\end{table}

\begin{table}[H]
\begin{tabular}{|p{\textwidth}|}\hline
4 Mangelnde Kenntnisse in Datenbanksystemen\\ \hline
Es werden Kenntnisse in Datenbanksystemen vorausgesetzt, da wir für die Bibliothek eine Datenbank verwenden. Dabei werden SQL- oder SQL-ähnliche Abfragen verwendet und entsprechende Kenntnisse verlangt \\ \hline
\textbf{Einflussfaktoren}\\
\begin{itemize}
\item O1 Time-To-Market
\item O2 Auslieferung von Produktfunktionen
\item O7 Kenntnisse in Datenbanksystemen
\item O6 Anzahl der Entwickler
\item T5 Client-Server System
\item T7 Implementierungssprache Java
\end{itemize}\\ \hline
\textbf{Lösung}\\
\begin{itemize}
\item Strategie 1: Bearbeitung von Gruppenmitgliedern mit Erfahrung in Datenbanksystemen \leavevmode\newline
Wir werden die Implementierung Gruppenmitgliedern überlassen, welches bereits Erfahrung mit Datenbanksystemen hat. So müssen sich die anderen nicht in Datenbanksystemen einarbeiten und können sich bei Fragen an diesen wenden.
\end{itemize}\\ \hline
\end{tabular}
\end{table}

\begin{table}[H]
\begin{tabular}{|p{\textwidth}|}\hline
5 Unzureichende Softwaretests\\ \hline
Es werden genügend Tests benötigt, welche Module und Komponenten testen, ob diese funktionieren\\ \hline
\textbf{Einflussfaktoren}\\
\begin{itemize}
\item O1 Time-To-Market
\item O5 Kenntnisse in J-Unit
\item T7 Implementierungssprache Java
\item P1 Mindestanforderungen
\item P2 Performanz
\item P3 Benutzerrechte
\item P4 Fehlererkennung
\end{itemize}\\ \hline
\textbf{Lösung}\\
\begin{itemize}
\item Strategie 1: Modularisierung \leavevmode\newline
Es werden Tests für die jeweilig implementierten Module geschrieben, ob diese ihren Zweck erfüllen und danach werden Module zusammen getestet. 
\end{itemize} \\ \hline
\end{tabular}
\end{table}

\begin{table}[H]
\begin{tabular}{|p{\textwidth}|}\hline
6 Ausfall eines Gruppenmitglieds\\ \hline
Es kann jederzeit ein Gruppenmitglied aus der Gruppe austreten oder durch Krankheit etc. für eine gewisse Zeit ausfallen.\\ \hline
\textbf{Einflussfaktoren}\\
\begin{itemize}
\item O1 Time-To-Market
\item O2 Auslieferung von Produktfunktionen
\item O6 Anzahl der Entwickler
\item P1 Mindestanforderungen
\end{itemize}\\ \hline
\textbf{Lösung}\\
\begin{itemize}
\item Strategie 1: Modularisierung \leavevmode\newline
Der Code wird in verschiedene Module aufgeteilt, welche von einem Entwickler bearbeitet werden. Wenn nun ein Entwickler ausfällt, kann ein Modul von einem anderen Entwickler übernommen werden.
\end{itemize} \\ \hline
\end{tabular}
\end{table}

\begin{table}[H]
\begin{tabular}{|p{\textwidth}|}\hline
7 Mehrere parallele Nutzer\\ \hline
Es greifen mehrere Nutzer zur gleichen Zeit auf das System zu, auf welche der Server antworten muss. Dabei soll der Server die Daten nicht an alle Clients senden\\ \hline
\textbf{Einflussfaktoren}\\
\begin{itemize}
\item O1 Time-To-Market
\item O3 Budget
\item T3 SQL-Datenbank
\item T4 Mehrere parallele Nutzer
\item P1 Mindestanforderungen
\item P2 Performanz
\item P3 Benutzerrechte
\end{itemize}\\ \hline
\textbf{Lösung}\\
\begin{itemize}
\item Strategie 1: Thread \leavevmode\newline
Die Clients bekommen jeweils einen Thread, somit können sie zeitgleich auf den Server zugreifen und bekommen nur ihre Daten zurück.
\end{itemize} \\ \hline
\end{tabular}
\end{table}

\begin{table}[H]
\begin{tabular}{|p{\textwidth}|}\hline
8 Performanz\\ \hline
Die Software sollte kurze Ausführungszeiten haben. Dabei ist zu beachten, dass die Software/App auch auf Geräten mit geringer Leistung schnell und problemlos läuft\\ \hline
\textbf{Einflussfaktoren}\\
\begin{itemize}
\item O1 Time-To-Market
\item T1 Software funktioniert unter Windows, Linux und MacOS
\item T2 Software funktioniert als App(Android 2.3 oder höher)
\item T3 SQL-Datenbank
\item T4 Mehrere parallele Nutzer
\item P1 Mindestanforderungen
\item P2 Performanz
\item P3 Benutzerrechte
\end{itemize}\\ \hline
\textbf{Lösung}\\
\begin{itemize}
\item Strategie 1: Code effizient schreiben \leavevmode\newline
Den Code effizient schreiben, damit die Software kurze Ausführungszeiten hat.
\end{itemize}\\ \hline
\end{tabular}
\end{table}

\begin{table}[H]
\begin{tabular}{|p{\textwidth}|}\hline
9 unterschiedliche Benutzerrechte\\ \hline
Die Software hat unterschiedliche Benutzer, welche unterschiedliche Rechte besitzen und diese müssen unterschieden werden\\ \hline
\textbf{Einflussfaktoren}\\
\begin{itemize}
\item O1 Time-To-Market
\item T3 SQL-Datenbank
\item T4 Mehrere parallele Nutzer
\item P1 Mindestanforderungen
\item P3 Benutzerrechte
\end{itemize}\\ \hline
\textbf{Lösung}\\
\begin{itemize}
\item Strategie 1: Identifikation durch Group Id \leavevmode\newline
In der Datenbank wird eine Group Id eingefügt, welche dann die verschiedenen Nutzer speichert. Über diese werden dann die verschiedenen Rechte geregelt.
\end{itemize} \\ \hline
\end{tabular}
\end{table}

\newpage

\section{Konzeptionelle Sicht}
\label{sec:konzeptionell}

Wir haben mithilfe von UML-Diagrammen die konzeptionelle Sicht realisiert. Im Folgenden werden die einzelnen Diagramme aufgezeigt und beschrieben und in nachfolgenden Sichten zusätzlich verfeinert und konkretsiert.

\subsection{Überblick}
\label{Ueberblick}

\begin{figure} [H] 
\caption{Konzeptionelle Sicht (Klein)} \centering
	\includegraphics[scale=2]{Diagramme/KonzeptionelleSichtKlein.png} 
	\label{pic:konzeptionellesichtklein} 
\end{figure}

Unsere Architektur besteht aus zwei grundlegenden Komponenten, der Serverkomponente und der Clientkomponente. Diese Komponenten beinhalten wiederum weitere Komponenten . Auf der einen Seite haben wir unsere Serverkomponente, die alle benötigten Daten der Medien, der Nutzer und der Ausleihvorgänge der Bibliothek speichert.

Auf der anderen Seite, der Clientkomponente, muss zwischen zwei Teilen unterschieden werden. Einmal der Gui-Client, welcher sich in erster Linie an die Bibliothekare richtet und der mobile Android-Client, der sich ausschließlich an die Leser richtet. \\
Der Gui-Client stellt für die Bibliothekare alle benötigten Funktionen bereit um eine Bibliothek zu verwalten. Der Android-Client ermöglicht dem Leser sich Mediendetails, Ausleihstatus, seine eigene Ausleihhistorie sowie persönliche Daten anzeigen zu lassen. Desweiteren kann der Leser sich mittels der App Bücher zur Ausleihe vormerken und Informationen über die Bibliothek abrufen.

\begin{figure} [H] 
\caption{Konzeptionelle Sicht}  \centering
	\includegraphics[width=1\textwidth]{Diagramme/KonzeptionelleSicht.png} 
	\label{pic:konzeptionellesicht} 
\end{figure}

{\centering Als Architekturstil verwenden wir das Model-View-Controller-Pattern.\\}

\subsection{Serverkomponente}
\label{sec:server}

\begin{figure} [H] 
\caption{Konzeptionelle Sicht Server}  \centering
	\includegraphics[scale=1.85]{Diagramme/KonzeptionelleSichtServer.png} 
	\label{pic:konzeptionellesichtserver} 
\end{figure}

Die Serverkomponente besteht aus insgesamt drei Teilkomponenten. welche sich wie folgt aufgliedern:

\begin{itemize}
\item{Communication}

Die Komponente Communication nimmt Anfragen des Clients entgegen und leitet sie an die Komponente BusinessLogic weiter wo die Anfragen verarbeitet werden und sendet die Ergebnisse zurück an den Client.

\item{BusinessLogic}

Die Komponente BusinessLogic dient zum verarbeiten der Anfragen und leitet diese verarbeiteten Anfragen dann an die Komponente Persistence weiter.

\item{Persistence}

Die Komponente Persistence ist die Schnittstelle zur Datenbank. Über das Interface DBControl werden die verarbeiteten Anfragen von der Komponente BusinessLogic empfangen und in Datenbankabfragen umgewandelt, welche dann von der Datenbank entgegen genommen werden.

\end{itemize}

\subsection{Clientkomponente}
\label{sec:client}

\begin{figure} [H] 
\caption{Konzeptionelle Sicht Client}  \centering
	\includegraphics[scale=1.7]{Diagramme/KonzeptionelleSichtClient.png} 
	\label{pic:konzeptionellesichtclient} 
\end{figure}

Die Clientkomponente besteht so wie die Serverkomponente aus drei Teilkomponenten, welche sich wie folgt aufgliedern:

\begin{itemize}
\item{Communication}

Die Komponente Communication sendet Anfragen des Clients an den Server, welche dort verarbeitet werden und nimmt die Ergebnisse entgegen um diese an die Komponente Model zu übergeben wo die Ergebnisse der Anfrage weiter verarbeitet werden.

\item{Model}

Die Komponente Model nimmt Ergebnisse von der Komponente Communication entgegen und schickt diese an die Komponente User Interface. 

\item{User Interface}

Die Kompnente User Interface muss in zwei unterschiedliche Komponenten zerlegt werden:
\begin{itemize}
\item{GUI}

Die GUI richtet sich in erster Linie an Bibliothekare.
\item{Android-App}
\end{itemize}

\end{itemize}

\section{Modulsicht}
\label{sec:modulsicht}

\subsection{Pakete}
\label{sec:pakete}

Wir haben ein Hauptpaket eu.it\_r3v in dem sich weitere Unterpakete befinden. Diese dienen der Bündelung gemeinsamer Quellcodedateien.

\begin{figure} [H] 
\caption{Pakete Übersicht} \centering
	\includegraphics[width=0.6\textwidth]{Diagramme/PackageUebersicht.png} 
	\label{pic:PackageUebersicht} 
\end{figure}
\label{sec:PackageUebersicht}

\subsubsection{Paket bibclient}
\label{sec:bibclient}

\begin{figure} [H] 
\caption{Paket bibclient} \centering
	\includegraphics[width=1\textwidth]{Diagramme/Packagebibclient.png} 
	\label{pic:PackageClient} 
\end{figure}


\subsubsection{Pakete bibcommon}
\label{sec:bibcommon}

\begin{figure} [H] 
\caption{Paket bibcommon} \centering
	\includegraphics[width=1\textwidth]{Diagramme/Packagebibcommon.png} 
	\label{pic:PackageCommon} 
\end{figure}


\subsubsection{Paket bibjsf}
\label{sec:bibclient}

\begin{figure} [H] 
\caption{Paketübersicht bibjsf} \centering
	\includegraphics[width=0.45\textwidth]{Diagramme/Packagebibjsfuebersicht.png} 
	\label{pic:PackagebibjsfUebersicht} 
\end{figure}

\begin{figure} [H] 
\caption{Paket bibjsf} \centering
	\includegraphics[width=1\textwidth]{Diagramme/Packagebibjsf.png} 
	\label{pic:PackagebibjsfUebersicht} 
\end{figure}





{\it
Diese Sicht beschreibt den statischen Aufbau des Systems mit Hilfe von
Modulen, Subsystemen, Schichten und Schnittstellen. 
Diese Sicht ist hierarchisch, d.h. Module werden in Teilmodule
zerlegt. Die Zerlegung endet bei Modulen, die ein klar umrissenes
Arbeitspaket für eine Person darstellen und in einer Kalenderwoche
implementiert werden können. Die Modulbeschreibung der Blätter dieser
Hierarchie muss genau genug und ausreichend sein, um das Modul 
implementieren zu können.

Die Modulsicht wird durch {UML}-Paket- und Klassendiagramme visualisiert.

Die Module werden durch ihre Schnittstellen beschrieben. 
Die Schnittstelle eines Moduls $M$ ist die Menge aller Annahmen, die
andere Module über $M$ machen dürfen, bzw.\ jene Annahmen, die $M$
über seine verwendeten Module macht (bzw. seine Umgebung, wozu auch
Speicher, Laufzeit etc.\ gehören).
Konkrete Implementierungen dieser Schnittstellen sind das Geheimnis des Moduls
und können vom Programmierer festgelegt werden. Sie sollen hier
dementsprechend nicht beschrieben werden. 

Die Diagramme der Modulsicht sollten die zur Schnittstelle gehörenden Methoden
enthalten. Die Beschreibung der einzelnen Methoden (im Sinne der Schnittstellenbeschreibung)
geschieht allerdings per Javadoc im zugehörigen Quelltext. Das bedeutet, dass Ihr
für alle Eure Module Klassen, Interfaces und Pakete erstellt und sie mit den Methoden der
Schnittstellen verseht. Natürlich noch ohne Methodenrümpfe bzw.\ mit minimalen Rümpfen.
Dieses Vorgehen vereinfacht den Schnittstellenentwurf und stellt Konsistenz sicher.

Jeder Schnittstelle liegt ein
Protokoll zugrunde. Das Protokoll beschreibt die Vor- und
Nachbedingungen der Schnittstellenelemente. Dazu gehören die erlaubten
Reihenfolgen, in denen Methoden der Schnittstelle aufgerufen werden
dürfen, sowie Annahmen über Eingabeparameter und Zusicherungen über
Ausgabeparameter. Das Protokoll von Modulen wird in der Modulsicht beschrieben.
Dort, wo es sinnvoll ist, sollte es mit Hilfe von Zustands- oder
Sequenzdiagrammen spezifiziert werden. Diese sind dann einzusetzen, wenn der
Text allein kein ausreichendes Verständnis vermittelt (insbesondere
bei komplexen oder nicht offensichtlichen Zusammenhängen).

Der Bezug zur konzeptionellen Sicht muss klar ersichtlich sein. Im
Zweifel sollte er explizit erklärt werden. Auch für diese Sicht muss
die Entstehung anhand der Strategien erläutert werden.
}

\section{Datensicht}
\label{sec:datensicht}

{\it Hier wird das der Anwendung zugrundeliegende Datenmodell
  beschrieben. Hierzu werden neben einem erläuternden Text auch ein
  oder mehrere {UML}-Klassendiagramme verwendet. Das hier beschriebene
  Datenmodell wird u.a. jenes der Anforderungsspezifikation enthalten,
  allerdings mit implementierungsspezifischen Änderungen und
  Erweiterungen. Siehe die gesonderten Hinweise.}
  
  

\section{Ausführungssicht}

Das folgende Diagramm \vref{ausfuehrungssicht} zeigt das Laufzeitverhalten der Software.
Auf der einen Seite haben wir den mobilen Zugang auf welchem die Android App als Prozess läuft. Die App selber verwendet die Module \emph{bibclient} und \emph{bibcommon}. Von diesem mobilem Zugang kann eine TCP/IP-Verbindung zu dem Bibliotheksserver aufgebaut werden. Dabei gibt es mehrere Verbindungen zu immer nur einem Server, daher die Multiplizitäten \emph{*} und \emph{1}.

Auf der anderen Seite nimmt nun der Bibliotheksserver die TCP/IP-Verbindungen an. Er ist gleichzeitig Server und Datenbankserver. Die Datenbank läuft auf dem Subsystem Glassfish Server. Der Server verwendet die Module \emph{businesslogic, exception, isbnsearch, persistence, presentation, properties, renderer, services} und \emph{util}.

Das komplette System läuft mit zwei Prozessen: einmal mit der Android App und der andere Prozess ist \emph{bibjsf} der das Subsystem mit der Datenbank enthält. Prinzipiell gibt es unendlich viele mobile Zugänge bzw. App-Prozesse die auf einen Bibliotheksserver zugreifen.

\begin{figure} [H] 
\caption{Ausführungssicht} 
	\includegraphics[width=1\textwidth]{Diagramme/ausfuehrungssicht.png} 
	\label{ausfuehrungssicht} 
\end{figure}
\label{sec:ausfuehrung}

\section{Zusammenhänge zwischen Anwendungsfällen und Architektur}
\label{sec:anwendungsfaelle}

{\it In diesem Abschnitt sollen Sequenzdiagramme mit Beschreibung(!)
  für {zwei bis drei von Euch ausgewählte
    Anwendungsfälle}{einen von Euch ausgewählten Anwendungsfall}
  erstellt werden. Ein Sequenzdiagramm beschreibt den
  Nachrichtenverkehr zwischen allen Modulen, die an der Realisierung
  des Anwendungsfalles beteiligt sind.  {Wählt die
    Anwendungsfälle so, dass nach Möglichkeit alle Module Eures
    entworfenen Systems in mindestens einem Sequenzdiagramm
    vorkommen. Falls Euch das nicht gelingt, versucht möglichst viele
    und die wichtigsten Module abzudecken.}{Dazu könnt ihr Euch einen
    Anwendungsfall heraussuchen, der möglichst viele Module der
    Architektur abdeckt. In SWP-2 werden wir mehrere Anwendungsfälle
    betrachten und eine umfangreichere Abdeckung der Architektur
    anstreben.} }

\section{Evolution}


\label{sec:evolution}

In diesem Abschnitt geht es um mögliche Änderungen, Anpassungen bzw. Erweiterungen, die vorgenommen werden müssten, wenn sich Anforderungen des Systems ändern. \\
Dabei ist wichtig, dass sich solche Änderungen möglichst modular realisieren lassen, ohne die bestehende Architektur komplett zu verändern, was sehr aufwändig und somit nicht wünschenswert wäre. \\
Im Folgenden werden einige wichtige mögliche neue Anforderungen bzw. Erweiterungen aufgelistet und deren jeweiligen zu implementierenden Änderungen an der Architektur beschrieben. 

\subsection*{Erweiterungsmöglichkeiten aus der Anforderungsspezifikation}

Wir haben bereits in der Anforderungsspezifikation im Punkt ''Ausblick'' zwei mögliche Änderungen genannt, die wir im hier detaillierter beschreiben möchten. 

\subsubsection*{1. Medientypenzuwachs}

Wie schon in der Anforderungsspezifikation im Punkt ''2.7 Ausblick'' angedeutet, wäre eine zu erwartende bzw. recht wahrscheinliche Erweiterung der Bibliothekssoftware, neben den vorhandenen Medien wie Büchern, Zeitschriften, CDs usw. weitere Medien wie z.B. Blu-Rays verwenden zu wollen und diese entsprechend im System aufzunehmen.\\
Da in unserem System nicht alle Medien allein für sich stehen, sondern ihre gemeinsamen Eigenschaften ihrer Klassen in einer Superklasse namens \texttt{Medium} zusammengefasst werden, ist es recht einfach, einen solchen Medientypenzuwachs ohne großen Aufwand zu realisieren. Man muss lediglich eine weitere Subklasse der Klasse \texttt{Medium}, z.B. mit Namen \texttt{BluRay} ins Modul \texttt{bibcommon} einbauen.

\subsubsection*{2. Erweiterung des Systems für iOS-Geräte}

Wir werden für unser System eine Android-App entwickeln, die für den mobilen Zugang mit Smartphones einen Zugang zu dem Bibliothekssystem bietet. Da Android sehr verbreitet ist, werden wir mit dieser Lösung sicherlich viele Nutzer erreichen können, jedoch natürlich längst nicht alle. \\
Insofern wäre zu überlegen, ob man nicht auch eine zusätzliche App für iOS-Geräte entwickeln könnte. Um dies zu realisieren, bedarf es jedoch nicht, wie oben angestrebt, modularen Änderungen, sondern der Entwicklung eines grundlegend anderen Systems. Insofern kann an dieser Stelle auch nicht beschrieben werden, welche Änderungen vorzunehmen sind, weil es sich hierbei ja eben nicht um aufbauende Erweiterungen des bestehenden Systems, sondern im Prinzip um eine Neuentwicklung einer zweiten App handeln  würde.

\subsection*{Zusätzliche denkbare Erweiterungen}

Über die genannten Punkte aus der Anforderungsspezifikation hinaus könnten sich weitere Änderungen ergeben, die im Folgenden beschrieben werden.

\subsubsection*{1. Erweiterung des GUI-Layouts}

Es wäre möglicherweise durchaus wünschenswert, wenn man als Benutzer nicht nur ein GUI-Design verwenden könnte, sondern mehrere. Dafür müsste eine Funktionalität hinzugefügt werden, um zwischen verschiedenen Benutzeroberflächen wählen zu können. \\
Dazu müssten entsprechende Referenzierungen von neuen GUI-Style-Änderungen mit Bilddateien stattfinden sowie für Textanpassungen die XML-Dateien im Paket \texttt{biblient/res/layout} verändert bzw. erweitert werden. 

\subsubsection*{2. Mehrsprachigkeit}

Gemäß den Mindestanforderungen wird das System so implementiert sein, dass die Möglichkeit besteht, mehrere Sprachen für das Bibliothekssystem zu unterstützen. Insofern wäre es denkbar, dass neben Deutsch eine weitere Sprache eingebaut werden könnte. Englisch würde sich natürlich anbieten.
\end{document}


%%% Local Variables: 
%%% mode: latex
%%% mode: reftex
%%% mode: flyspell
%%% ispell-local-dictionary: "de_DE"
%%% TeX-master: t
%%% End: 
