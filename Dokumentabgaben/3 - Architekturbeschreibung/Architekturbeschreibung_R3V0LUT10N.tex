\documentclass[fontsize=12pt,paper=a4,twoside]{scrartcl}

\newcommand{\grad}{\ensuremath{^{\circ}} }
\renewcommand{\strut}{\vrule width 0pt height5mm depth2mm}

\usepackage[utf8]{inputenc}
\usepackage[final]{pdfpages}
% obere Seitenränder gestalten können
\usepackage{fancyhdr}
\usepackage{moreverb}
% Graphiken als jpg, png etc. einbinden können
\usepackage{graphicx}
\usepackage{stmaryrd}
% Floats Objekte mit [H] festsetzen
\usepackage{float}
% setzt URL's schön mit \url{http://bla.laber.com/~mypage}
\usepackage{url}
% Externe PDF's einbinden können
\usepackage{pdflscape}
% Verweise innerhalb des Dokuments schick mit " ... auf Seite ... "
% automatisch versehen. Dazu \vref{labelname} benutzen
\usepackage[ngerman]{varioref}
\usepackage[ngerman]{babel}
\usepackage{ngerman}
% Bibliographie
\usepackage{bibgerm}
% Tabellen
\usepackage{tabularx}
\usepackage{supertabular}
\usepackage[colorlinks=true, pdfstartview=FitV, linkcolor=blue,
            citecolor=blue, urlcolor=blue, hyperfigures=true,
            pdftex=true]{hyperref}
\usepackage{bookmark}

\hyphenation{Arbeits-paket}

\newboolean{langversion} %Deklaration
\setboolean{langversion}{true} %Zuweisung ist 'false' für Blockkurs
\newcommand{\highlight}[1]{\textcolor{blue}{\textbf{#1}}}
\newcommand{\nurlangversion}[0]{%
\ifthenelse{\boolean{langversion}}{\highlight{Muss in SWP-2 ausgefüllt werden}}{\highlight{Entfällt in SWP-1}}}

% erstes Argument: SWP-2, zweites SWP-1
\newcommand{\variante}[2]{\ifthenelse{\boolean{langversion}}{#1}{#2}}

% Damit Latex nicht zu lange Zeilen produziert:
\sloppy
%Uneinheitlicher unterer Seitenrand:
%\raggedbottom

% Kein Erstzeileneinzug beim Absatzanfang
% Sieht aber nur gut aus, wenn man zwischen Absätzen viel Platz einbaut
\setlength{\parindent}{0ex}

% Abstand zwischen zwei Absätzen
\setlength{\parskip}{1ex}

% Seitenränder für Korrekturen verändern
\addtolength{\evensidemargin}{-1cm}
\addtolength{\oddsidemargin}{1cm}

\bibliographystyle{gerapali}

% Lustige Header auf den Seiten
  \pagestyle{fancy}
  \setlength{\headheight}{70.55003pt}
  \fancyhead{}
  \fancyhead[LO,RE]{Software--Projekt 1\\ SoSe 2013
  \\Anforderungsspezifikation}
  \fancyhead[LE,RO]{Seite \thepage\\\slshape \leftmark\\\slshape \rightmark}

%
% Und jetzt geht das Dokument los....
%

\begin{document}

% Lustige Header nur auf dieser Seite
  \thispagestyle{fancy}
  \fancyhead[LO,RE]{ }
  \fancyhead[LE,RO]{Universität Bremen\\FB 3 -- Informatik\\
  Prof. Dr. Rainer Koschke \\TutorIn: Sabrina Wilske}
  \fancyfoot[C]{}

% Start Titelseite
  \vspace{3cm}

  \begin{minipage}[H]{\textwidth}
  \begin{center}
  \bf
  \Large
  Software--Projekt 1 2013\\
  \smallskip
  \small
  VAK 03-BA-901.02\\
  \vspace{3cm}
  \end{center}
  \end{minipage}
  \begin{minipage}[H]{\textwidth}
  \begin{center}
  \vspace{1cm}
  \bf
  \Large Architekturbeschreibung\\
  \vspace{3ex} \small IT\_R3V0LUT10N\\
  \vfill
  \end{center}
  \end{minipage}
  \vfill
  \begin{minipage}[H]{\textwidth}
  \begin{center}
  \sf
  \begin{tabular}{lrr}
 			Sebastian Bredehöft & sbrede@tzi.de & 2751589\\
 			Patrick Damrow & damsen@tzi.de & 2056170\\
 			Tobias Dellert & tode@tzi.de & 2936941\\
 			Tim Ellhoff & tellhoff@tzi.de & 2520913\\
 			Daniel Pupat & dpupat@tzi.de & 2703053\\
  \end{tabular}
  \\ ~
  \vspace{2cm}
  \\
  \it Abgabe: 22. Dezember 2013 --- Version 1.1\\ ~
  \end{center}
  \end{minipage}

% Ende Titelseite

% Start Leerseite

\newpage

  \thispagestyle{fancy}
  \fancyhead{}
  \fancyhead[LO,RE]{Software--Projekt \\  2013
  \\Architekturbeschreibung}
  \fancyhead[LE,RO]{Seite \thepage\\\slshape \leftmark\\~}
  \fancyfoot{}
  \renewcommand{\headrulewidth}{0.4pt}
  \tableofcontents

\newpage

  \fancyhead[LE,RO]{Seite \thepage\\\slshape \leftmark\\\slshape \rightmark}


%%%%%%%%%%%%%%%%%%%%%%%%%%%%%%%%%%%%%%%%%%%%%%%%%%%%%%%%%%%%%%%%%%%%%%%%
\section*{Version und Änderungsgeschichte}

{\em Die aktuelle Versionsnummer des Dokumentes sollte eindeutig und gut zu
identifizieren sein, hier und optimalerweise auf dem Titelblatt.}

\begin{tabular}{ccl}
Version & Datum & Änderungen \\
\hline
1.0 & 25.11.2013 & Dokumentvorlage als initiale Fassung kopiert \\
1.1 & 08.12.2013 & Einflussfaktoren \\
\end{tabular}


%%%%%%%%%%%%%%%%%%%%%%%%%%%%%%%%%%%%%%%%%%%%%%%%%%%%%%%%%%%%%%%%%%%%%%%%
\section{Einführung}

\subsection{Zweck}
\nurlangversion

  {\em Was ist der Zweck dieser Architekturbeschreibung? Wer sind
  die LeserInnen?}

\subsection{Status}
\nurlangversion


  
\subsection{Definitionen, Akronyme und Abkürzungen}


\subsection{Referenzen}

\subsection{Übersicht über das Dokument}
\nurlangversion


\section{Globale Analyse}
\label{sec:globale_analyse}

\subsection{Einflussfaktoren}
\label{sec:einflussfaktoren}

Die Einflussfaktoren werden im Folgenden unterteilt in:

\begin{itemize}
\item{Organisatorische Faktoren}
\item{Technische Faktoren}
\item{Produktfaktoren}
\end{itemize}

\subsubsection{Organisatorische Faktoren}
\label{sec:orgfaktoren}

\begin{table}[H]
\centering
\caption{Organisatorische Faktoren}
\begin{tabular}{|l|l|} \hline
O1 & Time-To-Market \\ \hline
O2 & Auslieferung von Produktfunktionen \\ \hline
O3 & Budget \\ \hline
O4 & Kenntnisse in Java und Android \\ \hline
O5 & Kenntnisse in J-Unit \\ \hline
O6 & Anzahl der Entwickler\\ \hline
\end{tabular}
\end{table}

\begin{table}[H]
\begin{tabular}{|p{3cm}|p{12cm}|}\hline
O1 & Time-To-Market\\ \hline
Faktor & Auslieferungsdatum 23.02.2014\\ \hline
Flexibilität und Veränderlichkeit & Die Deadline kann nicht verändert werden\\ \hline
Auswirkungen & Die Software muss zum Abgabedatum lauffähig sein\\ \hline
\end{tabular}
\end{table}

\begin{table}[H]
\begin{tabular}{|p{3cm}|p{12cm}|}\hline
O2 & Auslieferung von Produktfunktionen\\ \hline
Faktor & Alle Mindestanforderungen\\ \hline
Flexibilität und Veränderlichkeit & Es müssen alle Mindestanforderungen erfüllt sein, sind jedoch vom Kunden oder beim verlassen eines Gruppenmitglieds veränderbar\\ \hline
Auswirkungen & Architektur muss alle Mindestanforderungen abdecken, es muss darauf geachtet werden, dass diese sich im Verlauf noch ändern\\ \hline
\end{tabular}
\end{table}

\begin{table}[H]
\begin{tabular}{|p{3cm}|p{12cm}|}\hline
O3 & Budget \\ \hline
Faktor & Kein finanzielles Budget\\ \hline
Flexibilität und Veränderlichkeit & Es werden keine finanziellen Unterstützungen für das Produkt geben \\ \hline
Auswirkungen & Es können keine Kostenpflichtigen Dienste in Anspruch genommen werden\\ \hline
\end{tabular}
\end{table}

\begin{table}[H]
\begin{tabular}{|p{3cm}|p{12cm}|}\hline
O4 & Kenntnisse in Java und Android \\ \hline
Faktor & Kenntnisse der Entwickler in Java und Android\\ \hline
Flexibilität und Veränderlichkeit & Kenntnisse sind nicht flexibel, es muss in Java programmiert werden und über Smartphone laufen. Die Kenntnisse können sich im Laufe ändern, durch neue Erfahrungen und neu erworbenen Kenntnissen\\ \hline
Auswirkungen & Bei wenig Kenntnissen muss mehr Zeit eingeplant werden um sich diese anzueignen\\ \hline
\end{tabular}
\end{table}

\begin{table}[H]
\begin{tabular}{|p{3cm}|p{12cm}|}\hline
O5 & Kenntnisse in J-Unit \\ \hline
Faktor & Kenntnisse in J-Unit Tests\\ \hline
Flexibilität und Veränderlichkeit & Da Tests mit J-Unit gefordert werden, sind diese nicht verhandelbar oder Flexibel\\ \hline
Auswirkungen & Bei unzureichenden Tests kann es später beim Programm zu Problemen kommen, da Fehler spät oder gar nicht erkannt werden\\ \hline
\end{tabular}
\end{table}

\begin{table}[H]
\begin{tabular}{|p{3cm}|p{12cm}|}\hline
O6 & Anzahl der Entwickler\\ \hline
Faktor & Die Anzahl der Entwickler\\ \hline
Flexibilität und Veränderlichkeit & Es können keine neuen Gruppenmitglieder dazukommen, es können aber jederzeit Gruppenmitglieder wegfallen \\ \hline
Auswirkungen & Wenn Gruppenmitglieder wegfallen, müssen die restlichen mehr Arbeiten und mehr Zeit einplanen. Auch müssen Projektplan und Architektur neu angepasst werden.\\ \hline
\end{tabular}
\end{table}

\subsubsection{Technische Faktoren}
\label{sec:techfaktoren}

\begin{table}[H]
\centering
\caption{Technische Faktoren}
\begin{tabular}{|l|l|} \hline
T1 & Software funktioniert unter Windows, Linux und MacOS \\ \hline
T2 & Software funktioniert als App(Andriod 2.3 oder höher) \\ \hline
T3 & SQL-Datenbank \\ \hline
T4 & Mehrere parallele Nutzer \\ \hline
T5 & Client-Server System \\ \hline
T6 & Benutzerschnittstelle \\ \hline
T7 & Implementierungssprache Java \\ \hline
T8 &  Beschränkungsfreiheit für Fremdbibliotheken\\ \hline
\end{tabular}
\end{table}

\begin{table}[H]
\begin{tabular}{|p{3cm}|p{12cm}|}\hline
T1 & Software funktioniert unter Windows, Linux und MacOS \\ \hline
Faktor & Die Software muss auf den Betriebssystemen Windows, Linux und MacOS laufen\\ \hline
Flexibilität und Veränderlichkeit & nicht Flexibel, da dies zu den Mindestanforderungen gehört. Veränderungen können jederzeit vom Kunden vorgenommen werden.  \\ \hline
Auswirkungen & Die Entwickler müssen sich mit allen Betriebsprogrammen befassen und sichergehen, dass es auf allen funktioniert\\ \hline
\end{tabular}
\end{table}


\begin{table}[H]
\begin{tabular}{|p{3cm}|p{12cm}|}\hline
T2 & Software funktioniert als App (Andriod 2.3 oder höher) \\ \hline
Faktor & Die Software muss als Android App auf einem Smartphone laufen\\ \hline
Flexibilität und Veränderlichkeit & nicht Flexibel, da dies zu den Mindestanforderungen gehört. Veränderungen können jederzeit vom Kunden vorgenommen werden.  \\ \hline
Auswirkungen & Die Software muss wie gefordert als App auf einem Android-Smartphone laufen\\ \hline
\end{tabular}
\end{table}


\begin{table}[H]
\begin{tabular}{|p{3cm}|p{12cm}|}\hline
T3 & SQL-Datenbank \\ \hline
Faktor & Software läuft über eine relationale Datenbank\\ \hline
Flexibilität und Veränderlichkeit & Flexibel jedoch muss eine Datenbank mit SQL oder SQL-ähnlichen abfragen verwendet werden  \\ \hline
Auswirkungen & Es muss eine relationale Datenbank für die serverseitige Persistenz benutzt werden. Es muss eine Datenbank mit SQL oder SQL-ähnlichen abfragen verwendet werden\\ \hline
\end{tabular}
\end{table}

\begin{table}[H]
\begin{tabular}{|p{3cm}|p{12cm}|}\hline
T4 & Mehrere parallele Nutzer \\ \hline
Faktor & Es greifen mehrere Nutzer zur gleichen Zeit auf die Software zu\\ \hline
Flexibilität und Veränderlichkeit & Es ist uns überlassen, wie viele Nutzer zur gleichen Zeit auf das System zugreifen dürfen  \\ \hline
Auswirkungen & Die Software muss darauf ausgelegt sein, mehrere Nutzer zur gleichen Zeit zu verwalten\\ \hline
\end{tabular}
\end{table}

\begin{table}[H]
\begin{tabular}{|p{3cm}|p{12cm}|}\hline
T5 & Client-Server System \\ \hline
Faktor & Die Software arbeitet über ein Client-Server System\\ \hline
Flexibilität und Veränderlichkeit & Da wir übers Internet auf den Server zugreifen müssen, ist es notwendig ein Server-Client System zu verwenden \\ \hline
Auswirkungen & Die Implementierung wird in Server und Client aufgeteilt(siehe \ref{sec:konzeptionell}) Übers Internet werden die Daten zwischen Server und Client ausgetauscht\\ \hline
\end{tabular}
\end{table}

\begin{table}[H]
\begin{tabular}{|p{3cm}|p{12cm}|}\hline
T6 & Benutzerschnittstelle \\ \hline
Faktor & \\ \hline
Flexibilität und Veränderlichkeit &  \\ \hline
Auswirkungen & \\ \hline
\end{tabular}
\end{table}

\begin{table}[H]
\begin{tabular}{|p{3cm}|p{12cm}|}\hline
T7 & Implementierungssprache Java \\ \hline
Faktor & Die Software muss in Java 5 oder höher geschrieben werden\\ \hline
Flexibilität und Veränderlichkeit & Nicht Flexibel, da dies zu den Mindestanforderungen gehört\\ \hline
Auswirkungen & Die Software muss in Java geschrieben werden, daher müssen alle Entwickler diese Sprache beherrschen \\ \hline
\end{tabular}
\end{table}

\begin{table}[H]
\begin{tabular}{|p{3cm}|p{12cm}|}\hline
T8 & Beschränkungsfreiheit für Fremdbibliotheken\\ \hline
Faktor & Fremdbibliotheken dürfen für den Einsatz in Forschung\\
& und Lehre keine Beschränkungen aufweisen \\ \hline
Flexibilität und Veränderlichkeit & Nicht Flexibel, da dies zu den Mindestanforderungen gehört\\ \hline
Auswirkungen & Es darf keine Software oder Bibliothek verwendet werden, die Kostenpflichtig ist \\ \hline
\end{tabular}
\end{table}

\subsubsection{Produkt Faktoren}
\label{sec:produktfaktoren}

\begin{table}[H]
\centering
\caption{Produkt Faktoren}
\begin{tabular}{|l|l|} \hline
P1 & Mindestanforderung \\ \hline
P2 &  Performanz\\ \hline
P3 &  Benutzerrechte \\ \hline
P4 &  Fehlererkennung \\ \hline
\end{tabular}
\end{table}

\begin{table}[H]
\begin{tabular}{|p{3cm}|p{12cm}|}\hline
P1 & Mindestanforderung \\ \hline
Faktor & Das Produkt muss alle Mindestanforderungen enthalten\\ \hline
Flexibilität und Veränderlichkeit & Alle Anforderungen müssen zum Bestehen erfüllt werden. Die Anforderungen können vom Kunden oder Dozenten verändert werden oder die Anforderungen werden bei einem Austritt eines Mitglieds verringert.\\ \hline
Auswirkungen & Es müssen alle Mindestanforderungen implementiert werden \\ \hline
\end{tabular}
\end{table}


\begin{table}[H]
\begin{tabular}{|p{3cm}|p{12cm}|}\hline
P2 &  Performanz\\ \hline
Faktor & Möglichst schnelle Ausführungszeiten\\ \hline
Flexibilität und Veränderlichkeit & Flexibel, da nichts davon in den Mindestanforderungen steht\\ \hline
Auswirkungen & Es sollte bei der Implementierung auf einen schnellen Datenaustausch zwischen Server und Client geachtet werden \\ \hline
\end{tabular}
\end{table}

\begin{table}[H]
\begin{tabular}{|p{3cm}|p{12cm}|}\hline
P3 &  Benutzerrechte \\ \hline
Faktor & Es gibt verschiedene Benutzer mit unterschiedlichen Rechten\\ \hline
Flexibilität und Veränderlichkeit & Nicht Flexibel, da dies vom Kunden gefordert wird\\ \hline
Auswirkungen & Es müssen unterschiedliche Benutzer implementiert werden, die unterschiedliche Rechte haben und diese auch nicht überschreiten dürfen \\ \hline
\end{tabular}
\end{table}

\begin{table}[H]
\begin{tabular}{|p{3cm}|p{12cm}|}\hline
P4 &  Fehlererkennung \\ \hline
Faktor & Fehler sollten von der Software erkannt werden und entsprechend behandelt werden\\ \hline
Flexibilität und Veränderlichkeit & Flexibel, da dies nicht ausdrücklich vom Kunden gefordert wird\\ \hline
Auswirkungen & Fehler müssen erkannt werden und durch Exception muss es dann entsprechend Korrigiert werden. Die Software sollte weiter laufen  \\ \hline
\end{tabular}
\end{table}


\subsection{Probleme und Strategien}
\label{sec:strategien}

{\it Aus einer Menge von Faktoren ergeben sich Probleme, die nun in
  Form von Problemkarten beschrieben werden. Diese resultieren
  z.B. aus
  \begin{itemize}
  \item Grenzen oder Einschränkungen durch Faktoren
  \item der Notwendigkeit, die Auswirkung eines Faktors zu begrenzen
  \item der Schwierigkeit, einen Produktfaktor zu erfüllen, oder
  \item der Notwendigkeit einer allgemeinen Lösung zu globalen
    Anforderungen.
  \end{itemize}
  Dazu entwickelt Ihr Strategien, um mit den identifizierten Problemen
  umzugehen.

  Achtet auch hier darauf, dass die Probleme und Strategien wirklich
  die Architektur betreffen und nicht etwa das Projektmanagement. Die
  Strategien stellen im Prinzip die Designentscheidungen dar. Sie
  sollten also die Erklärung für den konkreten Aufbau der
  verschiedenen Sichten liefern.}


\textit{Beschreibt möglichst mehrere Alternativen und gebt
  an, für welche Ihr Euch letztlich aus welchem Grunde entschieden
  habt. Natürlich müssen die genannten Strategien in den folgenden
  Sichten auch tatsächlich umgesetzt werden!}

\textit{Ein sehr häufiger Fehler ist es, dass SWP-Gruppen
  arbeitsteilig vorgehen: die eine Gruppe schreibt das Kapitel zur
  Analyse von Faktoren und zu den Strategien, die andere Gruppe
  beschreibt die diversen Sichten, ohne dass diese beiden Gruppen sich
  abstimmen. Natürlich besteht aber ein Zusammenhang zwischen den
  Faktoren, Strategien und Sichten. Dieser muss erkennbar sein, indem
  sich die verschiedenen Kapitel eindeutig aufeinander beziehen.}

\section{Konzeptionelle Sicht}
\label{sec:konzeptionell}

{\it Diese Sicht beschreibt das System auf einer hohen Abstraktionsebene,
d.h. mit sehr starkem Bezug zur Anwendungsdomäne und den geforderten
Produktfunktionen und -attributen. Sie legt die Grobstruktur fest,
ohne gleich in die Details von spezifischen Technologien abzugleiten. 
Sie wird in den nachfolgenden Sichten konkretisiert und verfeinert. Die
konzeptionelle Sicht wird mit {UML}-Komponentendiagrammen visualisiert.}

\section{Modulsicht}
\label{sec:modulsicht}

{\it
Diese Sicht beschreibt den statischen Aufbau des Systems mit Hilfe von
Modulen, Subsystemen, Schichten und Schnittstellen. 
Diese Sicht ist hierarchisch, d.h. Module werden in Teilmodule
zerlegt. Die Zerlegung endet bei Modulen, die ein klar umrissenes
Arbeitspaket für eine Person darstellen und in einer Kalenderwoche
implementiert werden können. Die Modulbeschreibung der Blätter dieser
Hierarchie muss genau genug und ausreichend sein, um das Modul 
implementieren zu können.

Die Modulsicht wird durch {UML}-Paket- und Klassendiagramme visualisiert.

Die Module werden durch ihre Schnittstellen beschrieben. 
Die Schnittstelle eines Moduls $M$ ist die Menge aller Annahmen, die
andere Module über $M$ machen dürfen, bzw.\ jene Annahmen, die $M$
über seine verwendeten Module macht (bzw. seine Umgebung, wozu auch
Speicher, Laufzeit etc.\ gehören).
Konkrete Implementierungen dieser Schnittstellen sind das Geheimnis des Moduls
und können vom Programmierer festgelegt werden. Sie sollen hier
dementsprechend nicht beschrieben werden. 

Die Diagramme der Modulsicht sollten die zur Schnittstelle gehörenden Methoden
enthalten. Die Beschreibung der einzelnen Methoden (im Sinne der Schnittstellenbeschreibung)
geschieht allerdings per Javadoc im zugehörigen Quelltext. Das bedeutet, dass Ihr
für alle Eure Module Klassen, Interfaces und Pakete erstellt und sie mit den Methoden der
Schnittstellen verseht. Natürlich noch ohne Methodenrümpfe bzw.\ mit minimalen Rümpfen.
Dieses Vorgehen vereinfacht den Schnittstellenentwurf und stellt Konsistenz sicher.

Jeder Schnittstelle liegt ein
Protokoll zugrunde. Das Protokoll beschreibt die Vor- und
Nachbedingungen der Schnittstellenelemente. Dazu gehören die erlaubten
Reihenfolgen, in denen Methoden der Schnittstelle aufgerufen werden
dürfen, sowie Annahmen über Eingabeparameter und Zusicherungen über
Ausgabeparameter. Das Protokoll von Modulen wird in der Modulsicht beschrieben.
Dort, wo es sinnvoll ist, sollte es mit Hilfe von Zustands- oder
Sequenzdiagrammen spezifiziert werden. Diese sind dann einzusetzen, wenn der
Text allein kein ausreichendes Verständnis vermittelt (insbesondere
bei komplexen oder nicht offensichtlichen Zusammenhängen).

Der Bezug zur konzeptionellen Sicht muss klar ersichtlich sein. Im
Zweifel sollte er explizit erklärt werden. Auch für diese Sicht muss
die Entstehung anhand der Strategien erläutert werden.
}

\section{Datensicht}
\label{sec:datensicht}

{\it Hier wird das der Anwendung zugrundeliegende Datenmodell
  beschrieben. Hierzu werden neben einem erläuternden Text auch ein
  oder mehrere {UML}-Klassendiagramme verwendet. Das hier beschriebene
  Datenmodell wird u.a. jenes der Anforderungsspezifikation enthalten,
  allerdings mit implementierungsspezifischen Änderungen und
  Erweiterungen. Siehe die gesonderten Hinweise.}

\section{Ausführungssicht}
\nurlangversion

\label{sec:ausfuehrung}

{\it
Die Ausführungssicht beschreibt das Laufzeitverhalten. Hier
werden die Laufzeitelemente aufgeführt und beschrieben, welche Module
sie zur Ausführung bringen. Ein Modul kann von mehreren
Laufzeitelementen zur Laufzeit verwendet werden. Die Ausführungssicht
beschreibt darüber hinaus, welche Laufzeitelemente spezifisch
miteinander kommunizieren. Zudem wird bei verteilten Systemen
(z.B. Client-Server-Systeme) dargestellt, welche Module von welchen
Prozessen auf welchen Rechnern ausgeführt werden.}


\section{Zusammenhänge zwischen Anwendungsfällen und Architektur}
\label{sec:anwendungsfaelle}

{\it In diesem Abschnitt sollen Sequenzdiagramme mit Beschreibung(!)
  für \variante{zwei bis drei von Euch ausgewählte
    Anwendungsfälle}{einen von Euch ausgewählten Anwendungsfall}
  erstellt werden. Ein Sequenzdiagramm beschreibt den
  Nachrichtenverkehr zwischen allen Modulen, die an der Realisierung
  des Anwendungsfalles beteiligt sind.  \variante{Wählt die
    Anwendungsfälle so, dass nach Möglichkeit alle Module Eures
    entworfenen Systems in mindestens einem Sequenzdiagramm
    vorkommen. Falls Euch das nicht gelingt, versucht möglichst viele
    und die wichtigsten Module abzudecken.}{Dazu könnt ihr Euch einen
    Anwendungsfall heraussuchen, der möglichst viele Module der
    Architektur abdeckt. In SWP-2 werden wir mehrere Anwendungsfälle
    betrachten und eine umfangreichere Abdeckung der Architektur
    anstreben.} }

\section{Evolution}
\nurlangversion

\label{sec:evolution}

{\it
  Beschreibt in diesem Abschnitt, welche Änderungen Ihr
  vornehmen müsst, wenn sich Anforderungen oder Rahmenbedingungen
  ändern. Insbesondere sollten hierbei die in der
  Anforderungsspezifikation unter "`Ausblick"' bereits genannten
  Punkte behandelt werden.}

\dots


\end{document}


%%% Local Variables: 
%%% mode: latex
%%% mode: reftex
%%% mode: flyspell
%%% ispell-local-dictionary: "de_DE"
%%% TeX-master: t
%%% End: 
